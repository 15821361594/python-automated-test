% This is "sig-alternate.tex" V2.0 May 2012
% This file should be compiled with V2.5 of "sig-alternate.cls" May 2012
%
% This example file demonstrates the use of the 'sig-alternate.cls'
% V2.5 LaTeX2e document class file. It is for those submitting
% articles to ACM Conference Proceedings WHO DO NOT WISH TO
% STRICTLY ADHERE TO THE SIGS (PUBS-BOARD-ENDORSED) STYLE.
% The 'sig-alternate.cls' file will produce a similar-looking,
% albeit, 'tighter' paper resulting in, invariably, fewer pages.
%
% ----------------------------------------------------------------------------------------------------------------
% This .tex file (and associated .cls V2.5) produces:
%       1) The Permission Statement
%       2) The Conference (location) Info information
%       3) The Copyright Line with ACM data
%       4) NO page numbers
%
% as against the acm_proc_article-sp.cls file which
% DOES NOT produce 1) thru' 3) above.
%
% Using 'sig-alternate.cls' you have control, however, from within
% the source .tex file, over both the CopyrightYear
% (defaulted to 200X) and the ACM Copyright Data
% (defaulted to X-XXXXX-XX-X/XX/XX).
% e.g.
% \CopyrightYear{2007} will cause 2007 to appear in the copyright line.
% \crdata{0-12345-67-8/90/12} will cause 0-12345-67-8/90/12 to appear in the copyright line.
%
% ---------------------------------------------------------------------------------------------------------------
% This .tex source is an example which *does* use
% the .bib file (from which the .bbl file % is produced).
% REMEMBER HOWEVER: After having produced the .bbl file,
% and prior to final submission, you *NEED* to 'insert'
% your .bbl file into your source .tex file so as to provide
% ONE 'self-contained' source file.
%
% ================= IF YOU HAVE QUESTIONS =======================
% Questions regarding the SIGS styles, SIGS policies and
% procedures, Conferences etc. should be sent to
% Adrienne Griscti (griscti@acm.org)
%
% Technical questions _only_ to
% Gerald Murray (murray@hq.acm.org)
% ===============================================================
%
% For tracking purposes - this is V2.0 - May 2012

\documentclass{sig-alternate}

\usepackage{code}

\begin{document}
%
% --- Author Metadata here ---
\conferenceinfo{ISSTA}{'15 Baltimore, MA USA}
%\CopyrightYear{2007} % Allows default copyright year (20XX) to be over-ridden - IF NEED BE.
%\crdata{0-12345-67-8/90/01}  % Allows default copyright data (0-89791-88-6/97/05) to be over-ridden - IF NEED BE.
% --- End of Author Metadata ---

\title{Demo: TSTL: a Language and Tool for Testing}
%
% You need the command \numberofauthors to handle the 'placement
% and alignment' of the authors beneath the title.
%
% For aesthetic reasons, we recommend 'three authors at a time'
% i.e. three 'name/affiliation blocks' be placed beneath the title.
%
% NOTE: You are NOT restricted in how many 'rows' of
% "name/affiliations" may appear. We just ask that you restrict
% the number of 'columns' to three.
%
% Because of the available 'opening page real-estate'
% we ask you to refrain from putting more than six authors
% (two rows with three columns) beneath the article title.
% More than six makes the first-page appear very cluttered indeed.
%
% Use the \alignauthor commands to handle the names
% and affiliations for an 'aesthetic maximum' of six authors.
% Add names, affiliations, addresses for
% the seventh etc. author(s) as the argument for the
% \additionalauthors command.
% These 'additional authors' will be output/set for you
% without further effort on your part as the last section in
% the body of your article BEFORE References or any Appendices.

\numberofauthors{1} %  in this sample file, there are a *total*
% of EIGHT authors. SIX appear on the 'first-page' (for formatting
% reasons) and the remaining two appear in the \additionalauthors section.
%
\author{
% You can go ahead and credit any number of authors here,
% e.g. one 'row of three' or two rows (consisting of one row of three
% and a second row of one, two or three).
%
% The command \alignauthor (no curly braces needed) should
% precede each author name, affiliation/snail-mail address and
% e-mail address. Additionally, tag each line of
% affiliation/address with \affaddr, and tag the
% e-mail address with \email.
%
% 1st. author
\alignauthor
Alex Groce, Jervis Pinto, Pooria Azimi, Pranjal Mittal \\
\affaddr{Oregon State University}
\email{agroce@gmail}
}
% There's nothing stopping you putting the seventh, eighth, etc.
% author on the opening page (as the 'third row') but we ask,
% for aesthetic reasons that you place these 'additional authors'
% in the \additional authors block, viz.
\additionalauthors{Additional authors: John Smith (The Th{\o}rv{\"a}ld Group,
email: {\texttt{jsmith@affiliation.org}}) and Julius P.~Kumquat
(The Kumquat Consortium, email: {\texttt{jpkumquat@consortium.net}}).}
\date{30 July 1999}
% Just remember to make sure that the TOTAL number of authors
% is the number that will appear on the first page PLUS the
% number that will appear in the \additionalauthors section.

\maketitle
\begin{abstract}

Writing a test harness is a difficult and repetitive programming task,
and the lack of tool support for customized automated testing is an
obstacle to the adoption of more sophisticated testing in industry.
This paper presents TSTL, a language (and tool) allowing users to
specify the general form of valid tests in a simple but expressive
language.  TSTL is a minimal language, using the language of the
Software Under Test (SUT) to support most operations, but adding
idioms and automatic expansion into an interface for testing.  TSTL,
provides a common testing interface that hides that details of the
SUT, making it possible to write universal testing tools (such as
random testers or model checkers).  TSTL is currently available for
Python, but easily adapted to other languages as well.

\end{abstract}

% A category with the (minimum) three required fields
%\category{H.4}{Information Systems Applications}{Miscellaneous}
%A category including the fourth, optional field follows...
\category{D.2.5}{Software Engineering}{Testing and Debugging}

\terms{Reliability}

\keywords{Domain specific languages, testing tools}

\section{Introduction}

TSTL \cite{NFM15} is a language (and tool) intended to make these
difficulties less onerous.

\begin{code}
@import avl
@import math

@<
def heightOk(tree):
     h = tree.tree\_height()
     l = len(tree.inorder\_traverse())
     if (l == 0):
        return True
     m = math.log(l,2)
     return h <= (m + 1)

 def items(s):
     l = []
     for i in s:
        l.append(i)
     return sorted(l)
@>

source: avl.py

pool: %INT% 4
pool: %AVL% 2 REF
pool: %LIST% 1

log: 3 %AVL%.inorder\_traverse()

property: heightOk(%AVL%) 
property: %AVL%.check\_balanced()

%LIST%:=[]
~%LIST%.append(%INT%) 
%INT%:=%[1..20]%
%AVL%:=avl.AVLTree()
%AVL%:=avl.AVLTree(%LIST%)
~%AVL%.insert(%INT%)
~%AVL%.delete(%INT%)
~%AVL%.find(%INT%)
%AVL%.inorder\_traverse()

reference: avl.AVLTree ==> set
reference: insert ==> add
reference: delete ==> discard
reference: find ==> \_\_contains\_\_
reference: (\\S+)\\.inorder\_traverse\\(\\) ==> items(\\1)

compare: find
compare: inorder\_traverse
\end{code}

\bibliographystyle{abbrv}
\bibliography{bibliography}

\end{document}

