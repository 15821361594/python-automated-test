\section{Related Work}

There is a vast amount of previous work on automated generation of
tests for (API-based) software systems
\cite{Pacheco,FA11,GodefroidKS05} and random testing in particular
\cite{ICSEDiff,Pacheco,AMFL11,ARTChen,ISSTAART,FASE,HamletOnly,Hamlet94,ClaessenH00,CiupaLOM07,RandFormal,WODA08,andrews-etal-rute-rt,ASE08,evalrand,csmith},
some dating back to the early 1980s. It is far beyond the scope of
this paper to explore that literature in detail.  We rely primarily on past
results showing that random testing can be highly effective for
finding software faults in large libraries
\cite{Pacheco,ICSEDiff,RandFormal,CFV08,AMAI}.  To our knowledge, there is almost
no work on automated/random testing for GIS in particular, and very
little work on allowing non-traditional software developers to apply
automated (random) testing to APIs.

The approach to testing ArcPy taken in this paper uses the TSTL tool
for automated testing \cite{NFM15,ISSTA15,tstl}.  TSTL is a
domain-specific language (DSL) \cite{Fow10} for testing, in the spirit
of the famous QuickCheck tool for Haskell \cite{ClaessenH00}; other
DSLs for testing exist \cite{UDITA}, but these are more limited in
scope and function.  As part of the efforts described in this paper,
some of the authors have submitted for review a paper on the
normalization and generalization of test cases in TSTL
\cite{ICSTnorm}.  The quick tests for regression across ArcPy/ArcGIS
versions are based on using cause reduction
\cite{icst2014,stvrcausereduce} to convert long random tests into
short sequences that preserve code coverage, and a large body of
previous work on test case selection and prioritization that indicates
code coverage \cite{ISSTA13,ICSE14Cov} can be a useful measure of test
case effectiveness
\cite{SelectTest,YooHarman,Graves:2001:ESR:367008.367020}.

The literature on testing GIS software in particular seems to consist
of one paper proposing a very limited application of automated testing
to assist GIS users, primarily in model development \cite{GISTest}.
That work does not target the reliability or correctness of the
underlying GIS engine, or GIS libraries.  There is also some
discussion of automated testing for GIS in various blog posts and
discussion groups (e.g., \cite{gisblog1,gisblog2}), but no formal
academic case studies.  These discussions also tend to focus on
application testing or GUI testing, rather than testing of library
code used across GIS applications.

There is also relatively little written about software testing in the
academic literature that focuses on the pitfalls, best practices, and
engineering challenges of testing real-world systems.  The series of
papers by the NASA Jet Propulsion Laboratory on testing the file
systems for the Curiosity Mars Rover is notable, though even there the
focus is more on general technique than details of approach
\cite{ICSEDiff,CFV08,AMAI}.  Literature that at least touches on more
practical aspects, however, is plentiful: e.g., Lei and Andrews
\cite{MinUnit} demonstrated that random test generation essentially
requires test-case minimization \cite{DD} in order to be useful.

There is a significant body of work on end-user testing of software,
part of the larger field of end-user software engineering
\cite{burnettEUSE,Silos}.  End-user software engineering examines how
software can best be produced by developers who do not have a
traditional computer-science background, and are often primarily
interested in an application of programming, rather than software
development as a profession.  GIS developers are a (we believe)
typical example \cite{Segal07}:  they are technically skilled  individuals whose
primary expertise is not in software development, but who, in order to
pursue their goals, must develop, maintain, and test significant
software systems.

The earliest work focusing on software testing for
end-user software engineers explored how to test spreadsheets
\cite{rothermelTOSEM,rothermel2000wysiwyt}.  Other work has focused on
errors end-users make in specifying systems \cite{Phalgune}, and how
end-users of machine learning systems (who may be machine learning
experts, or individuals with no programming knowledge at all) can test
such systems \cite{OnlyOracle,kulesza-eud11,shinsel-vlhcc}.  To our
knowledge, no previous work considers API-sequence testing for
end-users.  Previous work on API testing has largely not even included
traditional software developers, but been performed by software
testing researchers only.

Chen et al. propose metamorphic testing
\cite{MetaTest,isstamorph,metamorph,chentest} as a possible solution
to the oracle problem (defining correct behavior of a software system)
for end-user programmers \cite{MetamorphEndUser}.  In metamorphic
testing, while the correct output of a program is not known, the
effect of certain modifications to an input is known (e.g., increasing
a certain input should increase a certain output), allowing faults to
be detected even without a definition of the correct result.  In this
paper, we largely focus on building a basic framework capable of
extension to more complex oracles.  In the long run, however, more
sophisticated oracles are critical to finding deep faults.
