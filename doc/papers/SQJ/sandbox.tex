\subsection{Sandboxed Test Case Execution}

Previous use of TSTL had been limited to testing systems where failure
resulted in an uncaught exception or a bad return value from a call,
at worst.  With ArcPy, however, it is very common for a failure to
cause a crash, killing not only ArcPy but the Python environment
running the test case.

We added two features to TSTL's test generators to handle this
problem.  First, we modified the random tester to record each action
to a test case log as it was performed in order to recover a crashing
test after the generator terminates.  After further experimentation,
we discovered that recording just the current test case was not
sufficient; some ArcPy failures required maintaining a history of all
executed tests, since the corruption carried across reimports of the
Python module.  Second, we produced a function
to enable running a TSTL case in its own Python subprocess, to allow
test reduction, normalization, and generalization even of crashing
test cases.  It was not neccessary to modify the TSTL interface's
reduction and other functions, as they already took an arbitrary
function as reduction predicate, allowing us to simply produce a
sandboxed version of replay and pass that to TSTL's {\tt reduce}, {\tt
  normalize} etc. calls.