\section{The ArcPy TSTL Test Harness}

Figures \ref{preamble}, \ref{actions1}, and \ref{actions2} show a
version of the actual ArcPy test harness.  This version can the faults
described in this paper, though in practice some faults are easier to
detect than others, and for practical testing it is best to disable,
e.g., the various {\tt Delete} calls.  

\begin{figure}
{\scriptsize
\begin{code}
@import shutil
@import os
@import glob
@import arcpy
@import exceptions
@import gc

<@
def cleanupFiles():
    gc.collect() \# Get rid of cursors
    \# First get rid of modified files
    for l in ["l1","l2","l3"]:
    	arcpy.Delete\_management(l)
    
    for f in glob.glob("C:\\Arctmp\\*"):
        try:
            shutil.rmtree(f)
        except:
            print "UNABLE TO REMOVE:",f
    \# Now remove the old directory
    for i in xrange(0,1000000):
        new\_workspace = "C:\\Arctmp\\workspace." + str(i)
        if not os.path.exists(new\_workspace):
            break             
    print "TESTING USING WORKSPACE",new\_workspace
    \# Now move in fresh copies
    shutil.copytree("C:\\Arcbase",new\_workspace)
    arcpy.env.workspace = new\_workspace
    print sorted(glob.glob(arcpy.env.workspace + "\\*.shp")),
    print sorted(glob.glob(arcpy.env.workspace + "\\*.lyr")),
    print sorted(glob.glob(arcpy.env.workspace + "\\*.gdb"))


def fcsInGdb(gdb):
    old\_workspace = arcpy.env.workspace
    arcpy.env.workspace = gdb
    fcs = []
    for fds in arcpy.ListDatasets('','feature') + ['']:
        for fc in arcpy.ListFeatureClasses('','',fds):
            fcs.append(os.path.join(gdb,fds,fc))
    arcpy.env.workspace = old\_workspace
    return fcs

def report():
    print arcpy.GetMessages()
@>

pool: <shapefilelist> 2
pool: <shapefile> 2 CONST

pool: <gdbfilelist> 2
pool: <gdbfile> 2 CONST

pool: <gdbfeatureclasslist> 2

pool: <featureclass> 4 CONST
pool: <classorlayer> 4 CONST

pool: <classorlayerlist> 2

pool: <spatialref> 2

pool: <prjfilelist> 2
pool: <prjfile> 2 CONST

pool: <transformlist> 2
pool: <transform> 2 CONST

pool: <newlayer> 2 CONST

pool: <layerlist> 2
pool: <layer> 2

pool: <fieldname> 2 CONST
pool: <fieldtype> 2 CONST
pool: <fieldlist> 2
pool: <fieldnamelist> 2

pool: <stattype> 2 CONST
pool: <statfields> 2

pool: <dist> 2 CONST

pool: <sorttype> 2 CONST
pool: <spatialsort> 2 CONST

pool: <sort> 1
pool: <sortlist> 2

pool: <joinattributes> 2

pool: <overlaptype> 2 CONST
pool: <selectiontype> 2 CONST
pool: <op> 2 CONST
pool: <val> 2 CONST
pool: <whereclause> 2 CONST

pool: <errtable> 1 CONST
pool: <polytable> 1 CONST
pool: <stattable> 1 CONST

pool: <insertcursor> 3
pool: <searchcursor> 3
pool: <updatecursor> 3

pool: <irow> 3
pool: <srow> 3
pool: <urow> 3

init: cleanupFiles()

log: 1 arcpy.GetMessages()
\end{code}
}
\caption{ArcPy TSTL test harness definition preamble (pools, functions, logging).}
\label{preamble}
\end{figure}

\begin{figure}
{\scriptsize
\begin{code}
<gdbfilelist> := sorted(glob.glob(arcpy.env.workspace + "\\*.gdb"))
len(<gdbfilelist,1>) >= 1 -> <gdbfile> := <gdbfilelist> [0]
<gdbfilelist> = <gdbfilelist> [1:]

<shapefilelist> := sorted(glob.glob(arcpy.env.workspace + "\\*.shp"))
len(<shapefilelist,1>) >= 1 -> <shapefile> := <shapefilelist> [0]
<shapefilelist> = <shapefilelist> [1:]

<layerfile> := arcpy.env.workspace + "\\new1.lyr"
<layerfile> := arcpy.env.workspace + "\\new2.lyr"
<layerfile> := arcpy.env.workspace + "\\new3.lyr"

<shapefile> := arcpy.env.workspace + "\\new1.shp"
<shapefile> := arcpy.env.workspace + "\\new2.shp"
<shapefile> := arcpy.env.workspace + "\\new3.shp"

<prjfilelist> := sorted(glob.glob("C:\\Program Files (x86)\\ArcGIS\\Desktop10.3\\Reference Systems\\*.prj"))
len(<prjfilelist,1>) >= 1 -> <prjfile> := <prjfilelist> [0]
<prjfilelist> = <prjfilelist> [1:]

<transformlist> := arcpy.ListTransformations(<spatialref>,<spatialref>)
<transformlist> = <transformlist> [1:]
len(<transformlist,1>) >= 1 -> <transform> := <transformlist> [0]

<newlayer> := "l1"
<newlayer> := "l2"
<newlayer> := "l3"

<spatialref> := arcpy.SpatialReference(<prjfile>)

<gdbfeatureclasslist> := fcsInGdb(<gdbfile>)
<gdbfeatureclasslist> = <gdbfeatureclasslist>[1:]

<featureclass> := <shapefile>
len(<gdbfeatureclasslist,1>) >= 1 -> <featureclass> := <gdbfeatureclasslist>[0]

<classorlayer> := <featureclass>
<classorlayer> := <newlayer>

<classorlayerlist> := []
<classorlayerlist>.append(<classorlayer>)

<fieldtype> := "TEXT"
<fieldtype> := "FLOAT"
<fieldtype> := "DOUBLE"
<fieldtype> := "SHORT"
<fieldtype> := "LONG"
<fieldtype> := "DATE"

<fieldname> := "newf1"
<fieldname> := "newf2"
<fieldname> := "newf3"

<dist> := "100 Feet"
<dist> := "500 Feet"
<dist> := "1000 Feet"
<dist> := "1 Mile"
<dist> := "2 Miles"

<joinattributes> := "ALL"
<joinattributes> := "NO\_FID"
<joinattributes> := "ONLY\_FID"

<overlaptype> := "INTERSECT"
<overlaptype> := "CONTAINS"
<overlaptype> := "COMPLETELY\_CONTAINS"
<overlaptype> := "WITHIN"
<overlaptype> := "SHARE\_A\_LINE\_SEGMENT\_WITH"
<overlaptype> := "CROSSED\_BY\_THE\_OUTLINE\_OF"

<selectiontype> := "NEW\_SELECTION"
<selectiontype> := "ADD\_TO\_SELECTION"
<selectiontype> := "REMOVE\_FROM\_SELECTION"
<selectiontype> := "SUBSET\_SELECTION"
<selectiontype> := "SWITCH\_SELECTION"
<selectiontype> := "CLEAR\_SELECTION"

<sorttype> := "ASCENDING"
<sorttype> := "DESCENDING"

<spatialsort> := "UR"
<spatialsort> := "UL"
<spatialsort> := "LR"
<spatialsort> := "LL"
<spatialsort> := "PEANO"

<sort> := [<fieldname>,<sorttype>]
<sortlist> := []
<sortlist>.append(<sort>)

\{IOError\} <insertcursor> := arcpy.InsertCursor(<classorlayer>)
\{IOError\} <insertcursor> := arcpy.InsertCursor(<classorlayer>,<spatialref>)
\{IOError\} <searchcursor> := arcpy.SearchCursor(<classorlayer>)
\{IOError,exceptions.RuntimeError\} <searchcursor> := arcpy.SearchCursor(<classorlayer>,<whereclause>)
{IOError,exceptions.RuntimeError} <searchcursor> := arcpy.SearchCursor(<classorlayer>,<whereclause>,<spatialref>)
\{IOError\} <updatecursor> := arcpy.UpdateCursor(<classorlayer>)
\{IOError\} <updatecursor> := arcpy.UpdateCursor(<classorlayer>,<spatialref>)

<irow> := <insertcursor>.newRow()
\{exceptions.RuntimeError\} <insertcursor>.insertRow(<irow>)

<irow> := <insertcursor>.next()
<urow> := <updatecursor>.next()
<srow> := <searchcursor>.next()

\{exceptions.RuntimeError\} <val> := <irow>.getValue(<fieldname>)
\{exceptions.RuntimeError\} <val> := <srow>.getValue(<fieldname>)
\{exceptions.RuntimeError\} <val> := <urow>.getValue(<fieldname>)

\{exceptions.RuntimeError\} <irow>.setValue(<fieldname>,<val>)

\{exceptions.RuntimeError\} <urow>.setNull(<fieldname>)
\{exceptions.RuntimeError\} <urow>.setValue(<fieldname>,<val>)
\{exceptions.RuntimeError\} <updatecursor>.deleteRow(<urow>)
\{exceptions.RuntimeError\} <updatecursor>.updateRow(<urow>)

\end{code}
}
\caption{ArcPy TSTL test harness actions, part 1.}
\label{actions1}
\end{figure}

\begin{figure}
{\scriptsize
\begin{code}
<op> := ">"
<op> := "<"
<op> := "<="
<op> := ">="
<op> := "="
<op> := "!="

<val> := "10"
<val> := "20"
<val> := "30"
<val> := "100"
<val> := "1000"

<whereclause> := '"' + <fieldname> + '" ' + <op> + str(<val>)

<whereclause> := <whereclause> + ' AND ' + <whereclause>

<whereclause> := <whereclause> + ' OR ' +  <whereclause>

<whereclause> := 'NOT' + <whereclause>

<errtable> := arcpy.env.workspace + "\\geomerr.dbf"

<polytable> := arcpy.env.workspace + "\\polyneig.dbf"

<stattable> := arcpy.env.workspace + "\\stats.dbf"

\{IOError\} <fieldlist> := arcpy.ListFields(<classorlayer>)
len(<fieldlist,1>) >= 1 -> <fieldname> := <fieldlist> [0].name
<fieldlist> = <fieldlist> [1:]

<fieldnamelist> := []
<fieldnamelist>.append(<fieldname>)

<stattype> := "SUM"
<stattype> := "MEAN"
<stattype> := "MIN"
<stattype> := "MAX"
<stattype> := "RANGE"
<stattype> := "STD"
<stattype> := "COUNT"
<stattype> := "FIRST"
<stattype> := "LAST"

<statfields> := []
<statfields>.append([<fieldname>,<stattype>])

\{arcpy.ExecuteError\} arcpy.MakeFeatureLayer\_management(<featureclass>,<newlayer>); report()

\{arcpy.ExecuteError\} arcpy.MakeFeatureLayer\_management(<featureclass>,<newlayer>,where\_clause=<whereclause>); report()

\{arcpy.ExecuteError\} arcpy.Project\_management(<featureclass>,<featureclass>,<spatialref>,<transform>); report()

\{arcpy.ExecuteError\} arcpy.AddField\_management(<featureclass>,<fieldname>,<fieldtype>); report()

\{arcpy.ExecuteError\} arcpy.DeleteField\_management(<featureclass>,<fieldname>); report()

\{arcpy.ExecuteError\} arcpy.Buffer\_analysis(<classorlayer>,<featureclass>,<dist>); report()

\{arcpy.ExecuteError\} arcpy.Buffer\_analysis(<classorlayer>,<featureclass>,<dist>,dissolve\_option="ALL"); report()

\{arcpy.ExecuteError\} arcpy.Buffer\_analysis(<classorlayer>,<featureclass>,<dist>,dissolve\_option="LIST",
   dissolve\_field=<fieldnamelist>); report()

\{arcpy.ExecuteError\} arcpy.Erase\_analysis(<classorlayer>,<classorlayer>,<featureclass>); report()

\{arcpy.ExecuteError\} arcpy.Erase\_analysis(<classorlayer>,<classorlayer>,<featureclass>,
   cluster\_tolerance=<dist>); report()

\{arcpy.ExecuteError\} arcpy.Intersect\_analysis(<classorlayerlist>,<featureclass>); report()

\{arcpy.ExecuteError\} arcpy.Intersect\_analysis(<classorlayerlist>,<featureclass>,join\_attributes=<joinattributes>);
   report()

\{arcpy.ExecuteError\} arcpy.Intersect\_analysis(<classorlayerlist>,<featureclass>,cluster\_tolerance=<dist>); report()

\{arcpy.ExecuteError\} arcpy.Intersect\_analysis(<classorlayerlist>,<featureclass>,join\_attributes=<joinattributes>,
   cluster\_tolerance=<dist>); report()

\{arcpy.ExecuteError\} arcpy.Union\_analysis(<classorlayerlist>,<featureclass>); report()

\{arcpy.ExecuteError\} arcpy.Union\_analysis(<classorlayerlist>,<featureclass>,join\_attributes=<joinattributes>); report()

\{arcpy.ExecuteError\} arcpy.Union\_analysis(<classorlayerlist>,<featureclass>,cluster\_tolerance=<dist>); report()

\{arcpy.ExecuteError\} arcpy.Union\_analysis(<classorlayerlist>,<featureclass>,join\_attributes=<joinattributes>,
   cluster\_tolerance=<dist>); report()

\{arcpy.ExecuteError\} arcpy.SpatialJoin\_analysis(<classorlayer>,<classorlayer>,<featureclass>); report()

\{arcpy.ExecuteError\} arcpy.SymDiff\_analysis(<classorlayer>,<classorlayer>,<featureclass>); report()

\{arcpy.ExecuteError\} arcpy.SymDiff\_analysis(<classorlayer>,<classorlayer>,<featureclass>,
   join\_attributes=<joinattributes>); report()

\{arcpy.ExecuteError\} arcpy.SymDiff\_analysis(<classorlayer>,<classorlayer>,<featureclass>,
   join\_attributes=<joinattributes>,cluster\_tolerance=<dist>); report()

\{arcpy.ExecuteError\} arcpy.PolygonNeighbors\_analysis(<classorlayer>,~<polytable>); report()

\{arcpy.ExecuteError\} arcpy.Statistics\_analysis(<classorlayer>,~<stattable>,<statfields>); report()

\{arcpy.ExecuteError\} arcpy.SelectLayerByLocation\_management(<newlayer>,select\_features=<newlayer>,
   overlap\_type=<overlaptype>)

\{arcpy.ExecuteError\} arcpy.SelectLayerByLocation\_management(<newlayer>,select\_features=<newlayer>,
   overlap\_type=<overlaptype>,search\_distance=<dist>)

\{arcpy.ExecuteError\} arcpy.SelectLayerByLocation\_management(<newlayer>,select\_features=<newlayer>,
   overlap\_type=<overlaptype>,search\_distance=<dist>,selection\_type=<selectiontype>)

\{arcpy.ExecuteError\} arcpy.SelectLayerByAttribute\_management(<newlayer>,selection\_type=<selectiontype>,
   where\_clause=<whereclause>)

\{arcpy.ExecuteError\} arcpy.Select\_analysis(<classorlayer>,<featureclass>,where\_clause=<whereclause>)

\{arcpy.ExecuteError\} arcpy.CopyFeatures\_management(<featureclass>,<featureclass>); report()

\{arcpy.ExecuteError\} arcpy.Sort\_management(<featureclass>,<featureclass>,<sortlist>); report()

\{arcpy.ExecuteError\} arcpy.Sort\_management(<featureclass>,<featureclass>,<sortlist>,<spatialsort>); report()

\{arcpy.ExecuteError\} arcpy.Sort\_management(<featureclass>,<featureclass>,[["Shape",<sorttype>]],<spatialsort>); report()

\{arcpy.ExecuteError\} arcpy.CheckGeometry\_management(<classorlayer>,~<errtable>); report()

\{arcpy.ExecuteError\} arcpy.CheckGeometry\_management(<classorlayerlist>,~<errtable>); report()

\{arcpy.ExecuteError\} arcpy.Delete\_management(<featureclass>); report()
\end{code}
}
\caption{ArcPy TSTL test harness definition actions, part 2.}
\label{actions2}
\end{figure}