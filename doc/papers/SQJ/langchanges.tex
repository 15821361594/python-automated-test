\section{Language Changes}

The original syntax for TSTL \cite{NFM15} used the {\tt \%} sign to
indicate TSTL constructs and pool variables.  Unfortunately, this
produced code that was difficult to read.  The first author suggested
using a notation more similar to that used to describe the grammar of
TSTL itself, enclosing pools in angle brackets.  Because C++ and a few
other languages use angle brackets for other purposes, and in order to
avoid breaking old harnesses, TSTL continues to allow the {\tt \%}
notation, but future TSTL harnessess for Python will use the more
readable syntax.

We are also considering, as a result of the experience of developing
the ArcPy harness, moving to a more structured form for TSTL files.
The current language allows pool definitions, actions, logging code,
raw Python code, and all other TSTL elements to be freely mixed,
without any requirements as to order.  Each line must indicate if it
is not an action definition, with some prefix such as {\tt pool:},
{\tt logging:}, {\tt reference:}, etc.; in practice, however, TSTL
harnessess are always written in an ordered style, with raw code
first, then pool definitions, properties, and logging information,
followed by a long section of action definitions.  Enforcing this
would allow all pool declarations to be prefaced by a single {\tt
  pool:} line at the beginning of the pool definitions, raw Python
code to be contained in a section marked{\tt raw:}, and all other
non-action declarations to be handled in the same way. 

There is also be a need for richer structure to avoid repeated
elements in action definitions.  For example, in the TSTL harness, 36
actions allow the {\tt arcpy.ExecuteError} exception to be raised,
which has to be stated for every individual action, and to avoid some
faults a large number of actions may eventually be disallowed for
feature classes or layers with active cursors.  Introducing
nested action groups, which can share guards, allowed exceptions, and post-conditions
could make reading complex TSTL code easier.  We are currently working to
define these language changes, without breaking existing TSTL code.
Discovering the need for this kind of feature without testing a system
as complex as ArcPy would be difficult.
