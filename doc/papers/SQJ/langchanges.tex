\section{Language Changes}

The original syntax for TSTL \cite{NFM15} used the {\tt \%} sign to
indicate TSTL constructs and pool variables.  For example, the
assignment of range values to the integer pool in the AVL example
would read:  {\tt \%INT\% := \%[1..20]\%}.  This syntax
produced code that was difficult to read and (in our opinion) unattractive.  The first
author suggested using a notation more similar to that used to
describe the grammar of TSTL itself, enclosing pools in angle
brackets.  On examination, this syntax better reflects the nature of
TSTL pools, since it resembles a BNF (Backus-Naur Form) grammar, and
pools in actions are
conceptually more similar to a grammar non-terminals than simple variables.  Because
C++ and a few other languages use angle brackets for other purposes,
and in order to avoid breaking old harnesses, TSTL continues to allow
the {\tt \%} notation, but future TSTL harnessess for Python will use
the more readable syntax.

Another language improvement inspited by development of the ArcPy harness was
extending the range construct {\tt <[start..end]>} to also express a
list of options, which can be constants or Python expressions.  An
action containing {\tt <[item1, item2, ...]>} expands to multiple
actions, each of which has the literal text of one item.  The
construct {\tt <,[item1,, item2,, ...],>} works the same way, except
that items are delimited by double-commas, allowing the items to even
be partial Python expressions containing commas, for use in,
e.g. constructing function arguments.  The second form is not used in
the ArcPy harness at this time, because it is powerful but somewhat
difficult to read.  In some cases, using multiple lines to define
actions that, in theory, could be handled with a single line is best
for readability reasons (see Section \ref{harness}).

We are also considering, as a result of the experience of developing
the ArcPy harness, moving to a more structured form for TSTL harness definitions.
The current language allows pool definitions, actions, logging code,
raw Python code, and all other TSTL elements to be freely mixed,
without any requirements as to order.  Each line must indicate if it
is not an action definition, with some prefix such as {\tt pool:},
{\tt logging:}, {\tt reference:}, etc.; in practice, however, TSTL
harnessess are always written in an ordered style, with raw code
first, then pool definitions, properties, and logging information,
followed by a long section of action definitions.  Enforcing this
would allow all pool declarations to be prefaced by a single {\tt
  pool:} line at the beginning of the pool definitions, raw Python
code to be contained in a section marked{\tt raw:}, and all other
non-action declarations to be handled in the same way. 

There is also a need for richer structure to avoid repeated
elements in action definitions.  For example, in the TSTL harness, 36
actions allow the {\tt arcpy.ExecuteError} exception to be raised,
which has to be stated for every individual action, and to avoid some
faults a large number of actions may eventually be disallowed for
feature classes or layers with active cursors\footnote{The reader
  not familiar with GIS terminology is directed to the ArcPy
  documentation \cite{ArcPy} and Esri's GIS dictionary \cite{GISDict}.}.  Introducing nested
action groups, which can share guards, allowed exceptions, and
post-conditions could make reading complex TSTL code easier.  We are
currently working to define these language changes in a way that does
not break existing TSTL code.  Discovering the need for this kind of
feature without testing a system as complex as ArcPy would be
difficult.
