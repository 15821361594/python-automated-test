\subsection{Regression Generation}

One difficulty for ArcPy users is ensuring that their existing scripts
and tools work on new versions of ArcGIS.  Each recent major release
(10.2 and 10.3) after ArcPy's introduction has produced some changes
in the behavior of API calls.  Detecting when such changes cause a
script to break is difficult.  A first step would be an automatic way
to find when the return values for calls differ between ArcPy
versions.
Because installing multiple versions of ArcGIS on the same system is
difficult or impossible, our method for finding differences relies on
choosing a reference version (10.3 in our current efforts), and
generating a set of standalone tests that 1) cover a large amount of
ArcPy functionality, including invalid inputs to functions and 2)
record the return values and exceptions raised by calls.  These tests
can be run on any ArcPy version, and will report differences between
the tests and version 10.3.

We generate the tests using an approach called \emph{quick testing}
\cite{icst2014,stvrcausereduce}, which takes a set of tests produced
by random testing, and applies a test case reduction algorithm
\cite{DD} to produce short tests that have the same code coverage as
the very large, highly redundant, original set of test cases.
Automatic quick-testing was added to TSTL's random test generator to
support ArcPy testing.  Combined with standalone test generation, this
allowed us to produce test cases that can be run on any version of
ArcPy, and explore a large variety of behavior of the code.  Coverage
alone, however, unlike previous quick testing efforts, is insufficient
to ensure a useful regression test.  Because coverage only considers
the Python behavior of ArcPy (since we do not have access to the
source for the ArcGIS engine), it may group behaviors that are not
similar together.  We added the ability to combine coverage
preservation with preservation of all ArcPy messages indicating a
successful GIS engine operation, after abstracting away such details
as the runtime of the operation, and so forth.

However, just producing these coverage-and-engine-behavior preserving
standalone tests is not sufficient for good version comparison, since
standalone test cases as produced only check for properties defined in TSTL.  An
additional option was added to the standalone test generator, allowing
it to record the actual return values of all calls, the set of
exceptions thrown, and so forth to more precisely record a test's
behavior on an ArcPy version.
