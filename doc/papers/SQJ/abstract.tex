Geographic information systems are widely used for a variety of
analytical tasks, including conservation and environmental efforts,
epidemiology, defense and security, demographic analysis, and urban
planning.  ESRI's ArcGIS platform has the largest market share in the
field.  GIS analysis often requires performing complex analytical
tasks, so users often write scripts to automate these tasks, or even
to add new functionalities to the ESRI software's GUI.  Recent work
has shown that even libraries as widely used as Apache and the Java
language core can have significant faults that elude detection until
automated testing is applied.  Unfortunately,  
ArcGIS has numerous faults (and un-documented usage requirements for
APIs).  Behavior of calls also changes with new releases of the
software, breaking existing code in hard-to-detect ways.

In this paper, we present an approach allowing GIS users, who are not
traditonal software developers (but experienced with programming GIS
systems), to find faults, produce regression suites to detect changes
in API behavior between versions, and otherwise explore the
functionality of arcpy, ESRI's Python scripting interface to ArcGIS.
The starting point for the effort is a declarative language for
defining the valid tests of a system as a graph to be explored, TSTL
(the Template Scripting Testing Language).  In order to handle the
complexities of a large, commercial system like arcpy, and make TSTL
usable by GIS developers, numerous extensions, some novel, were
required, including some core language changes.  We hope this case
study paves the way for bringing automated testing to a wider
audience of users.
