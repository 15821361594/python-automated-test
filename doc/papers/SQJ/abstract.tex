Geographic Information System (GIS) tools are widely used for a variety of
analytical tasks, including conservation and environmental efforts,
epidemiology, defense and security, demographic analysis, and urban
planning.  Esri's ArcGIS platform has the largest market share in the
field.  GIS analysis often requires performing complex analytical
tasks, so users often write scripts to automate these tasks, or even
to add new functionality to the Esri software's GUI.  Recent work
has shown that even libraries as widely used as Apache and the Java
language core can have significant faults that elude detection until
automated testing is applied.  Unfortunately,  
ArcGIS has numerous faults (and un-documented usage requirements for
APIs).  Behavior of calls can change with new releases of the
software, breaking user code in hard-to-detect ways.

In this paper, we present an approach allowing GIS users, who are not
traditional software developers (but experienced with programming in a
GIS context), to find faults, produce regression suites for API
behavior, and otherwise explore the functionality of ArcPy, Esri's
Python scripting interface to ArcGIS.  The starting point for the
effort is a declarative language for defining the valid tests of a
system as a graph to be explored, TSTL (the Template Scripting Testing
Language).  In order to handle the complexities of a large, commercial
system like ArcPy, and make TSTL usable by GIS developers, numerous
extensions, some novel, were required, including some core language
changes.  We hope this case study paves the way for bringing automated
testing to a wider audience of users.
