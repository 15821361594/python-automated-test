\section{A Brief Primer on TSTL}

\begin{figure}
{\scriptsize
\begin{code}
$_{01}$ @import avl
$_{02}$ @import math
\vspace{0.1in}
$_{03}$ <@
$_{04}$ def it(s):
$_{05}$     l = []
$_{06}$     for i in s:
$_{07}$        l.append(i)
$_{08}$     return sorted(l)
$_{09}$ @>
\vspace{0.1in}
$_{10}$ source: avl.py
\vspace{0.1in}
$_{11}$ pool: <int> 4 CONST
$_{12}$ pool: <avl> 2 REF
$_{13}$ pool: <list> 2
\vspace{0.1in}
$_{14}$ log: 1 <avl>.inorder()
\vspace{0.1in}
$_{15}$ property: <avl>.check\_balanced()
\vspace{0.1in}
$_{16}$ <list>:=[]
$_{17}$ ~<list>.append(<int>) 
$_{18}$ <int>:= <[1..20]>
\vspace{0.1in}
$_{19}$ <avl>:=avl.AVLTree()
$_{20}$ <avl>:=avl.AVLTree(<list>)
\vspace{0.1in}

$_{21}$ ~<avl>.insert(<int>) => (len(<avl,1>.inorder()) == pre<(len(<avl,1>.inorder()))>+1) 
  or pre<(<avl,1>.find(<int,1>))>
$_{22}$ ~<avl>.delete(<int>) => (len(<avl,1>.inorder()) == pre<(len(<avl,1>.inorder()))>-1) 
  or not pre<((<avl,1>.find(<int,1>)))>
$_{23}$ ~<avl>.find(<int>)
$_{24}$ <avl>.inorder()
$_{25}$ len(<avl,1>.inorder()) > 5 -> <avl>.display()
\vspace{0.1in}
$_{26}$ reference: avl.AVLTree ==> set
$_{27}$ reference: insert ==> add
$_{28}$ reference: delete ==> discard
$_{29}$ reference: find ==> \_\_contains\_\_
$_{30}$ reference: METHOD(inorder) ==> CALL(it)
$_{31}$ reference: METHOD(display) ==> CALL(print)
\vspace{0.1in}
$_{32}$ compare: find
$_{33}$ compare: inorder
\end{code}
}
\caption{Part of a TSTL definition of AVL tree tests.}
\label{fig:example}
\end{figure}


\begin{figure}
{\scriptsize 
\begin{code}
avl1 = avl.AVLTree()  
int3 = 10  
int1 = 11  
avl1.insert(int1) 
int1 = 1  
avl1.insert(int3) 
avl1.insert(int1) 
int3 = 9  
avl1.insert(int3) 
int2 = 11  
avl1.delete(int2) 
\end{code}
}
\caption{An example TSTL-produced test}
\label{fig:avlrun}
\end{figure}


TSTL \cite{NFM15,ISSTA15,tstl} is a language for defining the
structure of test cases (usually API-call sequences, but also
grammar-based tests using string construction), and a set of tools for
use in generating, manipulating, and understanding those test cases.
Figure \ref{fig:example} shows a TSTL definition of tests (known as a
harness definition, or \emph{harness} for short) for a Python class
implementing AVL trees, in the latest syntax for TSTL (modified in the
course of the work described in this paper).  Given a harness like the
one in Figure \ref{fig:example}, TSTL compiles it into a Python file
defining a class that gives an interface for testing.  The class
interface provides features such as
querying the set of available testing actions, restarting a test,
replaying a test, collecting data about which lines and branches in
the source code have been executed by each test, and other commonly
needed testing features.  The TSTL release \cite{tstl} provides
testing tools that use the interface for test generation and
debugging.

A TSTL test harness defines a set of \emph{pools} that hold values
produced and used during testing.  Pools \cite{AndrewsTR} are a common
approach to defining API-testing sequences.  It also defines a set of
actions that are possible during testing, typically API calls and
assignments to pool values.  In this example, there are three pools,
{\tt int}, {\tt list}, and {\tt avl}.  There are four instances of the
{\tt int} pool, which means that a test in progress can store up to 4
{\tt int}s at one time (in variables named {\tt int0}, {\tt int1},
{\tt int2}, and {\tt int3}), and three instances of the {\tt avl}
pool.  The actions defined here include setting the value of an {\tt int}
pool to any integer in the range 1-20 inclusive, setting the value of
an {\tt avl} pool to a newly constructed AVL tree, and calling an AVL
tree's {\tt insert}, {\tt delete}, {\tt find} and {\tt inorder}
methods.  Figure \ref{fig:avlrun} shows a valid test case produced by
running a random test generator on the TSTL-compiled interface
produced by this definition.  TSTL automatically enforces the requirement that
tests are well-formed: for example, no pool instance (such as {\tt
  avl1} can appear in an action until it has been assigned a value),
and no pool instance that has been assigned a value can be assigned a
different value until it has been used in an action, to avoid
degenerate sequences such as {\tt int3 = 10} followed by {\tt int3 =
  4}.  Each action in a test case is called a ``step'' --- the first
step of the first test case in Figure \ref{fig:avlrun} is storing a
new AVL tree in {\tt avl1}.

Figure \ref{fig:example} shows some additional features of TSTL used
in ArcPy testing.  The syntax {\tt guard -> action} allows {\tt
  action} to take place only if {\tt guard} is an expression that
evaluates to true.  In the example, only AVL trees of size 5 or larger
are displayed.  One feature that is now shown is the syntax for
allowing an action to throw an exception without stopping
testing. This construct, {\tt \{exception [, exception2, ...]\}
  action}, is used many times in the TSTL harness, because operations
in ArcPy tend to signal failure by throwing an exception.  Because
thus far our testing has been limited to checking for crash failures
and comparing outputs between versions, we have not needed the TSTL
features for checking an assertion after an action ({\tt action =>
  pred}, but expect to use it in future testing, as discussed in
Section \ref{conclusion}.  The final feature of interest here is Section \ref{sec:reftest} discusses the
possibility of using differential/reference testing
\cite{McKeeman,ICSEDiff} in ArcPy testing.  

