\section{A Brief Primer on TSTL}

\begin{figure}
{\scriptsize
\begin{code}
$_{01}$ @import avl
$_{02}$ @import math
\vspace{0.1in}
$_{03}$ <@
$_{04}$ def it(s):
$_{05}$     l = []
$_{06}$     for i in s:
$_{07}$        l.append(i)
$_{08}$     return sorted(l)
$_{09}$ @>
\vspace{0.1in}
$_{10}$ source: avl.py
\vspace{0.1in}
$_{11}$ pool: <int> 4 CONST
$_{12}$ pool: <avl> 2 REF
$_{13}$ pool: <list> 2
\vspace{0.1in}
$_{14}$ log: 1 <avl>.inorder()
\vspace{0.1in}
$_{15}$ property: <avl>.check\_balanced()
\vspace{0.1in}
$_{16}$ <list>:=[]
$_{17}$ ~<list>.append(<int>) 
$_{18}$ <int>:= <[1..20]>
\vspace{0.1in}
$_{19}$ <avl>:=avl.AVLTree()
$_{20}$ <avl>:=avl.AVLTree(<list>)
\vspace{0.1in}

$_{21}$ ~<avl>.insert(<int>) => (len(<avl,1>.inorder()) == pre<(len(<avl,1>.inorder()))>+1) 
  or pre<(<avl,1>.find(<int,1>))>
$_{22}$ ~<avl>.delete(<int>) => (len(<avl,1>.inorder()) == pre<(len(<avl,1>.inorder()))>-1) 
  or not pre<((<avl,1>.find(<int,1>)))>
$_{23}$ ~<avl>.find(<int>)
$_{24}$ <avl>.inorder()
$_{25}$ len(<avl,1>.inorder()) > 5 -> <avl>.display()
\vspace{0.1in}
$_{26}$ reference: avl.AVLTree ==> set
$_{27}$ reference: insert ==> add
$_{28}$ reference: delete ==> discard
$_{29}$ reference: find ==> \_\_contains\_\_
$_{30}$ reference: METHOD(inorder) ==> CALL(it)
$_{31}$ reference: METHOD(display) ==> CALL(print)
\vspace{0.1in}
$_{32}$ compare: find
$_{33}$ compare: inorder
\end{code}
}
\caption{Part of a TSTL definition of AVL tree tests.}
\label{fig:example}
\end{figure}


\begin{figure}
{\scriptsize 
\begin{code}
avl1 = avl.AVLTree()  
int3 = 10  
int1 = 11  
avl1.insert(int1) 
int1 = 1  
avl1.insert(int3) 
avl1.insert(int1) 
int3 = 9  
avl1.insert(int3) 
int2 = 11  
avl1.delete(int2) 
\end{code}
}
\caption{An example TSTL-produced test.}
\label{fig:avlrun}
\end{figure}


TSTL \cite{NFM15,ISSTA15,tstl} is a language for defining the
structure of test cases (usually API-call sequences, but also
grammar-based tests using string construction), and a set of tools for
use in generating, manipulating, and understanding those test cases.
Figure \ref{fig:example} shows a TSTL definition of tests (known as a
harness definition, or \emph{harness} for short) for a Python class
implementing AVL trees\footnote{AVL trees, named after Georgy
  Adelson-Velsky and Evgenii Landis \cite{AVL} are balanced tree
  structures similar to red-black trees; they are also frequently used as
  examples in software testing \cite{FASE,ISSRE}.} \cite{avltree}, in the latest syntax for TSTL (modified in the
course of the work described in this paper).  Given a harness like the
one in Figure \ref{fig:example}, TSTL compiles it into a Python file
defining a class that gives an interface for testing.  The class
interface provides features such as
querying the set of available testing actions, restarting a test,
replaying a test, collecting data about which lines and branches in
the source code of the SUT have been executed by each test, and other commonly
needed testing features.  The TSTL release \cite{tstl} provides
testing tools that use this interface for test generation and
debugging.

A TSTL test harness defines a set of \emph{pools} that hold values
produced and used during testing.   Using pools \cite{AndrewsTR} is a common
approach to defining API-testing sequences.  The harness also defines a set of
actions that are possible during testing, typically API calls and
assignments to pool values.  Code marked off with the {\tt @}
symbol (lines 1-9) is raw Python code.  The ability to call
arbitrary code in Python makes TSTL extremely expressive.  Other
elements of the ``preamble'' (before action definitions) are the
indication of the source files to be tested (line 10), the information
on how to produce logs of test activity for debugging (line 14), and a
correctness property to check after every test action (line 15).

In this example, there are three pools,
{\tt int}, {\tt list}, and {\tt avl}.  There are four instances of the
{\tt int} pool, which means that a test in progress can store up to 4
{\tt int}s at one time (in variables named {\tt int0}, {\tt int1},
{\tt int2}, and {\tt int3}), and two instances of the {\tt avl} and
{\tt list}
pools.  The actions defined here include setting the value of an {\tt int}
pool to any integer in the range 1-20 inclusive (line 18), setting the value of
an {\tt avl} pool to a newly constructed AVL tree (line 19), and calling an AVL
tree's {\tt insert}, {\tt delete}, {\tt find} and {\tt inorder}
methods (lines 21-24).  One line of TSTL typically defines more than one action. For
example, the line of TSTL code {\tt <avl>.find(<int>)} defines 8 actions, one
for each choice of {\tt avl} and {\tt int} pool instance:  the action set
includes {\tt avl0.find(int0)}, {\tt avl1.find(int0)}, {\tt
  avl0.find(int1)}, and so forth.  


Figure \ref{fig:avlrun} shows a valid test case produced by
running a random test generator on the TSTL-compiled interface
produced by this definition.  TSTL automatically enforces the requirement that
tests are well-formed: for example, no pool instance (such as {\tt
  avl1} can appear in an action until it has been assigned a value),
and no pool instance that has been assigned a value can be assigned a
different value until it has been used in an action, to avoid
degenerate sequences such as {\tt int3 = 10} followed by {\tt int3 =
  4}.  Each action in a test case is called a ``step'' --- the first
step of the first test case in Figure \ref{fig:avlrun} is storing a
new AVL tree in {\tt avl1}.  Prefacing a ``use'' of a pool with a
tilde {\tt ~} indicates that use does not allow the pool to be
re-initialized; in this example, an AVLtree object cannot be replaced
with a new tree until it has been displayed or traversed.

Figure \ref{fig:example} shows some additional features
of TSTL.  The syntax {\tt guard -> action} allows {\tt action} to take
place only if {\tt guard} is an expression that evaluates to true (see
line 25).  In the example, only AVL trees of size 5 or larger are
displayed.  One feature that is not shown is the syntax for allowing
an action to throw an exception without stopping testing. This
construct, {\tt \{exception [, exception2, ...]\} action}, is used
many times in the TSTL harness, because operations in ArcPy tend to
signal failure by throwing an exception.

Because thus far our testing
has been limited to checking for crash failures and comparing outputs
between versions, we have not needed the TSTL feature for checking an
assertion after an action ({\tt action => pred}, shown on lines 21 and
22), but we expect to use it in future testing, as discussed in Section
\ref{futureoracle}.  The construct {\tt <avl,1>} is a pool use that
always expands to the same pool variable as the indexed occurrence of
the pool in the current action.  The final feature of interest here is the full
support for reference testing.  Marking a pool as {\tt REF} (line 12) creates a
second copy of each pool variable, and transforms the operations
performed on the copy according to a reference mapping, as given in
lines 26-31 of the example code.  In the example, AVL trees are
checked for behavioral equivalence to Python sets.  Lines 32 and 33
also indicate the return values from {\tt find} and {\tt inorder}
should be compared to their reference equivalents.  Section
\ref{sec:reftest} discusses the possibility of using
differential/reference testing \cite{Differential,ICSEDiff} in ArcPy
within a single version.