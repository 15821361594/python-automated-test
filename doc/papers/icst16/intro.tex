\section{Introduction}

It has long been understood that effective automated testing requires
test case reduction \cite{DD,MinUnit,ICSEDiff} to produce test cases
that remove irrelevant operations.  In fact, test case reduction is
now standard practice in industral testing tools such as Mozilla's
{\tt jsfunfuzz}.  However, simply reducing the length of a test case
does not produce any kind of semantic simplicity.  There may be many
(often far more than a thousand, in practice) 1-minimal test cases
that present different variations of a single fault.  In many cases,
reading more than one of these test cases provides no useful
additional information on the cause of failure.

Consider the three test cases shown in Figure \ref{threetests}.  These
test cases are obviously very similar, and in fact all lead to a
violation of the property that an AVL tree must always be nearly
balanced, due to a missing call to {\tt rebalance} in {\tt delete} in
a Python implementation of AVL trees.  However, the test cases are
syntactically very different, and a testing system that collects
failing test cases will present all three tests for a user to examine.
While there are methods for attempting to determine which test cases
represent distinct faults \cite{PLDI13}, the ideal solution is
arguably to rewrite all three of these test cases into a single,
\emph{canonical} form that preserves the structure of the failure while removing
such accidental aspects of each test case as the particular integer
values and variables used, and the ordering of assignments and
insertions.

Figure \ref{normalgen} shows the result of applying our \emph{test case
  normalization} algorithm to these three tests, and then applying our
\emph{test case generalization} algorithm to the normalized test case.
\emph{All three test cases normalize to the same test
case}.  This single test case, in addition to the 10 steps required to produce
the failure, includes comments indicating what about the test case can
be changed while still failing in the same way.  For instance, the
value 1 assigned to {\tt int0} in step 0 is not essential.  It could
be changed to any value in the range 5-20 (the total set of values
allowed by the test generator) without changing the final result.  The
same is true of the assignment of 3 to {\tt int1}.  Similarly, the
exact ordering of many steps in the test case is not important.  Finally,
step 9 is annotated to show that instead of using the existing value
of {\tt int1} (4), a fresh assignment could be inserted before the
{\tt delete} call, setting {\tt
  int1} to 3 instead.  These possible
changes are not meant to be combined --- the annotation claims only
that changing these aspects of the test case one at a time will
preserve failure.  We call this annotation, which provides information
allowing a user to understand that a single test case is
representative of a \emph{family} of test cases that cause the same
kind of failure, test case \emph{generalization}.

Combining normalization and generalization avoids some common problems
with understanding automatically generated test cases.  For instance,
when a large integer appears in such a test case, the question always
arises --- is this unusual value important, or just a random number of
no significance \cite{MakeMost}?  After normalization, any large
values in a test case are essential, rather than
accidental, because normalization includes value minimization.
Without the additional step of generalization, however, it would be
easy to assume all small numeric values in normalized
tests are accidental.  Generalization informs a user when a small
value is required to reproduce a failure, and when it is simply an
artifact of normalization.

Normalization is not a complete solution to the problem of identifying
distinct faults (one key limitation is that our algorithms do not
apply to complex custom test generators such as CSmith \cite{csmith}
or {\tt jsfunfuzz} \cite{jsfunfuzz}), but it is highly effective when
it applies, in our experience.  Running 100,000 tests (of length 100)
on the faulty AVL tree produces 860 failing test cases with no
duplicates.  Normalizing these reduces the number of distinct failing
test cases to just 22.  Of course, ideally \emph{all} failures due to
the same fault in the SUT (Software Under Test) would normalize to a
single, representative test case, but we can only aim to approximate
such a canonical form for faults.  Figure \ref{diffnorm} shows a test
case that normalizes differently, and its normalized form (we omit the
generalization, which is not interestingly different than that for the
first normalized test case).

The contributions of this paper are 1) the idea of normalizing and
generalizing test cases, as steps toward a (likely unreachable) goal
of ``one test case to rule them all'' (per fault), 2) algorithms for normalizing
and generalizing test cases that make use of the abstract graph
interface for testing provided by the TSTL \cite{NFM15,ISSTA15}
testing language, and 3) some initial experimental results showing the
value of test case normalization and generalization.  We show that
normalization and generalization have been key to efforts to understand
complex failing test cases for a widely-used, highly complex library
for GIS automation.

\begin{figure}
{\scriptsize
{\bf Test case \#1}
\begin{code}
avl0 = avl.AVLTree() 
int0 = 4 
int2 = 13 
int3 = 7 
avl0.insert(int2) 
avl0.insert(int3) 
int1 = 15 
avl0.insert(int1) 
avl0.insert(int0) 
avl0.delete(int2)
\end{code}
{\bf Test case \#2}
\begin{code}
int0 = 14 
avl0 = avl.AVLTree() 
int2 = 13 
int1 = 15 
avl0.insert(int1) 
int1 = 11 
avl0.insert(int2) 
avl0.insert(int0) 
avl0.insert(int1) 
avl0.delete(int0) 
\end{code}
{\bf Test case \#3}
\begin{code}
avl1 = avl.AVLTree() 
int3 = 18 
avl1.insert(int3) 
int0 = 5 
int3 = 12 
avl1.insert(int0) 
int0 = 15 
avl1.insert(int0) 
avl1.insert(int3) 
int1 = 15 
avl1.delete(int1) 
\end{code}
}
\caption {Three randomly generated test cases for the same fault.}
\label{threetests}
\end{figure}

\begin{figure}
{\scriptsize
\begin{code}
\textcolor{black!45}{\#[}
int0 = 1                              \textcolor{black!45}{\# STEP 0}
\textcolor{black!45}{\#  or int0 = 5 }
\textcolor{black!45}{\#   - int0 = 20} 
\textcolor{black!45}{\#  swaps with step 4}
int1 = 3                              \textcolor{black!45}{\# STEP 1}
\textcolor{black!45}{\#  or int1 = 5 }
\textcolor{black!45}{\#   - int1 = 20} 
\textcolor{black!45}{\#  swaps with step 6}
avl0 = avl.AVLTree()                  \textcolor{black!45}{\# STEP 2}
\textcolor{black!45}{\#] (steps in [] can be in any order)}
avl0.insert(int0)                     \textcolor{black!45}{\# STEP 3}
\textcolor{black!45}{\#[}
int0 = 2                              \textcolor{black!45}{\# STEP 4}
\textcolor{black!45}{\#  swaps with step 0}
avl0.insert(int1)                     \textcolor{black!45}{\# STEP 5}
\textcolor{black!45}{\#] (steps in [] can be in any order)}
int1 = 4                              \textcolor{black!45}{\# STEP 6}
\textcolor{black!45}{\#  or int1 = 5 }
\textcolor{black!45}{\#   - int1 = 20} 
\textcolor{black!45}{\#  swaps with step 1}
avl0.insert(int1)                     \textcolor{black!45}{\# STEP 7}
avl0.insert(int0)                     \textcolor{black!45}{\# STEP 8}
avl0.delete(int1)                     \textcolor{black!45}{\# STEP 9}
\textcolor{black!45}{\#  or (}
\textcolor{black!45}{\#      int1 = 3  ;}
\textcolor{black!45}{\#      avl0.delete(int1) }
\textcolor{black!45}{\#     )}
\end{code}
}
\caption{Normalization and generalization for all three test cases.
  Lines beginning with \# are comments in Python, used for annotations.}
\label{normalgen}
\end{figure}

\begin{figure}
{\scriptsize
{\bf Test case \#4:}
\begin{code}
int0 = 10 
int2 = 7 
avl1 = avl.AVLTree() 
avl1.insert(int2) 
avl1.insert(int0) 
int1 = 1 
int3 = 1 
avl1.insert(int3) 
int3 = 15 
avl1.insert(int3) 
avl1.delete(int1) 
\end{code}
{\bf Normalized:}
\begin{code}
int0 = 1
int1 = 2
avl0 = avl.AVLTree()
avl0.insert(int0) 
avl0.insert(int1) 
int1 = 3  
avl0.insert(int1) 
int1 = 4  
avl0.insert(int1)  
avl0.delete(int0) 
\end{code}
}
\caption{A differently normalized test case for the same fault}
\label{diffnorm}
\end{figure}