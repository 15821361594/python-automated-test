\section{Conclusions and Future Work}

This paper introduces test case normalization and generalization.  The
methods presented are first steps towards a difficult goal: providing
users of automated testing with a \emph{single test case, as short and
  simple as possible, for each underlying fault in the SUT}, and
\emph{annotations describing the general conditions under which the
  fault manifests as failure}.  Normalization approaches this ideal by
rewriting numerous distinct failing test cases into a smaller, often
minimal, set of simpler test cases.  Generalization uses
automated experiments to identify essential and accidental aspects of
a test case.  The algorithms for normalization and generalization rely
on the ability of TSTL \cite{NFM15,ISSTA15} to define a total order
over test actions, based on an abstract form for test cases, suitable
for term rewriting.  TSTL-based normalization is thus applicable to
any SUT, any source language, and any test generation method,
including methods that already produce short tests
\cite{FA11,SoftBET}.  In our experiments, normalization reduces the
number of failures a user must examine by over an order of magnitude
compared to delta-debugging alone.

TSTL is available in a working version for Python \cite{tstl} that
supports normalization and generalization.  Further experimental
evaluation of normalization and generalization over more SUTs is
important to quantify effectiveness and motivate new rewrites and
generalizations.  The TSTL implementations are designed to allow these
to be easily added, in order to bring testing closer to the
goal of ``one test case to rule them all.''

 The goal of normalization and generalization can also be pursued in
settings other than API sequence or string grammar testing.  The
difficulties of defining a normal form for JavaScript \cite{jsfunfuzz}
or C \cite{CReduce} test cases are formidable. Less effective methods
than ours might still aid debugging and assist fuzzer taming
\cite{PLDI13}.  Simple generalization (e.g., is this numeric constant
essential, can these two statements be swapped?) and a limited form of
fresh value generalization should be easy to apply, even for complex
programming language test cases.  