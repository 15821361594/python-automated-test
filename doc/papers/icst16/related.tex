\section{Related Work}

This work builds on the ideas behind delta-debugging \cite{DD}, that test cases
should not contain extraneous information that is not needed to
reproduce failure.  It extends this beyond length to include a notion
where some actions in a test are considered simpler than others, which
in some cases also allows further reduction of the length of the test
case as well.  Delta-debugging and slicing \cite{TCminim} for test case
reduction are limited, generally, to producing subsets of the original
test case, not modifying parts of the test to obtain further
simplicity.

Normalization is in part motivated by the fuzzer taming \cite{PLDI13}
problem:  determining how many distinct faults are present in a large
set of failing test cases.  This is a key problem in practical
application of automated testing.  Our previous work on fuzzer taming
in fact uses delta-debugging as a simple normalization method that
reduces some test cases to syntactical duplicates; the methods are orthogonal.

Zhang \cite{SaiSimple} proposed an alternative approach to semantic
test simplification.  Like our approach, Zhang's is semantic, able to
modify, rather than simply remove, portions of a test.  However,
because Zhang operates directly over a fragment of the Java language,
rather than using an abstraction of test actions allowed, the set of
rewrite operations performed is highly restricted:  no new methods can
be invoked, statements cannot be re-ordered, and no new values are
used.  This limits its ability to simplify tests and makes it a fairly
weak normalizer (in fact it makes no attempt to perform syntactic
normalization, e.g., not making sure to use the same variables when
variable name is irrelevant to failure).

CReduce \cite{CReduce} performs some simple normalization as part of a
complex test-case reduction scheme for C code, and the
peephole-rewrite scheme used in CReduce is also an inspiration for the
approach taken by our normalizer.

Work on producing readable tests \cite{Guava,Readable} is also
related, in that it aims to ``simplify'' tests in a way that goes
beyond measuring test length.  Work on readable tests is primarily
intended to assist debugging in a human-factors sense, while our
normalization and generalization aims to increase the information
density of a test (hopefully increasing readability as well, of
course), and to address the fuzzer taming problem.  The approaches are
essentially orthogonal: our normalized tests could be altered to
improve readability by these methods.

Test case generalization is also related to dynamic invariant
generation, in that it informs the user of invariants over a series of
test executions satisfying some property \cite{Daikon}. 