\section {Case Studies}

\subsection{AVL Tree}

\subsection{XML Parser}

We also examined how normalization combined with multiple faults for a
simple XML parser with about 260 lines of code \cite{myxml}, with one
original fault (a crash instead of a parse error when given an empty
tag ({\tt <>}) and one seeded fault (failure when adding two nodes
with the same name).  Running 1,000 tests produces 848 failing test
cases.  Without normalization, it takes only 37.45 seconds to execute
all 1,000 tests, including delta-debugging these 848 failures.
However, the output is 717 distinct failing test cases.  Normalizing
increases the runtime to 354.7 seconds, but reduces the number of
distinct failing tests to 5, with 3 variations of the original fault
and 2 variations of the seeded fault.  Generalizing these 5 failures
increases the runtime by less than 5 seconds.

\subsection{TSTL}

\subsection{The numpy Library}

\begin{figure}
{\scriptsize
\begin{code}
 dim1 = 1 
 shape2 = (dim1, dim1, dim1) 
 array1 = np.ones(shape2) 
 array0 = array1 * array1 
 array1 = array1 + array1 
 array4 = array0 + array1 
 array0 = np.reshape(array4,shape2) 
 array3 = array1 * array4 
 array2 = np.ravel(array4) 
 array5 = array2 - array3 
 array4 = array5 * array2 
 array1 = np.unique(array0) 
 array5 = array5 * array3 
 array0 = array1 * array5 
 array5 = np.unique(array0) 
 array1 = array4 - array2 
 array2 = array0.flatten() 
 array0 = array5 + array5 
 array5 = array5 + array2 
 array2 = array0 * array2 
 np.copyto(array5,array2) 
 array2 = array2 * array5 
 array3 = array0 * array2 
 array0 = array3 - array1 
 array4 = array3 * array0 
 array1 = array5 + array4 
 array5 = array0 * array1 
 array0 = array5 - array1 
 array4 = array0 * array3 
 array3 = array4 * array0 
 array1 = array5 + array3 
 array0 = array2 + array1 
 array5 = array5 - array0 
 array5 = array3 * array5 
 array0 = array1 + array5 
 array2 = array3 - array0 
 array4 = array2 * array1 
 array3 = array4 * array2 
 array2 = array0 - array0 
 np.copyto(array1,array3) 
 array4 = array2.flatten() 
 array1 = array1 * array4
 assert (np.array\_equal(array1,array1))
\end{code}
}
\caption{Original failing test case for numpy (42 steps)}
\end{figure}

\begin{figure}
{\scriptsize
\begin{code}
dim0 = 1                            \# STEP 0
\#  or dim0 = 10 
shape0 = (dim0)                     \# STEP 1
\#  or shape0 = (dim0, dim0) 
\#  or shape0 = (dim0, dim0, dim0) 
array0 = np.ones(shape0)            \# STEP 2
array0 = array0 + array0            \# STEP 3
array0 = array0 + array0            \# STEP 4
\#  or array0 = array0 * array0 
array0 = array0 * array0            \# STEP 5
array0 = array0 * array0            \# STEP 6
array0 = array0 * array0            \# STEP 7
array0 = array0 * array0            \# STEP 8
array0 = array0 * array0            \# STEP 9
array0 = array0 * array0            \# STEP 10
array0 = array0 * array0            \# STEP 11
array0 = array0 * array0            \# STEP 12
array0 = array0 * array0            \# STEP 13
array0 = array0 - array0            \# STEP 14
\#  or array1 = array0 - array0 
\#  or array2 = array0 - array0 
\#  or array3 = array0 - array0 
\#  or array4 = array0 - array0 
\#  or array5 = array0 - array0 
assert (np.array\_equal(array0,array0))
\end{code}
}
\caption{Normalized and generalized test case (15 steps)}
\end{figure}

\subsection{ESRI arcpy GIS Library}

\begin{figure}
{\scriptsize 
\begin{code}
shapefile2 = "C:\\arctmp\\new3.shp" 
shapefile1 = "C:\\arctmp\\new3.shp" 
featureclass2 = shapefile2 
featureclass0 = shapefile1 
shapefilelist2 = 
   glob.glob("C:\\Arctmp\\*.shp") 
fieldname0 = "newf3" 
shapefile1 = shapefilelist2 [0] 
featureclass1 = shapefile1 
arcpy.CopyFeatures\_management
   (featureclass1,featureclass2) 
op1 = ">" 
newlayer2 = "l2" 
val1 = "100" 
selectiontype2 = "SWITCH\_SELECTION" 
fieldname1 = "newf1" 
arcpy.MakeFeatureLayer\_management
   (featureclass0, newlayer2) 
arcpy.SelectLayerByAttribute\_management
   (newlayer2,selectiontype2,
   ' "'+fieldname0+'" '+op1+val1) 
op0 = ">" 
arcpy.Delete\_management(featureclass2) 
arcpy.SelectLayerByAttribute\_management
   (newlayer2,selectiontype2,
   ' "'+fieldname1+'" '+op0+val1) 
\end{code}
}
\caption{Original test case for ESRI arcpy library (19 steps)}
\end{figure}

\begin{figure}
{\scriptsize 
\begin{code}
shapefilelist0 = 
   glob.glob("C:\\Arctmp\\*.shp")        \# STEP 0
\#[
shapefile0 = shapefilelist0 [0]        \# STEP 1
newlayer0 = "l1"                       \# STEP 2
\#  or newlayer0 = "l2" 
\#  or newlayer0 = "l3" 
\#  swaps with steps 3 4 5 6 7
\#] (steps in [] can be in any order)
\#[
featureclass0 = shapefile0             \# STEP 3
\#  swaps with step 2
fieldname0 = "newf1"                   \# STEP 4
\#  or fieldname0 = "newf2" 
\#  or fieldname0 = "newf3" 
\#  swaps with steps 2 8
selectiontype0 = "SWITCH\_SELECTION"    \# STEP 5
\#  or selectiontype0 = "NEW\_SELECTION" 
\#  or selectiontype0 = "ADD\_TO\_SELECTION" 
\#  or selectiontype0 = "REMOVE\_FROM\_SELECTION"
\#  or selectiontype0 = "SUBSET\_SELECTION"
\#  or selectiontype0 = "CLEAR\_SELECTION"   
\#  swaps with steps 2 8
op0 = ">"                              \# STEP 6
\#  or op0 = "<" 
\#  swaps with steps 2 8
val0 = "100"                           \# STEP 7
\#  or val0 = "1000" 
\#  swaps with steps 2 8
\#] (steps in [] can be in any order)
arcpy.MakeFeatureLayer\_management
   (featureclass0, newlayer0)          \# STEP 8
\#  swaps with steps 4 5 6 7
arcpy.SelectLayerByAttribute\_management
   (newlayer0,selectiontype0,
   ' "'+fieldname0+'" '+op0+val0)      \# STEP 9
arcpy.Delete\_management(featureclass0) \# STEP 10
arcpy.SelectLayerByAttribute\_management
   (newlayer0,selectiontype0,
   ' "'+ fieldname0+'" '+op0+val0)     \# STEP 11
\end{code}
}
\caption{Normalized and generalized test case for ESRI arcpy library
  (12 steps)}
\end{figure}