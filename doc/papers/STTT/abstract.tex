A test harness, in automated test generation, defines the set of valid
tests for a system, as well as their correctness properties. The
difficulty of writing test harnesses is a major obstacle to the
adoption of automated test generation and model checking. Languages
for writing test harnesses are usually tied to a particular tool and
unfamiliar to programmers, and often limit expressiveness. Writing
test harnesses directly in the language of the Software Under Test
(SUT) is a tedious, repetitive, and error-prone task, offers little or
no support for test case manipulation and debugging, and produces
hard-to-read, hard-to-maintain code. Using existing harness languages
or writing directly in the language of the SUT also tends to limit
users to one algorithm for test generation, with little ability to
explore alternative methods. In this paper, we present TSTL, the
Template Scripting Testing Language, a domain-specific language (DSL)
for writing test harnesses. TSTL compiles harness definitions into an
interface for testing, making generic test generation and manipulation
tools for all SUTs possible. TSTL includes tools for generating,
manipulating, and analyzing test cases, including simple model
checkers. This paper motivates TSTL via a large-scale testing effort,
directed by an end-user, to find faults in the most widely used
Geographic Information Systems tool.

This paper emphasizes a new approach to automated testing, where,
rather than focus on developing a monolithic tool to extend, the aim
is to convert a test harness into a language extension. This approach
makes testing not a separate activity to be performed using a tool,
but as natural to users of the language of the system under test as is
the use of domain-specific libraries such as ArcPy, NumPy, or QIIME,
in their domains. TSTL is a language and tool infrastructure, but is
also a way to bring testing activities under the control of an
existing programming language in a simple, natural way.