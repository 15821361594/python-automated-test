\section{Faults Discovered}

\begin{figure}
{\scriptsize 
\begin{code}
shapefilelist0 = glob.glob("C:\\Arctmp\\*.shp")                             \textcolor{black!60}{\# STEP 0}
\textcolor{black!60}{\#[}
shapefile0 = shapefilelist0 [0]                                           \textcolor{black!60}{\# STEP 1}
newlayer0 = "l1"                                                          \textcolor{black!60}{\# STEP 2}
\textcolor{black!60}{\#  or newlayer0 = "l2" }
\textcolor{black!60}{\#  or newlayer0 = "l3" }
\textcolor{black!60}{\#  swaps with steps 3 4 5 6 7}
\textcolor{black!60}{\#] (steps in [] can be in any order)}
\textcolor{black!60}{\#[}
featureclass0 = shapefile0                                                \textcolor{black!60}{\# STEP 3}
\textcolor{black!60}{\#  swaps with step 2}
fieldname0 = "newf1"                                                      \textcolor{black!60}{\# STEP 4}
\textcolor{black!60}{\#  or fieldname0 = "newf2" }
\textcolor{black!60}{\#  or fieldname0 = "newf3" }
\textcolor{black!60}{\#  swaps with steps 2 8}
selectiontype0 = "SWITCH\_SELECTION"                                       \textcolor{black!60}{\# STEP 5}
\textcolor{black!60}{\#  or selectiontype0 = "NEW\_SELECTION" }
\textcolor{black!60}{\#  or selectiontype0 = "ADD\_TO\_SELECTION" }
\textcolor{black!60}{\#  or selectiontype0 = "REMOVE\_FROM\_SELECTION"}
\textcolor{black!60}{\#  or selectiontype0 = "SUBSET\_SELECTION"}
\textcolor{black!60}{\#  or selectiontype0 = "CLEAR\_SELECTION"   }
\textcolor{black!60}{\#  swaps with steps 2 8}
op0 = ">"                                                                 \textcolor{black!60}{\# STEP 6}
\textcolor{black!60}{\#  or op0 = "<" }
\textcolor{black!60}{\#  swaps with steps 2 8}
val0 = "100"                                                              \textcolor{black!60}{\# STEP 7}
\textcolor{black!60}{\#  or val0 = "1000" }
\textcolor{black!60}{\#  swaps with steps 2 8}
\textcolor{black!60}{\#] (steps in [] can be in any order)}
arcpy.MakeFeatureLayer\_management(featureclass0, newlayer0)               \textcolor{black!60}{\# STEP 8}
\textcolor{black!60}{\#  swaps with steps 4 5 6 7}
arcpy.SelectLayerByAttribute\_management(newlayer0,selectiontype0,
   ' "'+fieldname0+'" '+op0+val0)                                         \textcolor{black!60}{\# STEP 9}
arcpy.Delete\_management(featureclass0)                                    \textcolor{black!60}{\# STEP 10}
arcpy.SelectLayerByAttribute\_management(newlayer0,selectiontype0,
   ' "'+ fieldname0+'" '+op0+val0)                                        \textcolor{black!60}{\# STEP 11}
\end{code}
}
\caption{Deleting a feature class does not invalidate or delete layers that depend on it.}
\label{fault1}
\end{figure}

\begin{figure}
{\scriptsize 
\begin{code}
shapefilelist0 = sorted(glob.glob(arcpy.env.workspace + "\\*.shp"))            \textcolor{black!60}{\# STEP 0}
\textcolor{black!60}{\#[}
shapefile0 = shapefilelist0 [0]                                               \textcolor{black!60}{\# STEP 1}
newlayer0 = "l1"                                                              \textcolor{black!60}{\# STEP 2}
\textcolor{black!60}{\#  or newlayer0 = "l2" }
\textcolor{black!60}{\#  or newlayer0 = "l3" }
\textcolor{black!60}{\#  swaps with step 3}
\textcolor{black!60}{\#] (steps in [] can be in any order)}
\textcolor{black!60}{\#[}
featureclass0 = shapefile0                                                    \textcolor{black!60}{\# STEP 3}
\textcolor{black!60}{\#  swaps with step 2}
classorlayer0 = newlayer0                                                     \textcolor{black!60}{\# STEP 4}
\textcolor{black!60}{\#  swaps with steps 10 11 12}
fieldtype0 = "DATE"                                                           \textcolor{black!60}{\# STEP 5}
\textcolor{black!60}{\#  or fieldtype0 = "TEXT" }
\textcolor{black!60}{\#  or fieldtype0 = "FLOAT" }
\textcolor{black!60}{\#  or fieldtype0 = "DOUBLE" }
\textcolor{black!60}{\#  or fieldtype0 = "SHORT" }
\textcolor{black!60}{\#  or fieldtype0 = "LONG" }
\textcolor{black!60}{\#  swaps with steps 10 11 12}
fieldname0 = "newf1"                                                          \textcolor{black!60}{\# STEP 6}
\textcolor{black!60}{\#  swaps with steps 10 11 12}
op0 = ">"                                                                     \textcolor{black!60}{\# STEP 7}
\textcolor{black!60}{\#  or op0 = "<" }
\textcolor{black!60}{\#  or op0 = "<=" }
\textcolor{black!60}{\#  or op0 = ">=" }
\textcolor{black!60}{\#  or op0 = "=" }
\textcolor{black!60}{\#  swaps with steps 10 11 12 14}
val0 = "100"                                                                  \textcolor{black!60}{\# STEP 8}
\textcolor{black!60}{\#  or val0 = "1000" }
\textcolor{black!60}{\#  swaps with steps 10 11 12 14}
stattable0 = arcpy.env.workspace + "\\stats.dbf"                               \textcolor{black!60}{\# STEP 9}
\textcolor{black!60}{\#  swaps with steps 10 11 12 14}
\textcolor{black!60}{\#] (steps in [] can be in any order)}
\textcolor{black!60}{\#[}
fieldlist0 = arcpy.ListFields(featureclass0)                                  \textcolor{black!60}{\# STEP 10}
\textcolor{black!60}{\#  swaps with steps 4 5 6 7 8 9 14}
stattype0 = "FIRST"                                                           \textcolor{black!60}{\# STEP 11}
\textcolor{black!60}{\#  or stattype0 = "LAST" }
\textcolor{black!60}{\#  swaps with steps 4 5 6 7 8 9}
statfields0 = []                                                              \textcolor{black!60}{\# STEP 12}
\textcolor{black!60}{\#  swaps with steps 4 5 6 7 8 9}
\textcolor{black!60}{\#] (steps in [] can be in any order)}
\textcolor{black!60}{\#[}
statfields0.append([fieldname0,stattype0])                                    \textcolor{black!60}{\# STEP 13}
arcpy.AddField\_management(featureclass0,fieldname0,fieldtype0); report()      \textcolor{black!60}{\# STEP 14}
\textcolor{black!60}{\#  swaps with steps 7 8 9 10}
\textcolor{black!60}{\#] (steps in [] can be in any order)}
fieldname0 = fieldlist0 [0].name \textcolor{black!60}{\# STEP 15}
arcpy.MakeFeatureLayer\_management(featureclass0,newlayer0,
   where\_clause=' "' + fieldname0 + '" ' + op0 + val0); report()              \textcolor{black!60}{\# STEP 16}
fieldname0 = "newf1"                                                          \textcolor{black!60}{\# STEP 17}
arcpy.DeleteField\_management(featureclass0,fieldname0); report()              \textcolor{black!60}{\# STEP 18}
arcpy.Statistics\_analysis(classorlayer0,stattable0,statfields0); report()     \textcolor{black!60}{\# STEP 19}
\end{code}
}
\caption{Deleting a field then computing statistics on it causes a crash.}
\label{fault2}
\end{figure}

\begin{figure}
{\scriptsize 
\begin{code}
shapefilelist0 = sorted(glob.glob(arcpy.env.workspace + "\\*.shp"))         \textcolor{black!60}{\# STEP 0}
shapefile0 = shapefilelist0 [0]                                            \textcolor{black!60}{\# STEP 1}
featureclass0 = shapefile0                                                 \textcolor{black!60}{\# STEP 2}
\textcolor{black!60}{\#[}
classorlayer0 = featureclass0                                              \textcolor{black!60}{\# STEP 3}
fieldtype0 = "DOUBLE"                                                      \textcolor{black!60}{\# STEP 4}
\textcolor{black!60}{\#  or fieldtype0 = "TEXT"}
\textcolor{black!60}{\#  or fieldtype0 = "FLOAT"}
\textcolor{black!60}{\#  or fieldtype0 = "SHORT"}
\textcolor{black!60}{\#  or fieldtype0 = "LONG"}
\textcolor{black!60}{\#  or fieldtype0 = "DATE"}
\textcolor{black!60}{\#  swaps with step 6}
fieldname0 = "newf1"                                                       \textcolor{black!60}{\# STEP 5}
\textcolor{black!60}{\#  or fieldname0 = "newf3"}
\textcolor{black!60}{\#  swaps with steps 6 8}
\textcolor{black!60}{\#] (steps in [] can be in any order)}
\textcolor{black!60}{\#[}
insertcursor0 = arcpy.InsertCursor(classorlayer0)                          \textcolor{black!60}{\# STEP 6}
\textcolor{black!60}{\#  swaps with steps 4 5}
arcpy.AddField\_management(featureclass0,fieldname0,fieldtype0); report()   \textcolor{black!60}{\# STEP 7}
\textcolor{black!60}{\#] (steps in [] can be in any order)}
fieldname0 = "newf2"                                                       \textcolor{black!60}{\# STEP 8}
\textcolor{black!60}{\#  or fieldname0 = "newf3"}
\textcolor{black!60}{\#  swaps with step 5}
arcpy.AddField\_management(featureclass0,fieldname0,fieldtype0); report()   \textcolor{black!60}{\# STEP 9}
insertcursor0 = arcpy.InsertCursor(classorlayer0)                          \textcolor{black!60}{\# STEP 10}
\end{code}
}
\caption{Creating two insert cursors on a layer, after adding two fields to the feature class underlying it, causes a crash.}
\label{fault3}
\end{figure}

\begin{figure}
{\scriptsize
\begin{code}
shapefilelist0 = sorted(glob.glob(arcpy.env.workspace + "\\*.shp"))         \textcolor{black!60}{\# STEP 0}
\textcolor{black!60}{\#[}
shapefile0 = shapefilelist0 [0]                                            \textcolor{black!60}{\# STEP 1}
newlayer0 = "l1"                                                           \textcolor{black!60}{\# STEP 2}
\textcolor{black!60}{\#  or newlayer0 = "l2" }
\textcolor{black!60}{\#  swaps with step 3}
\textcolor{black!60}{\#] (steps in [] can be in any order)}
\textcolor{black!60}{\#[}
featureclass0 = shapefile0                                                 \textcolor{black!60}{\# STEP 3}
\textcolor{black!60}{\#  swaps with step 2}
classorlayer0 = newlayer0                                                  \textcolor{black!60}{\# STEP 4}
\textcolor{black!60}{\#  or classorlayer0 = featureclass0 }
\textcolor{black!60}{\#  or (}
\textcolor{black!60}{\#      newlayer0 = "l3"  ;}
\textcolor{black!60}{\#      classorlayer0 = newlayer0 }
\textcolor{black!60}{\#     )}
fieldtype0 = "FLOAT"                                                       \textcolor{black!60}{\# STEP 5}
\textcolor{black!60}{\#  or fieldtype0 = "TEXT" }
\textcolor{black!60}{\#  or fieldtype0 = "DOUBLE" }
\textcolor{black!60}{\#  or fieldtype0 = "SHORT" }
\textcolor{black!60}{\#  or fieldtype0 = "LONG" }
\textcolor{black!60}{\#  or fieldtype0 = "DATE" }
fieldname0 = "newf1"                                                       \textcolor{black!60}{\# STEP 6}
\textcolor{black!60}{\#  or fieldname0 = "newf3" }
\textcolor{black!60}{\#  swaps with step 11}
op0 = ">"                                                                  \textcolor{black!60}{\# STEP 7}
\textcolor{black!60}{\#  or op0 = "<" }
\textcolor{black!60}{\#  or op0 = "<=" }
\textcolor{black!60}{\#  or op0 = ">=" }
\textcolor{black!60}{\#  or op0 = "=" }
val0 = "10"                                                                \textcolor{black!60}{\# STEP 8}
\textcolor{black!60}{\#  or val0 = "20" }
\textcolor{black!60}{\#  or val0 = "30" }
\textcolor{black!60}{\#  or val0 = "100" }
\textcolor{black!60}{\#  or val0 = "1000" }
\textcolor{black!60}{\#] (steps in [] can be in any order)}
\textcolor{black!60}{\#[}
whereclause0 = '"' + fieldname0 + '" ' + op0 + str(val0)                   \textcolor{black!60}{\# STEP 9}
arcpy.AddField\_management(featureclass0,fieldname0,fieldtype0); report()   \textcolor{black!60}{\# STEP 10}
\textcolor{black!60}{\#] (steps in [] can be in any order)}
\textcolor{black!60}{\#[}
fieldname0 = "newf2"                                                       \textcolor{black!60}{\# STEP 11}
\textcolor{black!60}{\#  or fieldname0 = "newf3" }
\textcolor{black!60}{\#  swaps with step 6}
arcpy.MakeFeatureLayer\_management(featureclass0,newlayer0,where\_clause=whereclause0);
   report()                                                                \textcolor{black!60}{\# STEP 12}
\textcolor{black!60}{\#] (steps in [] can be in any order)}
searchcursor0 = arcpy.SearchCursor(classorlayer0,whereclause0)             \textcolor{black!60}{\# STEP 13}
\textcolor{black!60}{\#  or searchcursor0 = arcpy.SearchCursor(classorlayer0) }
arcpy.AddField\_management(featureclass0,fieldname0,fieldtype0); report()   \textcolor{black!60}{\# STEP 14}
srow0 = searchcursor0.next()                                               \textcolor{black!60}{\# STEP 15}
\textcolor{black!60}{\#  or srow1 = searchcursor0.next() }
\textcolor{black!60}{\#  or srow2 = searchcursor0.next()}
\end{code}
}
\caption{Advancing a search cursor on a layer, after adding a field to
  the underlying feature class twice, causes a crash.}
\label{fault4}
\end{figure}

Thus far, our fault-finding process has focused on crashes.  Because
test cases that cause crashes stop the testing process, it is critical
to identify and avoid behavior that causes crashes before proceeding
to add further correctness properties.  Additionally, other than
hard-to-detect data corruption, crashes may be the most frustrating
faults for a developer to fix.  When ArcGIS or a standalone ArcPy
script causes a system crash, there is no readable error message, or
symptom in incorrect data to use in debugging the program.
Understanding system core dumps or analyzing the XML logs produced by
the ArcGIS engine is difficult for end users.  It would be ideal to
fix all ArcPy crashes by (at least) changing the library behavior to
issue an error message on invalid calls, but in the absence of bug
fixes, it is helpful to identify the root causes of various crashes
for users.  These test cases all involve calls that, as far as we
can tell, are not forbidden by ArcPy documentation.

Figures \ref{fault1}-\ref{fault4} show four test cases that result in
an ArcPy crash (the Python interpreter stops functioning, terminating
the test prematurely).  If executed in ArcGIS, these fragments will
crash ArcGIS.  Because these test cases are 1-minimal \cite{DD},
normalized, and generalized \cite{ICSTnorm}, we are able to describe in some detail
the general sequence of actions that produces each crash.  We have
discovered a few other crash faults; these are either not yet
well understood, or (we believe) equivalent in underlying cause
to these crashes.  The crashes discovered also serve as a basic proof
of concept that ArcPy test generation works and discovers
unanticipated interactions of API calls in the system.

\subsection{First Crash Fault}

ArcPy crashes when the feature class from which a layer is produced is
deleted, and the layer is used in a {\tt SelectLayer} call (this
version shows an attribute-based selection, but location selection
will cause the same problem): (Figure \ref{fault1}).  The underlying issue seems to be that
while operations on a deleted feature class properly notify a user the
feature class does not exist, ArcPy or ArcGIS does not track that
layers depending on that feature should also be deleted/invalidated
when the feature class is deleted.  Layers are not copies
of a feature class, but essentially new views of a feature class.
This means that when the underlying feature class is modified or
deleted, the view needs to be updated to reflect that change, and this
is not always correctly implemented.


\subsection{Second Crash Fault}

ArcPy crashes when asked to compute statistics (wth the {\tt FIRST} or
{\tt LAST} statistics types) over a field of a layer, when that field
has been deleted from the underlying feature class:  Figure
\ref{fault2}.  This is possibly related to the first crash fault:
ArcGIS again does not seem to properly propagate changes to an underlying
feature class to layers (which seem to be views) created on that feature class using a {\tt
  MakeFeatureLayer} call.



\subsection{Third and Fourth Crash Faults}

Creating an insert cursor on a feature class, before or after adding
one field to the feature class, then adding a second new field, then
creating a second insert cursor, causes ArcPy to crash:  Figure
\ref{fault3}.  A similar, but not obviously equivalent problem is
shown in Figure \ref{fault4}, where the combination and cursor type
are different, but the same issue of cursors interacting with feature
class or layer changes appears.  It seems likely that ArcPy should
simply add a requirement that feature classes or layers with active
cursors should not be modified at all, except by the active cursor.

