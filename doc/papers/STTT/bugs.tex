\section{Faults Discovered Using TSTL}
\label{sec:bugs}

\subsection{ArcPy Faults}

\begin{figure}
{\scriptsize 
\begin{code}
shapefilelist0 = glob.glob("C:\\Arctmp\\*.shp")                             \textcolor{black!60}{\# STEP 0}
\textcolor{black!60}{\#[}
shapefile0 = shapefilelist0 [0]                                           \textcolor{black!60}{\# STEP 1}
newlayer0 = "l1"                                                          \textcolor{black!60}{\# STEP 2}
\textcolor{black!60}{\#  or newlayer0 = "l2" }
\textcolor{black!60}{\#  or newlayer0 = "l3" }
\textcolor{black!60}{\#  swaps with steps 3 4 5 6 7}
\textcolor{black!60}{\#] (steps in [] can be in any order)}
\textcolor{black!60}{\#[}
featureclass0 = shapefile0                                                \textcolor{black!60}{\# STEP 3}
\textcolor{black!60}{\#  swaps with step 2}
fieldname0 = "newf1"                                                      \textcolor{black!60}{\# STEP 4}
\textcolor{black!60}{\#  or fieldname0 = "newf2" }
\textcolor{black!60}{\#  or fieldname0 = "newf3" }
\textcolor{black!60}{\#  swaps with steps 2 8}
selectiontype0 = "SWITCH\_SELECTION"                                       \textcolor{black!60}{\# STEP 5}
\textcolor{black!60}{\#  or selectiontype0 = "NEW\_SELECTION" }
\textcolor{black!60}{\#  or selectiontype0 = "ADD\_TO\_SELECTION" }
\textcolor{black!60}{\#  or selectiontype0 = "REMOVE\_FROM\_SELECTION"}
\textcolor{black!60}{\#  or selectiontype0 = "SUBSET\_SELECTION"}
\textcolor{black!60}{\#  or selectiontype0 = "CLEAR\_SELECTION"   }
\textcolor{black!60}{\#  swaps with steps 2 8}
op0 = ">"                                                                 \textcolor{black!60}{\# STEP 6}
\textcolor{black!60}{\#  or op0 = "<" }
\textcolor{black!60}{\#  swaps with steps 2 8}
val0 = "100"                                                              \textcolor{black!60}{\# STEP 7}
\textcolor{black!60}{\#  or val0 = "1000" }
\textcolor{black!60}{\#  swaps with steps 2 8}
\textcolor{black!60}{\#] (steps in [] can be in any order)}
arcpy.MakeFeatureLayer\_management(featureclass0, newlayer0)               \textcolor{black!60}{\# STEP 8}
\textcolor{black!60}{\#  swaps with steps 4 5 6 7}
arcpy.SelectLayerByAttribute\_management(newlayer0,selectiontype0,
   ' "'+fieldname0+'" '+op0+val0)                                         \textcolor{black!60}{\# STEP 9}
arcpy.Delete\_management(featureclass0)                                    \textcolor{black!60}{\# STEP 10}
arcpy.SelectLayerByAttribute\_management(newlayer0,selectiontype0,
   ' "'+ fieldname0+'" '+op0+val0)                                        \textcolor{black!60}{\# STEP 11}
\end{code}
}
\caption{Deleting a feature class does not invalidate or delete layers that depend on it.}
\label{fault1}
\end{figure}

In the process of testing ArcPy with TSTL, we discovered at least five
distinct faults
(thus far) that can cause an ArcPy script to crash.  While we have (as
discussed in Section \ref{sec:lang}) some properties that check for
data corruption and determinism of GIS analysis, we are not focusing
on these until we have a reliable way to avoid system crashes.   In
order to give an idea of what TSTL test cases look like, we discuss
briefly one of these ArcPy crashes.

ArcPy crashes when the feature class from which a layer is produced is
deleted, and the layer is used in a {\tt SelectLayer} call (this
version shows an attribute-based selection, but location selection
will cause the same problem): (Figure \ref{fault1}).  The underlying issue seems to be that
while operations on a deleted feature class properly notify a user the
feature class does not exist, ArcPy or ArcGIS does not track that
layers produced from a feature class should also be deleted/invalidated
when the feature class is deleted.  Layers are not copies
of a feature class, but essentially new \emph{views} of a feature class.
This means that when the underlying feature class is modified or
deleted, the view needs to be updated to reflect that change, and this
is not correctly implemented.  Figure \ref{fault1} shows part of an
annotated, reduced, normalized, and generalized test stand-alone test
case (with the boilerplate, function definitions, and imports
removed) for this fault.  The final line of code crashes ArcPy and the
Python interpreter.  Comments indicate alternative similar tests that
also fail.  In this case, TSTL's additional reduction steps (based on
term rewriting in the action language) remove almost half the steps in
the original, delta-debugged test case.

Other faults (or documentation lapses) in ArcPy we have discovered include crashes when computing
statisics over database fields of a layer using a deleted field and
crashes due to seemingly reasonable modifications of feature classes
while a database cursor is active.  In order to deal with the latter,
which seems more in the line of an undocumented behavioral restriction
than a ``bug,'' we now drastically limit database modification when a
cursor is active.  We have reported these problems to Esri, but have
not received a response.  The problems discovered may be previously
known to Esri, but are not generally known to the ArcPy user
community, and those that could be considered API limitations (that
cause unexplained crashes when violated) are not documented.

\subsection{Faults in Other Systems}

TSTL  is only slightly over a year old.  However, students using
TSTL in graduate classes on software testing have already, with
minimal assistance, discovered faults in some real-world systems.  Not
all of these are confirmed and reported yet.

First, TSTL testing revealed a fault in either the widely-used {\tt
  PyOpenCL} library, the even more widely-used {\tt OpenCL}
infrastructure, or (possibly) the NVIDIA hardware being used.  We are
still investigating this problem, but it appears to be a genuine
fault, though debugging and assigning blame is complex due to the
layers of software involved.  Second, TSTL testing found cases where distance metrics that were supposed to
be symmetric in the popular fuzzy-string-matching library FuzzyWuzzy
were asymmetric, if the default Python string match library was used
instead of a Levenshtein-distance library.
Third, TSTL testing revealed numerous problems with the {\tt
  astropy.table} module of the AstroPy
library, used by many professional astronomers and astrophysicists.
TSTL has also been used to discover faults in the TSTL API itself.

Most significantly, TSTL was able to discover at least 15
previously undiscovered faults in the widely used SymPy library for
symbolic mathematics in Python \cite{sympy}.  We have reported these
faults, and hope to collaborate with the SymPy team to provide
assistance in localizing and fixing them using TSTL.  The SymPy
effort was able to move from
decision-to-test to first discovered fault in the course of a single day.