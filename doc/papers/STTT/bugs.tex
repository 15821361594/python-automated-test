\section{Faults Discovered Using TSTL}

\subsection{ArcPy Faults}

\begin{figure}
{\scriptsize 
\begin{code}
shapefilelist0 = glob.glob("C:\\Arctmp\\*.shp")                             \textcolor{black!60}{\# STEP 0}
\textcolor{black!60}{\#[}
shapefile0 = shapefilelist0 [0]                                           \textcolor{black!60}{\# STEP 1}
newlayer0 = "l1"                                                          \textcolor{black!60}{\# STEP 2}
\textcolor{black!60}{\#  or newlayer0 = "l2" }
\textcolor{black!60}{\#  or newlayer0 = "l3" }
\textcolor{black!60}{\#  swaps with steps 3 4 5 6 7}
\textcolor{black!60}{\#] (steps in [] can be in any order)}
\textcolor{black!60}{\#[}
featureclass0 = shapefile0                                                \textcolor{black!60}{\# STEP 3}
\textcolor{black!60}{\#  swaps with step 2}
fieldname0 = "newf1"                                                      \textcolor{black!60}{\# STEP 4}
\textcolor{black!60}{\#  or fieldname0 = "newf2" }
\textcolor{black!60}{\#  or fieldname0 = "newf3" }
\textcolor{black!60}{\#  swaps with steps 2 8}
selectiontype0 = "SWITCH\_SELECTION"                                       \textcolor{black!60}{\# STEP 5}
\textcolor{black!60}{\#  or selectiontype0 = "NEW\_SELECTION" }
\textcolor{black!60}{\#  or selectiontype0 = "ADD\_TO\_SELECTION" }
\textcolor{black!60}{\#  or selectiontype0 = "REMOVE\_FROM\_SELECTION"}
\textcolor{black!60}{\#  or selectiontype0 = "SUBSET\_SELECTION"}
\textcolor{black!60}{\#  or selectiontype0 = "CLEAR\_SELECTION"   }
\textcolor{black!60}{\#  swaps with steps 2 8}
op0 = ">"                                                                 \textcolor{black!60}{\# STEP 6}
\textcolor{black!60}{\#  or op0 = "<" }
\textcolor{black!60}{\#  swaps with steps 2 8}
val0 = "100"                                                              \textcolor{black!60}{\# STEP 7}
\textcolor{black!60}{\#  or val0 = "1000" }
\textcolor{black!60}{\#  swaps with steps 2 8}
\textcolor{black!60}{\#] (steps in [] can be in any order)}
arcpy.MakeFeatureLayer\_management(featureclass0, newlayer0)               \textcolor{black!60}{\# STEP 8}
\textcolor{black!60}{\#  swaps with steps 4 5 6 7}
arcpy.SelectLayerByAttribute\_management(newlayer0,selectiontype0,
   ' "'+fieldname0+'" '+op0+val0)                                         \textcolor{black!60}{\# STEP 9}
arcpy.Delete\_management(featureclass0)                                    \textcolor{black!60}{\# STEP 10}
arcpy.SelectLayerByAttribute\_management(newlayer0,selectiontype0,
   ' "'+ fieldname0+'" '+op0+val0)                                        \textcolor{black!60}{\# STEP 11}
\end{code}
}
\caption{Deleting a feature class does not invalidate or delete layers that depend on it.}
\label{fault1}
\end{figure}


ArcPy crashes when the feature class from which a layer is produced is
deleted, and the layer is used in a {\tt SelectLayer} call (this
version shows an attribute-based selection, but location selection
will cause the same problem): (Figure \ref{fault1}).  The underlying issue seems to be that
while operations on a deleted feature class properly notify a user the
feature class does not exist, ArcPy or ArcGIS does not track that
layers produced from a feature class should also be deleted/invalidated
when the feature class is deleted.  Layers are not copies
of a feature class, but essentially new views of a feature class.
This means that when the underlying feature class is modified or
deleted, the view needs to be updated to reflect that change, and this
is not correctly implemented.