\section{Introduction}

Software test automation can be divided into two aspects:  automated execution
and determination of results for existing, human-created tests, and
the automatic generation of tests, based on some definition of valid
tests.  Both are critical for effective, efficient software testing,
but only automatic generation offers the promise of discovering faults
without human determination that a particular execution scenario has
the potential to behave incorrectly.  Automated generation of tests
relies on the construction of \emph{test harnesses}, which define the
set of valid tests (and, usually, a set of correctness properties for
those tests).
This paper presents a language and tool that apply insights from the
world of explict-state model checking to the problem of generating
test harnesses for automated test generation.  

Building a test harness is a task that even expert users of model
checkers or automated testing often find painful
\cite{woda08,woda12}.  The difficulty of harness generation is one
reason for the limited adoption of sophisticated testing and model
checking by the typical developer who writes unit tests.  This is
unfortunate, as even simple random testing can often uncover subtle
faults.

The ``natural'' way to write a test harness is as code in the language
of the Software Under Test (SUT).  This is obviously how most unit
tests are written, as witnessed by the proliferation of tools like
JUnit \cite{JUnit} and its imitators (e.g., PyUnit, HUnit, etc.).  It
is also how many industrial-strength random testing systems are
written \cite{ICSEDiff,AMAI}.  A KLEE ``test harness'' \cite{KLEE} for
symbolic execution is written in C, with a few additional constructs
to indicate which values are symbolic.  This approach is common in
model checking as well: e.g., Java Pathfinder \cite{JPF,JPF2} can
easily be seen as offering a way to define a state space using Java
itself as the modeling language, and CBMC \cite{CBMC,CBMCp} performs a
similar function in C, using SAT/SMT-based bounded model checking
instead of explicit-state execution.  JPF in particular has shown how
writing a harness in the SUT's own language can make it easy to
perform ``apples to apples'' comparisons of various testing/model
checking strategies \cite{JPFRandTest}.


\begin{figure}[t]
{\scriptsize
\begin{code}
op = choice(operations);
val1 = choice(values);
val2 = choice(values);
switch (op) \{
case op1:  if (guard1)
              call1(val1);
           break;
case op2:  if (guard2)
              call2(val1,val2);
           break;
case op3:  if (guard3)
              call3(val1,val3);
           break;
\end{code}
}
\vspace{-0.15in}
\caption {A test harness in the SUT language}
\label{fig:badharness}
\end{figure}



Unfortunately, writing test harnesses this way is a highly repetitive
and error-prone programming task, with many conceptual ``code clones''
(e.g. Figure \ref{fig:badharness}). A user faces difficult choices in
constructing such a harness. For example, the way guards and choices
are interleaved means that the state-space will be pointlessly
expanded to include many action and value choices that don't produce
any useful behavior.  This harness also always assigns {\tt val2} even
though {\tt call1} only uses {\tt val1}, to avoid having to repeat the
choice code for calls 2 and 3.  Moreover, this harness is possibly
sub-optimal for a method such as random testing, where the lack of any
memory for previously chosen values can make it hard to exercise code
behaviors that rely on providing the same arguments to multiple method
calls (e.g., {\tt insert} and {\tt delete} for container classes).
The construction of a harness becomes even more complex in realistic
cases, where the tested behaviors involve building up complex types as
inputs to method calls, rather than simple integer choices. For
example, consider the problem of testing or model checking a binomial
heap that supports several operations, defined in Figure
\ref{fig:binheap}.  Such a harness must manage the creation and
storage of values of multiple types, including heaps and references.
Moreover, because building up heaps and references is complicated,
they cannot simply be produced on each iteration, but must be
remembered.  As the interactions of multiple heaps (via {\tt union})
and references into a heap are the source of all interesting behavior,
the harness needs to decide how many heaps and references to store.
The code quickly becomes hard to read, hard to maintain, and hard to
debug.  In some cases \cite{AMAI} the code for a sophisticated test
harness approaches the SUT in complexity and even size!  The code's
structure also tends to lock in many choices (such as how to handle
storing heaps and references) that would ideally be configurable.

\begin{figure}[t]
{\scriptsize
\begin{code}
heap()                : returns a new heap
heap.insert(key,val)  : inserts a key with value, returning ref
heap.union(heap)      : merges two heaps
heap.extractMin()     : extracts the minimum
ref.delete()          : given a reference to a node, deletes it
ref.decreaseKey(key)  : decreases the key of ref's node
\end{code}
}
\vspace{-0.1in}
\caption{A binomial heap to test}
\label{fig:binheap}
\vspace{-0.20in}
\end{figure}

The definition of a harness also tends to be intimately tied to a
single tool, with the only testing strategies available being those
provided by that tool.  Writing novel testing strategies in even such
an extensible platform as Java Pathfinder is hardly a task for the
non-expert.
The harness in Figure \ref{fig:badharness} may support random testing and
some form of model checking, if it is written in Java and can use JPF
or a library for adaptation-based testing \cite{ISSRE12}. Such a
harness cannot support model checking or any sophisticated strategy
without being re-written if it is in a language like Python without
verification tool support.

What the user really wants is to simply provide the information in
Figure \ref{fig:binheap}, some configuration details (e.g., how many
{\tt ref}s to keep around), and some information on which testing
method to use (e.g., model checking, random testing, machine-learning
based approaches).  Some automated testing tools for Java \cite{FA11,Pacheco}
take a variation on this approach, automatically extracting the
signatures of methods from source code and testing them.
Unfortunately, completely automatic extraction often fails to handle
the subtle details of harness construction, such as defining guards
for some operations, or temporal constraints between API calls that
are not detectable by simple exception behavior.  Understanding
problems with automatic extraction can be hard with large libraries,
since the extraction will produce a huge, impenetrable mass of code
that is hard to examine by hand.  The user wants a
declarative harness, but often needs to program the details of a
harness, and build understanding by performing harness development in
small, incremental steps.

\subsection{Domain Specific Languages for Testing}

The properties of the problem at hand suggest the use of a
\emph{domain-specific language} (DSL) \cite{ISOLA12}.  DSLs
\cite{Fow10} provide abstractions and notations to support a
particular programming domain. The use of DSLs is a formalization of
the long-standing approach of using ``little languages'' in computer
science, as memorably advocated by Jon Bentley in one of his famous
Programming Pearls columns \cite{LitLang} and exemplified in such system
designs as UNIX.  DSLs typically come in two forms: \emph{external}
and \emph{internal}.  An external DSL is a stand-alone language, with
its own syntax.  An internal DSL, also known as a domain-specific
embedded language (DSEL), is hosted in a full-featured programming
language, restricting it to the syntax (and semantics) of that
language.  Many attempts to define harnesses can be seen as internal
DSLs \cite{UDITA,ISSRE12,JPF2,CBMCp,KLEE}.  Neither of these choices
is quite right for harness definition.  Simply adding operations for
nondeterministic choice, as is done in most cases, still leaves most of
the tedious work of harness definition to the user, and makes changing
testing approaches difficult at best.  With an external DSL, the user
must learn a new language, and the easier it is to learn, the less
likely it is to support the full range of features needed.

A novel approach is taken in recent versions of the SPIN model checker
\cite{SPIN}.  Version 4.0 of SPIN \cite{ModelDriven} made use of
SPIN's nature as a tool that \emph{outputs a C program} to allow users
to include calls to the C language in their PROMELA models.  The
ability to directly call C code makes it much easier to model check
large, complex C programs \cite{AMAI,ModelCode}.  C serves as a
``DSEL'' for SPIN, except that, rather than having a domain-specific
language inside a general-purpose one, here the domain-specific
language hosts a general-purpose language.  A similar embedding is
used in {\tt where} clauses of the LogScope language for testing Mars
Science Laboratory software \cite{scriptstospecs}.  We adopt this
approach in TSTL: embed a general-purpose language (for expressiveness) in a
DSL (for concision and ease-of-use).

\subsection{Contributions}

We showcase the TSTL
approach and tools using an ongoing case study, applying TSTL and
its tool suite to a large, real-world software system.
The test effort has been driven and directed not by a software testing
researcher (as is the usual case), but by a domain expert in the
Geographic Information System (GIS)
Software Under Test (SUT).  In the process, multiple faults and
undocumented restrictions of the API under test have been discovered,
and the language and tool suite have been improved.

This paper presents the first complete presentation of the TSTL
langauge and tools, with enough information on the language and the
API for two purposes:

\begin{itemize}
\item First, would-be users wanting to take advantage of automated
  test generation should be able to base their own testing
  efforts using TSTL on the small and large examples in this paper.
  This paper serves as a basic ``TSTL cookbook.''

\item Second, researchers should be able to use the information in this paper to
  extend existing TSTL tools or build their own tools to explore novel
  test generation strategies, automated debugging methods, and other
  software testing research projects.  TSTL enables easy comparison of
  implementations in a framework reducing the burden of implementation
  and irrelevant differences in performance.  The growing set of SUTs
  included in the TSTL distribution can serve as bases for
  experimental efforts, and include large and widely used Python
  software systems.
\end{itemize}

\subsection{Testing Esri ArcPy}


Esri is the single largest GIS software vendor, with about 40\% of
global market share.  Esri's ArcGIS tools are extremely widely used
for GIS analysis, in government, scientific research, commercial
enterprises, and education.  Automation of complex GIS analysis and
data management is essential, and Esri has long provided tools for
programming their GIS software tools.  The newest such method,
introduced in ArcGIS 10.0, is a Python site-package, ArcPy
\cite{ArcPy}.  ArcPy is a complex library, with dozens of classes and
hundreds of functions distributed over a variety of of toolboxes.
Most of the code executed in carrying out ArcPy functions is the code
for the ArcGIS engine itself.  This source code, written in C++, is
not available.  The source code for the latest version (10.3) of the
Python site-package alone, however, which interfaces with the ArcGIS
engine, is over 50,000 lines of code.  This is a very large system
(especially given the compactness of Python code), comparable in size
to the largest software systems previously tested using automated test
generation, such as core Java and Apache libraries
\cite{FA11,Pacheco}.

In order to improve the reliability of ArcPy, we are developing a
framework for automated testing of ArcPy itself, as well as libraries
based on ArcPy.  The source code for defining the structure of ArcPy
tests, written in the TSTL\footnote{TSTL stands for Template Scripting
  Testing Language.}\cite{NFM15,ISSTA14,tstl} domain-specific
language \cite{Fow10} for testing, is already more than six times as
large as the next-largest such definition previously implemented in
TSTL, even though the harness so far only includes a small portion of ArcPy
API (Application Program Interface) calls. The first stage of testing has resulted in discovery of
multiple faults in ArcPy/ArcGIS, and has required modifications to the
TSTL language and, especially, to the toolchain supporting test replay,
debugging, and test case understanding.

One of the contributions of this paper is a more in-depth discussion
of the problems, challenges, and tool utility aspects of testing
software than is typical in most research papers in the field.  Such
papers are (understandably) typically focused on novel algorithms or
empirical evaluations of known methods, rather than the practical aspects of finding
and understanding faults in a real-world software system.


\subsection{Automated Testing for the Rest of the World}

Previous work on automated testing for APIs has been largely carried
out by software testing researchers only, or (at most) by software
testing researchers working with individuals who are primarily
software developers.  This paper presents work largely directed by
the first author, who is not a software developer by
profession or education, but a GIS analyst.  

The problem of end-user testing
\cite{burnettEUSE,Silos,rothermelTOSEM} is known to be difficult.
Previous work has focused on end-users of systems that are not
traditional programming languages: e.g. spreadsheets \cite{rothermelTOSEM}, visual
languages, or machine-learning systems \cite{OnlyOracle}.  This paper aims to describe a
case study in how a user who is familiar with a software library but
not expert in software testing techniques can use (and adapt) existing
tools to test a traditional software API library.  In one sense, this
is a less difficult end-user testing scenario than testing
spreadsheets or visual forms, in that the testing is directed by an
individual used to writing and thinking about software source code.
The concepts in modern automated software testing are most easily
understood by those who are also familiar with the structure and semantics of
conventional programming languages.  On the other hand, the system to
be tested is large, not a small user-developed program: ArcPy is
a modern, complex, library, comparable in complexity and size to
core language libraries.   ArcPy was also not written by the end-user,
or by any of the authors of this paper, nor have the authors received
any  assistance in the effort from Esri.

Automated testing systems more advanced than a simple hand-written loop generating 
a few random inputs to a handful of functions, or more complicated to 
use than a fully push-button system are often considered too difficult 
for practical use even by software developers or software QA staff \cite{ISSRE}.
In fact, in the experience of the second author of this paper (an
academic software testing researcher), even ``push-button'' tools for
automated testing are often difficult for expert users to install, apply, and
configure \cite{AMAI,CFV08,ISSRE}.  Such tools typically do not approach the ease-of-use seen
in commercial static analysis tools \cite{Coverity,Klocwork,CodeSonar}.  One goal of this work
has been to mature the TSTL language and toolchain so Python
programmers from all backgrounds can easily apply it to their
automated testing problems.

The second major contribution of this paper is therefore a presentation of an
approach to automated testing that has been chosen by a GIS analyst, not a
software developer or testing researcher.  Moreover, we present this
paper as a proof-of-concept that modern automated testing, even in a
highly interactive, non-push-button form, can be used by a motivated
domain expert, with the support of  a domain-specific language
\cite{Fow10} and a set of tools for generating, analyzing, and replaying tests.


