\section{Conclusions}
\label{conclusion}

This paper presents the latest version of the TSTL
\cite{NFM15,ISSTA15,tstl} domain-specific language for testing, which
enables a declarative style of test harness development, where the
focus is on defining the actions in valid tests, not determining
exactly how tests are generated.  Because TSTL, inspired by the SPIN
model checker, produces a software-under-test-independent interface
for testing, TSTL makes it possible for users to easily apply
different test generation methods to the same system without undue
effort.  The same approach makes it possible for researchers to rapidly prototype novel test generation
methods, and evaluate them in a context where differences in test
infrastructure not relevant to the algorithms at hand can be minimized.

TSTL has, in the year since its initial introduction,
already been used to discover (to our knowledge) previously unknown
faults in multiple Python libraries, including the very widely-used
ArcPy site package for GIS scripting. 
As future work, we plan to continue to use TSTL to explore novel
testing algorithms, investigate the relative strengths of systematic,
stochastic, and directed test generation methods, and apply TSTL to
look for faults in widely used libraries.  Additionally, we plan to
port TSTL (which is only loosely coupled with underlying language) to additional
programming languages beyond Python and Java.
