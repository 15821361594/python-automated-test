\section{Related Work}

There is a vast amount of previous work on automated generation of
tests for (API-based) software systems
\cite{Pacheco,FA11,GodefroidKS05} and random testing in particular
\cite{ICSEDiff,Pacheco,AMFL11,ARTChen,ISSTAART,FASE,HamletOnly,Hamlet94,ClaessenH00,CiupaLOM07,RandFormal,woda08,andrews-etal-rute-rt,ASE08,evalrand,csmith},
some dating back to the early 1980s. It is far beyond the scope of
this paper to explore that literature in detail.  The interested
reader is directed to the cited papers, as well as general surveys of
automated test generation in particular \cite{anand2013orchestrated}
or recent software testing research in general \cite{orsofuse}.  

%This paper relies primarily on past
%results showing that random testing \cite{Hamlet94,Fuzz} can be highly effective for
%finding software faults in large libraries
%\cite{Pacheco,ICSEDiff,RandFormal,CFV08,AMAI}.  To our knowledge, there is almost
%no work on automated/random test generation for GIS in particular, and
%little or no work on allowing non-traditional software developers to apply
%automated (random) test generation to APIs.

%TSTL is a domain-specific language (DSL) \cite{Fow10} for testing, in
%the spirit of the famous QuickCheck tool for Haskell
%\cite{ClaessenH00}; other DSLs for testing exist \cite{UDITA}, but
%these are more limited in scope and function.  

To our knowledge, there has been no previous proposal of a concise
domain-specific-language \cite{Fow10} like TSTL, to assist users in building test
harnesses.  One line of related work is previous work on
building common frameworks for random testing and model checking
\cite{woda08} and proposing common terminology for imperative
harnesses \cite{woda12}.  Earlier publications on TSTL \cite{NFM15,ISSTA15} presented a
language considerably more limited in functionality and with a more
difficult-to-read syntax than the presentation in this paper, and
omitted details of the tools provided in the TSTL distribution for
off-the-shelf testing.

There exist various testing tools and languages of a somewhat
different flavor: e.g. Korat \cite{Korat}, which has a much more fixed
input domain specification, or the tools built to support the Next
Generation Air Transportation System (NextGen) software
\cite{TameInputs}.  The closest of these is the UDITA language
\cite{UDITA}, an extension of Java with non-deterministic choice
operators and {\tt assume}, which yields a very different language
that shares our goal.  TSTL aims more at the \emph{generation} of
tests than the \emph{filtering} of tests (as defined in the UDITA
paper), while UDITA supports both approaches.  This goal of UDITA (and
resulting need for first-class {\tt assume} statements) means that it
must be hosted inside a complex (and sometimes non-trivial to
install/use) tool, JPF \cite{JPF2}, rather than generating a
stand-alone simple interface to a test space, as with TSTL.  Building
``UDITA'' for a new language is far more challenging than porting
TSTL.  UDITA also supports far fewer constructs to assist test harness
development.

The design of the SPIN model checker \cite{SPIN} and its model-driven
extension to include native C code \cite{ModelDriven} inspired the
flavor of TSTL's domain-specific language, though our approach is more
declarative than the ``imperative'' model checker produced by SPIN.
Similarly, work at JPL on languages for analyzing spacecraft telemetry
logs in testing \cite{scriptstospecs} provided a working example of a
Python-based declarative language for testing purposes.  The pool
approach to test case construction is derived from work on canonical
forms and enumeration of unit tests \cite{AndrewsTR}.

There is relatively little written about software testing in the
academic literature that focuses on the pitfalls, best practices, and
engineering challenges of testing real-world systems.  The series of
papers by the NASA Jet Propulsion Laboratory on testing the file
systems for the Curiosity Mars Rover is notable, though even there the
focus is more on general technique than details of approach
\cite{ICSEDiff,CFV08,AMAI}.  Literature that at least touches on more
practical aspects, however, is plentiful: e.g., Lei and Andrews
\cite{MinUnit} demonstrated that random test generation essentially
requires test-case minimization \cite{DD} in order to be useful.
Recent work, in particular, has given more attention to issues of test case usability
than in the past; we suspect this shows a growing technological
maturity for automated test case generation.  For example, we found
that test case readability was important in our efforts, and recent
academic work \cite{Readable,Guava} has introduced systematic,
principled methods for improving the readability of test cases for
humans, even though this does not improve fault detection, coverage,
or other traditional measures of test effectiveness.

The literature on testing GIS (Geographic Information System) software in particular seems to consist
of one paper proposing a very limited application of automated testing
to assist GIS users, primarily in model development \cite{GISTest}.
That work does not target the reliability or correctness of the
underlying GIS engine, or GIS libraries.  There is also some
discussion of automated testing for GIS in various blog posts and
discussion groups (e.g., \cite{gisblog1,gisblog2}), but no formal
academic case studies.  These discussions also tend to focus on
application testing or GUI testing, rather than testing of library
code used across GIS applications.  There is a simple extension to
Python unit testing modules for the GRASS open source GIS system
\cite{GRASSunit}, but this does not provide any automated test generation.

There is a significant body of work on end-user testing of software,
part of the larger field of end-user software engineering
\cite{burnettEUSE,Silos}.  End-user software engineering examines how
software can best be produced by developers who do not have a
traditional computer science background, and are often primarily
interested in an application of programming, rather than software
development as a profession.  GIS developers are (we believe) a
typical example \cite{Segal07}:  they are technically skilled  individuals whose
primary expertise is not in software development, but who, in order to
pursue their goals, must develop, maintain, and test significant
software systems.

The earliest work focusing on software testing for
end-user software engineers explored how to test spreadsheets
\cite{rothermelTOSEM,rothermel2000wysiwyt}.  Other work has focused on
errors end-users make in specifying systems \cite{Phalgune}, and how
end-users of machine learning systems (who may be machine learning
experts, or individuals with no programming knowledge at all) can test
such systems \cite{OnlyOracle,kulesza-eud11,shinsel-vlhcc}.  To our
knowledge, no previous work considers API-sequence testing for
end-users.  Previous work on API testing has largely not even included
traditional software developers, but been performed by software
testing researchers only.

Chen et al. propose metamorphic testing
\cite{MetaTest,isstamorph,metamorph,chentest} as a possible solution
to the oracle problem (defining correct behavior of a software system)
for end-user programmers \cite{MetamorphEndUser}.  In metamorphic
testing, while the correct output of a program is not known, the
effect of certain modifications to an input is known (e.g., increasing
a certain input should increase a certain output), allowing faults to
be detected even without a definition of the correct result.  In this
paper, we largely focus on building a basic framework that works for
crash bugs and uncaught exceptions, and is capable of
extension to more complex correctness properties.  In the long run, however, more
sophisticated oracles are critical to finding deep faults.
