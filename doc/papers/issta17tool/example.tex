\section{The Basic TSTL Test Tools}

Instead of the fault-free stack, we can test a real-world program with real faults, such as the SymPy library for performing symbolic mathematics in Python \cite{SymPy}.  The SymPy harness can be found in the TSTL github repository {\tt examples/sympy} directory.

{\scriptsize
\begin{code}
 > tstl sympy.tstl
 > tstl\_rt --swarm --noCover --full
...
 UNCAUGHT EXCEPTION
 ERROR: (<type 'exceptions.RuntimeError'>,
 RuntimeError('maximum recursion depth exceeded',)
...
    return func(a, b)
...
 SAVING TEST AS failure.67076.test
...
 STOPPING TESTING DUE TO FAILED TEST
 36.4984600544 TOTAL RUNTIME
 > wc -l failure.67076.test
 72
 > head -n5 failure.67076.test
 self.p\_v[3] = sympy.Symbol('j',positive=True) 
 self.p\_expr[0].evalf() 
 self.p\_expr[0] = self.p\_expr[0] + self.p\_expr[3] 
 self.p\_expr[3] = self.p\_expr[3] + self.p\_expr[1] 
 self.p\_expr[1] = self.p\_expr[3] \% self.p\_expr[0] 

\end{code}
}

We have instructed the random tester to use swarm testing
\cite{ISSTA12} and not collect code coverage, in order to improve the
chances of quickly finding a fault.  By default {\tt tstl\_rt} uses
delta-debugging to minimize tests before saving them, but we have
also instructed {\tt tstl\_rt} to simply save the original test case {\tt
  --full}.  The unreduced test (which causes Python to enter an
infinite recursion sequence) consists of 72 steps, saved in a
non-executable, but basically human-readable, textual
format (in an automatically generated file name, based on the process ID).  This is not a very useful test, so we want to reduce
it:

{\scriptsize
\begin{code}
 > tstl\_reduce failure.67076.test reduced.test --noNormalize
 STARTING WITH TEST OF LENGTH 72
 REDUCING...
 REDUCED IN 31.3780119419 SECONDS
 NEW LENGTH 7
 ALPHA CONVERTING...
 c0 = sympy.Integer(4)                            \# STEP 0
 c1 = sympy.Integer(9)                            \# STEP 1
 v0 = sympy.Symbol('k',positive=True)             \# STEP 2
 expr0 = sympy.Rational(c1,c1)                    \# STEP 3
 expr1 = sympy.Product(expr0,(v0,c0,c0))          \# STEP 4
 expr2 = c1                                       \# STEP 5
 expr3 = expr2 \% expr1                            \# STEP 6
\end{code}
}

This test, reduced using standard delta-debugging \cite{DD}, is
short.  Also, note that TSTL automatically alpha-converts the test so
that it uses variables to store intermediate values in a reasonable
way (starting with {\tt v0} rather than arbitrarily beginning with
{\tt v3}, for example).  However, the test is neither as short as possible nor, more
importantly, as simple as possible.  For debugging we may well wonder:
does it matter that {\tt c0} is 4 and {\tt c1} is 9?  Is the use of
the variable {\tt k} relevant?  If we want to know the answers, we can
run the reducer to \emph{normalize} \cite{OneTest} the test, in place
of simply reducing it:

{\scriptsize
\begin{code}
 > tstl\_reduce reduced.test normalized.test --noReduce
 STARTING WITH TEST OF LENGTH 7
 NORMALIZING...
 NORMALIZED IN 383.565114975 SECONDS
 NEW LENGTH 5
 c0 = sympy.Integer(1)                            \# STEP 0
 v0 = sympy.Symbol('a')                           \# STEP 1
 expr0 = c0                                       \# STEP 2
 expr1 = sympy.Sum(expr0,(v0,c0,c0))              \# STEP 3
 expr0 = expr0 \% expr1                            \# STEP 4
\end{code}
}

Notice that normalizing a test is much more expensive than simply
reducing it, but the payoff is an even shorter and simpler
test\footnote{For details on how much shorter and simpler, see the
  conference paper on test case normalization and generalization
  \cite{OneTest}.  Note that normalization times are usually faster in
the current release than reported in that paper, due to the
implementation of a useful heuristic for reducing nearly 1-minimal
tests suggested by David R. MacIver, the author of the Hypothesis tool.}.  By default, the TSTL random test generator
reduces tests before saving them, and the standalone {\tt
  tstl\_reduce} tool is use to normalize interesting tests.  Calling
{\tt tstl\_rt} with the {\tt --normalize} option avoids going through
the standalone tool.  Normalization here pays off by revealing that
the use of a {\tt Rational} and exact numeric/symbol values are not
relevant.  

Now that we have an extremely simple test, we can replay it in a
``verbose'' mode to see more exactly what is happening during the test, as shown in
Figure \ref{fig:verbose}.  This shows the values, types, and changes
in values of every pool variable involved in each step of the test
(and would show the state of a reference implementation, if we were
performing automated differential testing).

\begin{figure}
{\scriptsize
\begin{code}
 > tstl\_replay normalized.test --verbose
 STEP \#0: ACTION: c0 = sympy.Integer(1) 
 c0 = None : <type 'NoneType'>
 => c0 = 1 : <class 'sympy.core.numbers.One'>
 ==================================================
 STEP \#1: ACTION: v0 = sympy.Symbol('a') 
 v0 = None : <type 'NoneType'>
 => v0 = a : <class 'sympy.core.symbol.Symbol'>
 ==================================================
 STEP \#2: ACTION: expr0 = c0 
 c0 = 1 : <class 'sympy.core.numbers.One'>
 expr0 = None : <type 'NoneType'>
 => expr0 = 1 : <class 'sympy.core.numbers.One'>
 ==================================================
 STEP \#3: ACTION: expr1 = sympy.Sum(expr0,(v0,c0,c0)) 
 c0 = 1 : <class 'sympy.core.numbers.One'>
 v0 = a : <class 'sympy.core.symbol.Symbol'>
 expr0 = 1 : <class 'sympy.core.numbers.One'>
 expr1 = None : <type 'NoneType'>
 => expr1 = Sum(1, (a, 1, 1)) : <class 'sympy.concrete.summations.Sum'>
 ==================================================
 STEP \#4: ACTION: expr0 = expr0 \% expr1 
 expr0 = 1 : <class 'sympy.core.numbers.One'>
 expr1 = Sum(1, (a, 1, 1)) : <class 'sympy.concrete.summations.Sum'>
 RAISED EXCEPTION: <type 'exceptions.RuntimeError'>
   maximum recursion depth exceeded in cmp
 FAILED STEP
 (<type 'exceptions.RuntimeError'>,
 RuntimeError('maximum recursion depth exceeded in cmp',)
 ...
\end{code}
}
\caption{Verbose replay of a TSTL test.}
\label{fig:verbose}
\end{figure}

In addition to replaying a single test, we can replay a number of
saved tests using the {\tt tstl\_regress} command, which takes as
input a list of all test files to run, and produces a coverage report
in addition to the outcome of each test.  By default it stops on the
first failing test, but can be directed to run all tests with {\tt
  --keepGoing}.  Regression runs can also generate an HTML coverage report using the
facilities of the {\tt coverage.py} library \cite{Coveragepy}.

Finally, we can \emph{generalize} the test, to see what
alternative, similar tests also produce the same failure:

{\scriptsize
\begin{code}
 > tstl\_generalize normalized.test
 GENERALIZING...
 \#[
 c0 = sympy.Integer(1)                               \# STEP 0
 \#  or c0 = sympy.Integer(2) 
 \#   - c0 = sympy.Integer(10) 
 v0 = sympy.Symbol('a')                              \# STEP 1
 \#  or v0 = sympy.Symbol('b') 
 \#   - v0 = sympy.Symbol('d') 
 \#  or v0 = sympy.Symbol('x') 
 \#   - v0 = sympy.Symbol('z') 
 \#  or v0 = sympy.Symbol('e',positive=True) 
 \#   - v0 = sympy.Symbol('l',positive=True) 
 \#  swaps with step 2
 \#] (steps in [] can be in any order)
 expr0 = c0                                          \# STEP 2
 \#  or expr0 = sympy.Rational(c0,c0) 
 \#  or expr0 = sympy.pi 
 \#  or expr0 = sympy.E 
 \#  or expr0 = sympy.I 
 \#  swaps with step 1
 expr1 = sympy.Sum(expr0,(v0,c0,c0))                 \# STEP 3
 \#  or expr1 = sympy.Product(expr0,(v0,c0,c0)) 
 expr0 = expr0 \% expr1                               \# STEP 4
 \#  or expr2 = expr0 \% expr1 
 \#  or expr3 = expr0 \% expr1 
 GENERALIZED IN 239.682291985 SECONDS
\end{code}
}

With this information, the basic underlying structure
of the fault is made clear:  using the modulo operator on a {\tt Sum}
or {\tt Product} over an empty range (whether that range is $2 \ldots 2$ or
$\pi \ldots \pi$, with any variable name allowed by our SymPy harness,
causes the failure.  The ordering of operations, other than to the
extent required for data flow, is not important.

Now that we understand the fault, we may want a non-TSTL test to run
in a debugger to try out possible solutions.  Generating a standalone
test is easy:

{\scriptsize
\begin{code}
 > tstl\_standalone normalized.test normalized.py
\end{code}
}

\section{Avoiding Slippage}

Test slippage \cite{PLDI13,slippage} is a phenomenon where a weak
labeling of failed tests (e.g., simply checking that a reduced test
still causes some kind of uncaught exception) causes a test that
originally fails due to one fault to be reduced to a test that fails
due to a different fault.

TSTL provides a few ways to avoid slippage, and also some ways to
intentionally induce ``good'' slippage where a failing test is reduced
to produce multiple tests that fail due to different faults
\cite{slippage}.  First, the random test generator and the reduction,
normalization, and generalization tools all take the {\tt --keepLast}
option, which forces reduced tests to have the same final action as
the original test.  This is a heuristic for avoiding slippage
discovered during file system testing at NASA \cite{ICSEDiff}.