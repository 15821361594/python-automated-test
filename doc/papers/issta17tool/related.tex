\section{Related Work}

The tools described here are obviously inspired by  delta-debugging
\cite{DD} and the idea that tests should not contain extraneous parts not needed to
cause test failure (or other behavior of interest \cite{icst2014,stvrcausereduce}).  Delta-debugging and slicing
\cite{TCminim} produce subsets of the
original test, but do not modify parts of the test to obtain further
simplicity.  Our work on normalization \cite{OneTest} extends this
idea to rewrite tests into a ``canonical'' form.  Normalization to a
canonical form is in part motivated by the fuzzer taming \cite{PLDI13}
problem.

Zhang \cite{SaiSimple} proposed an approach to semantic
test simplification that, like our approach, is able to modify, rather
than simply remove, portions of a test.  However, because the simplification
operates directly over a fragment of the Java language, rather than
using an abstraction of test actions, the set of rewrite
operations is very simple: no new methods can be
invoked, statements cannot be re-ordered, and no new values are used.
The approach also performs little syntactic normalization: e.g., it does not even force a test to use
fixed variable names when variable name is irrelevant.  CReduce
\cite{CReduce} performs some simple normalization as part of its
test reduction for C code, and the peephole-rewrite scheme
used in CReduce is also an inspiration for the approach taken by our
normalizer.  By writing a TSTL harness that is in the form of
constructor calls to create an AST, TSTL can normalize/reduce hierarchically
structured input data in ways similar to CReduce and Hierarchical
Delta Debugging \cite{HDD}.

The most closely related work to test generalization is Pike's
SmartCheck \cite{SmartCheck}.  SmartCheck works with algebraic data in
Haskell, and is an alternative approach to reduction
and generalization.  The only other work we are aware of that is
similar to generalization concerns causality in
model checking counterexamples \cite{FreeWill,MakeMost,SPIN03}.  