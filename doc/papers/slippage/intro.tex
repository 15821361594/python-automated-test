\section{Introduction}

Automated testing often goes hand in hand with
test reduction or minimization \cite{DD,MinUnit,TCminim,ICSEDiff,CReduce}.
Such reduction is
now standard practice in industrial testing tools such as Mozilla's
{\tt jsfunfuzz} \cite{jsfunfuzz2}.  Many automated
test generation methods, especially those based on random testing,
produce long tests with many irrelevant components.  Such tests are
hard to understand for debugging and slow to execute.  Delta debugging
automatically reduces the size of such tests,
producing a smaller test that still fails.

Unfortunately, the consequences of reducing a test are not always
desirable.  In particular, sometimes the reduced test no longer
detects the same faults it originally detected.  This phenomenon is
called \emph{slippage} \cite{PLDI13}.  

Formally, we define slippage as occurring whenever a test $t$ detects
faults $F$, and the reduction $r$ of $t$ detects a different set of
faults, $F' \neq F$.  Informally, most discussion of slippage is
concerned with the case where
$\exists f .  f \in F \wedge f \not\in F'$ --- that is, when slippage
causes loss of a fault.  Slippage is usually avoided by using
heuristics, such as that tests failing with the same error message or
ending in the same step are due to the same fault \cite{ICSEDiff}.  In
some cases, such as system invariant violations or compiler wrong-code
bugs \cite{PLDI13}, such methods do not work well.  The extent to
which slippage is a practically important problem in real-world
testing is unknown.

However, all slippage is not
harmful.  Even when a fault is lost, the new fault may be more
interesting or harder-to-find.  We suspect this is usually not the
case, as slippage probably tends to favor easy-to-detect faults, but
there is no hard data on the matter.  Considering the
possibility of beneficial slippage naturally leads to a new approach to
mitigating slippage.  In place of traditional delta debugging, which,
given one failing test, produces one reduced failing test, we propose
to modify delta debugging to produce a set of reduced
failing tests.  Ideally, this set exposes more than one fault, if possible.
Throughout the remainder of this paper, we refer to the original,
one-test, delta debugging algorithm \cite{DD} as {\tt ddmin}.  

We
propose two novel approaches to slippage mitigation (and
exploitation), both making use of any existing {\tt ddmin}
implementation, even (we believe) ones that are significantly different than the
version of Hildebrandt and Zeller \cite{DD}.  The contributions of this paper are:

\begin{enumerate}
\item A formal definition of slippage, and the notion that slippage
  may be either harmful or beneficial.
\item A proposal to avoid and make use of slippage by producing more
  than one reduced test per failing test.
\item Two novel approaches to slippage mitigation.
\item Some preliminary experimental results showing the basic effectiveness of these
  methods for a very large set of mutant-simulated faults in a Python
  AVL tree and for a set of real-world tests and faults for a complex
  JavaScript compiler.
%\item Further data on the prevalence of  and factors determining slippage.
\end{enumerate}

Both of our proposed methods reduce harmful slippage and result in
detection of more faults.

\section{Slippage Mitigation Approaches}

\subsection{The Combinatorial Blocking Algorithm}

The first algorithm uses additional unmodified {\tt ddmin} runs, but
``blocks'' them from producing the same reduced test by modifying the
test input to delta debugging.  The algorithm is called combinatorial
because it uses combinations of components in $t$ to construct new
starting tests, and blocking because the purpose of these combinations
is to ensure that we do not get the same solution as the first run of
{\tt ddmin}, in a way that is analogous to the use of blocking clauses
to force new solutions from SAT solvers.

Given test $t$ that reduces to $r$, we compute all subsets of
components of $r$, $C_r$.  The set of reduced tests is then computed
by running delta debugging starting with each $t-c$ that fails, for all
$c \in C_r$: $c$ is the blocked components of $r$.  So long as even
one component is blocked, it is impossible to reproduce the same $r$.
The intuition is that to find a reduced failing test that exhibits a
different fault than $r$, we want to run delta debugging in such a way
as to produce a test as different as possible from $r$; ideally we
would like a reduced test sharing no components with $r$.  However,
$r$ will likely contain components that must appear in any failing
test: for example, calls to {\tt mount} appear in all useful file
system tests, and interesting XML files seldom lack the {\tt <}
character.  Therefore, rather than only trying to block all components of $r$,
we try delta debugging with different combinations of components blocked.

In practice, iterating through all subsets may be too expensive if $r$
is long.  We therefore make a simplifying assumption:  if $t-c_1$ does not fail,
and $c_1 \subset c_2$, then $t-c_2$ also does not fail.  The blocking
algorithm begins its search by blocking all single components of $r$,
then proceeds to all combinations of 2 components, etc., at each stage
only considering combinations that contain no smaller combination that did not
yield a failing test.  With this optimization, the expense of blocking
becomes low enough to also apply the approach to the new
reduced tests found at each stage.  To block all previously discovered
reductions, it is necessary to compute combinations that include at
least one component from each reduced test produced thus far.

 Algorithm \ref{comb-block} shows
the formal definition of the {\tt comb-block} algorithm.  This algorithm depends on a
function {\tt block-all ($T$,$s$)}, which given a set of tests $T$ and
a combination size $s$,  returns all size $s$ combinations of components of tests $t
\in T$ such that each combination has at least one element
from each $t \in T$.  We omit the definition of {\tt block-all} in
the interests of space.  Our Python implementation \cite{tstl} simply
filters invalid combinations, which works well with the typically
small $s$ for reduced tests.

\begin{algorithm}
\caption{}
\label{comb-block}
\begin{algorithmic}[1]
\Require{failing test $t$, reduced failing test $r$, search depth
  $d$, max combinations to consider $m$}
\State {reductions = \{$r$\}; count = 0;}
\State {handled, notfailed = $\emptyset$;}
\While {$d$ > 0}
\State{new = $\emptyset$}
\For {$s$ = 1 to total components in reductions}
\For {$c \in$ {\tt block-all}(reductions-handled, $s$)}
\State {count = count + 1}
\If {count > $m$}
\Return {reductions}
\EndIf
\State {handled = handled $\cup$ \{$c$\}}
\If {$\neg \exists c' . c' \in c \wedge c' \in$ notfailed}
\If {fails($t-c$)}
\State{new = new $\cup$ \{\tt{ddmin}($t-c$)\}}
\Else
\State{notfailed = notfailed $\cup$ \{$c$\}}
\EndIf
\EndIf
\EndFor
\EndFor
\State{$d$ = $d - 1$}
\State{reductions = reductions $\cup$ new}
\EndWhile
\State {return reductions}
\end{algorithmic}
\end{algorithm}

\subsection{Randomized Multiple-Run ddmin}

Our second mitigation approach simply notes that slippage in {\tt ddmin} is
deterministic only because of the fixed order in which possible
reductions of a test  are considered.  However, if we randomize
the order of checks on smaller tests in {\tt ddmin}, it can produce different
reduced tests when executed with a different random seed.  This means
we can try to avoid slippage by simply running {\tt ddmin} multiple
times with different seeds.  The advantage of this {\tt multi-ddmin}
approach is that is can work even when no test removing
components of the original reduction fails.  The non-existence of such
tests does not mean
there is no alternative reduction, but that finding it must
require a different starting path for {\tt ddmin}.
This approach lacks {\tt comb-block}'s directed search
for dissimilar reductions and modifies the internals of {\tt ddmin}.
