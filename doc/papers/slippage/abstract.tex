Reducing the size of tests, typically by delta debugging or a related algorithm, is a critical component of effective automated testing and debugging.  Automatically generated or user-submitted tests are often far longer than required, full of unneccessary components that make debugging difficult.  Test reduction algorithms automatically remove components of such tests, whle preserving the property that the test fails.  Unfortunately, reduction can sometimes transform a failing test that detects a subtle, critical, and previously unknown fault into  a test that detects a trivial-to-find, unimportant, and already known fault.  When reducing a test detecting fault(s) $F$ produces a test that does not detect the same $F$, this is known as \emph{slippage}.  In the case where an interesting fault slips to an uninteresting fault, slippage is a problem, and must be avoided.  However, slippage can also be beneficial, when a long test can be reduced to detect a fault that has not otherwise been detected (including by the original test).  While traditional delta debugging only produces one reduced test, the concept of slippage suggests an alternative approach, where the output of reduction is a set of reduced tests, in order to avoid problematic slippage and induce beneficial slippage.
 In this paper, we present preliminary efforts to understand slippage, and compare two approaches to slippage mitigation.