\section{The ArcPy TSTL Test Harness}
\label{harness}

\begin{figure}
{\scriptsize
\begin{code}
@import shutil, os, glob, arcpy, exceptions, gc
@from arcpy import ExecuteError
<@
def cleanupFiles():
    gc.collect() \# Get rid of cursors
    for l in ["l1","l2","l3"]:
    	arcpy.Delete\_management(l)
    
    for f in glob.glob("C:\\Arctmp\\*"):
        try:
            shutil.rmtree(f)
        except:
            print "UNABLE TO REMOVE:",f
    for i in xrange(0,1000000): \# Find a workspace without a lock
        new\_workspace = "C:\\Arctmp\\workspace." + str(i)
        if not os.path.exists(new\_workspace):
            break             
    shutil.copytree("C:\\Arcbase",new\_workspace)
    arcpy.env.workspace = new\_workspace
    print sorted(glob.glob(arcpy.env.workspace + "\\*.shp")),
    print sorted(glob.glob(arcpy.env.workspace + "\\*.lyr")),
    print sorted(glob.glob(arcpy.env.workspace + "\\*.gdb"))


def fcsInGdb(gdb):
    old\_workspace = arcpy.env.workspace
    arcpy.env.workspace = gdb
    fcs = []
    for fds in arcpy.ListDatasets('','feature') + ['']:
        for fc in arcpy.ListFeatureClasses('','',fds):
            fcs.append(os.path.join(gdb,fds,fc))
    arcpy.env.workspace = old\_workspace
    return fcs

def report():
    print arcpy.GetMessages()
@>

pool: <shapefilelist> 2
pool: <shapefile> 2 CONST

pool: <gdbfilelist> 2
pool: <gdbfile> 2 CONST

pool: <gdbfeatureclasslist> 2

pool: <featureclass> 4 CONST
pool: <classorlayer> 4 CONST

pool: <classorlayerlist> 2

pool: <spatialref> 2

pool: <prjfilelist> 2
pool: <prjfile> 2 CONST

pool: <transformlist> 2
pool: <transform> 2 CONST

pool: <newlayer> 2 CONST

pool: <layerlist> 2
pool: <layer> 2

pool: <fieldname> 2 CONST
pool: <fieldtype> 2 CONST
pool: <fieldlist> 2
pool: <fieldnamelist> 2

pool: <stattype> 2 CONST
pool: <statfields> 2

pool: <dist> 2 CONST

pool: <sorttype> 2 CONST
pool: <spatialsort> 2 CONST

pool: <sort> 1
pool: <sortlist> 2

pool: <joinattributes> 2

pool: <overlaptype> 2 CONST
pool: <selectiontype> 2 CONST
pool: <op> 2 CONST
pool: <val> 2 CONST
pool: <whereclause> 2 CONST

pool: <errtable> 1 CONST
pool: <polytable> 1 CONST
pool: <stattable> 1 CONST

pool: <insertcursor> 3
pool: <searchcursor> 3
pool: <updatecursor> 3

pool: <irow> 3
pool: <srow> 3
pool: <urow> 3

init: cleanupFiles()

log: 1 arcpy.GetMessages()
\end{code}
}
\caption{ArcPy TSTL test harness definition preamble (pools, functions, logging).}
\label{preamble}
\end{figure}

\begin{figure}
{\scriptsize
\begin{code}
<gdbfilelist> := sorted(glob.glob(arcpy.env.workspace + "\\*.gdb"))
len(<gdbfilelist,1>) >= 1 -> <gdbfile> := <gdbfilelist> [0]
<gdbfilelist> = <gdbfilelist> [1:]

<shapefilelist> := sorted(glob.glob(arcpy.env.workspace + "\\*.shp"))
len(<shapefilelist,1>) >= 1 -> <shapefile> := <shapefilelist> [0]
<shapefilelist> = <shapefilelist> [1:]

<layerfile> := arcpy.env.workspace + <["\\new1.lyr", "\\new2.lyr", "\\new3.lyr"]>

<shapefile> := arcpy.env.workspace + <["\\new1.shp", "\\new2.shp", "\\new3.shp"]>

<prjfilelist> := sorted(glob.glob
   ("C:\\Program Files (x86)\\ArcGIS\\Desktop10.3\\Reference Systems\\*.prj"))
len(<prjfilelist,1>) >= 1 -> <prjfile> := <prjfilelist> [0]
<prjfilelist> = <prjfilelist> [1:]

<transformlist> := arcpy.ListTransformations(<spatialref>,<spatialref>)
<transformlist> = <transformlist> [1:]
len(<transformlist,1>) >= 1 -> <transform> := <transformlist> [0]

<newlayer> := <["l1", "l2", "l3"]>

<spatialref> := arcpy.SpatialReference(<prjfile>)

<gdbfeatureclasslist> := fcsInGdb(<gdbfile>)
<gdbfeatureclasslist> = <gdbfeatureclasslist>[1:]

<featureclass> := <shapefile>
len(<gdbfeatureclasslist,1>) >= 1 -> <featureclass> := <gdbfeatureclasslist>[0]

<classorlayer> := <featureclass>
<classorlayer> := <newlayer>

<classorlayerlist> := []
<classorlayerlist>.append(<classorlayer>)

<fieldtype> := <["TEXT", "FLOAT", "DOUBLE", "SHORT", "LONG", "DATE"]>

<fieldname> := <["newf1", "newf2", "newf3"]>

<dist> := <["100 Feet", "500 Feet", "1000 Feet", "1 Mile", "2 Miles">]

<joinattributes> := <["ALL", "NO\_FID", "ONLY\_FID"]>

<overlaptype> := <["INTERSECT", "CONTAINS", "COMPLETELY\_CONTAINS", "WITHIN",
   "SHARE\_A\_LINE\_SEGMENT\_WITH", "CROSSED\_BY\_THE\_OUTLINE\_OF"]>

<selectiontype> := <["NEW\_SELECTION", "ADD\_TO\_SELECTION", "REMOVE\_FROM\_SELECTION",
   "SUBSET\_SELECTION", "SWITCH\_SELECTION", "CLEAR\_SELECTION"]>

<sorttype> := <["ASCENDING", "DESCENDING">]

<spatialsort> := <["UR", "UL", "LR", "LL", "PEANO"]>

<sort> := [<fieldname>,<sorttype>]
<sortlist> := []
<sortlist>.append(<sort>)

\{IOError\} <insertcursor> := arcpy.InsertCursor(<classorlayer>)
\{IOError\} <insertcursor> := arcpy.InsertCursor(<classorlayer>,<spatialref>)
\{IOError\} <searchcursor> := arcpy.SearchCursor(<classorlayer>)
\{IOError,exceptions.RuntimeError\} <searchcursor> := arcpy.SearchCursor
   (<classorlayer>,<whereclause>)
\{IOError,exceptions.RuntimeError\} <searchcursor> := arcpy.SearchCursor
   (<classorlayer>,<whereclause>,<spatialref>)
\{IOError\} <updatecursor> := arcpy.UpdateCursor(<classorlayer>)
\{IOError\} <updatecursor> := arcpy.UpdateCursor(<classorlayer>,<spatialref>)

<irow> := <insertcursor>.newRow()
\{exceptions.RuntimeError\} <insertcursor>.insertRow(<irow>)

<irow> := <insertcursor>.next()
<urow> := <updatecursor>.next()
<srow> := <searchcursor>.next()

\{exceptions.RuntimeError\} <val> := <irow>.getValue(<fieldname>)
\{exceptions.RuntimeError\} <val> := <srow>.getValue(<fieldname>)
\{exceptions.RuntimeError\} <val> := <urow>.getValue(<fieldname>)

\{exceptions.RuntimeError\} <irow>.setValue(<fieldname>,<val>)

\{exceptions.RuntimeError\} <urow>.setNull(<fieldname>)
\{exceptions.RuntimeError\} <urow>.setValue(<fieldname>,<val>)
\{exceptions.RuntimeError\} <updatecursor>.deleteRow(<urow>)
\{exceptions.RuntimeError\} <updatecursor>.updateRow(<urow>)

<op> := <[">", "<", "<=", ">=", "=", "!=">]  

<val> := <["10", "20", "30", "100", "1000">]

<whereclause> := '"' + <fieldname> + '" ' + <op> + str(<val>)

<whereclause> := <whereclause> + ' AND ' + <whereclause>

<whereclause> := <whereclause> + ' OR ' +  <whereclause>

<whereclause> := 'NOT' + <whereclause>

<errtable> := arcpy.env.workspace + "\\geomerr.dbf"

<polytable> := arcpy.env.workspace + "\\polyneig.dbf"

<stattable> := arcpy.env.workspace + "\\stats.dbf"

\{IOError\} <fieldlist> := arcpy.ListFields(<classorlayer>)
len(<fieldlist,1>) >= 1 -> <fieldname> := <fieldlist> [0].name
<fieldlist> = <fieldlist> [1:]

<fieldnamelist> := []
<fieldnamelist>.append(<fieldname>)

<stattype> := <["SUM", "MEAN", "MIN", "MAX", "RANGE", "STD", "COUNT", "FIRST", "LAST"]>

<statfields> := []
<statfields>.append([<fieldname>,<stattype>])
\end{code}
}
\caption{ArcPy TSTL test harness actions, part 1.}
\label{actions1}
\end{figure}

\begin{figure}
{\scriptsize
\begin{code}
\{ExecuteError\} arcpy.MakeFeatureLayer\_management(<featureclass>,<newlayer>); report()

\{ExecuteError\} arcpy.MakeFeatureLayer\_management(<featureclass>,<newlayer>,
   where\_clause=<whereclause>); report()

\{ExecuteError\} arcpy.Project\_management(<featureclass>,<featureclass>,<spatialref>,
   <transform>); report()

\{ExecuteError\} arcpy.AddField\_management(<featureclass>,<fieldname>,<fieldtype>);
   report()

\{ExecuteError\} arcpy.DeleteField\_management(<featureclass>,<fieldname>); report()

\{ExecuteError\} arcpy.Buffer\_analysis(<classorlayer>,<featureclass>,<dist>); report()

\{ExecuteError\} arcpy.Buffer\_analysis(<classorlayer>,<featureclass>,<dist>,
   dissolve\_option="ALL"); report()

\{ExecuteError\} arcpy.Buffer\_analysis(<classorlayer>,<featureclass>,<dist>,
   dissolve\_option="LIST",dissolve\_field=<fieldnamelist>); report()

\{ExecuteError\} arcpy.Erase\_analysis(<classorlayer>,<classorlayer>,<featureclass>);
   report()

\{ExecuteError\} arcpy.Erase\_analysis(<classorlayer>,<classorlayer>,<featureclass>,
   cluster\_tolerance=<dist>); report()

\{ExecuteError\} arcpy.Intersect\_analysis(<classorlayerlist>,<featureclass>); report()

\{ExecuteError\} arcpy.Intersect\_analysis(<classorlayerlist>,<featureclass>,
   join\_attributes=<joinattributes>); report()

\{ExecuteError\} arcpy.Intersect\_analysis(<classorlayerlist>,<featureclass>,
   cluster\_tolerance=<dist>); report()

\{ExecuteError\} arcpy.Intersect\_analysis(<classorlayerlist>,<featureclass>,
   join\_attributes=<joinattributes>,cluster\_tolerance=<dist>); report()

\{ExecuteError\} arcpy.Union\_analysis(<classorlayerlist>,<featureclass>); report()

\{ExecuteError\} arcpy.Union\_analysis(<classorlayerlist>,<featureclass>,
   join\_attributes=<joinattributes>); report()

\{ExecuteError\} arcpy.Union\_analysis(<classorlayerlist>,<featureclass>,
   cluster\_tolerance=<dist>); report()

\{ExecuteError\} arcpy.Union\_analysis(<classorlayerlist>,<featureclass>,
   join\_attributes=<joinattributes>,cluster\_tolerance=<dist>); report()

\{ExecuteError\} arcpy.SpatialJoin\_analysis(<classorlayer>,<classorlayer>,
   <featureclass>); report()

\{ExecuteError\} arcpy.SymDiff\_analysis(<classorlayer>,<classorlayer>,<featureclass>);
   report()

\{ExecuteError\} arcpy.SymDiff\_analysis(<classorlayer>,<classorlayer>,<featureclass>,
   join\_attributes=<joinattributes>); report()

\{ExecuteError\} arcpy.SymDiff\_analysis(<classorlayer>,<classorlayer>,<featureclass>,
   join\_attributes=<joinattributes>,cluster\_tolerance=<dist>); report()

\{ExecuteError\} arcpy.PolygonNeighbors\_analysis(<classorlayer>,~<polytable>); report()

\{ExecuteError\} arcpy.Statistics\_analysis(<classorlayer>,~<stattable>,<statfields>);
   report()

\{ExecuteError\} arcpy.SelectLayerByLocation\_management(<newlayer>,
   select\_features=<newlayer>,overlap\_type=<overlaptype>)

\{ExecuteError\} arcpy.SelectLayerByLocation\_management(<newlayer>,
   select\_features=<newlayer>,overlap\_type=<overlaptype>,search\_distance=<dist>)

\{ExecuteError\} arcpy.SelectLayerByLocation\_management(<newlayer>,
   select\_features=<newlayer>,overlap\_type=<overlaptype>,search\_distance=<dist>,
   selection\_type=<selectiontype>)

\{ExecuteError\} arcpy.SelectLayerByAttribute\_management(<newlayer>,
   selection\_type=<selectiontype>,where\_clause=<whereclause>)

\{ExecuteError\} arcpy.Select\_analysis(<classorlayer>,<featureclass>,
   where\_clause=<whereclause>)

\{ExecuteError\} arcpy.CopyFeatures\_management(<featureclass>,<featureclass>); report()

\{ExecuteError\} arcpy.Sort\_management(<featureclass>,<featureclass>,<sortlist>);
   report()

\{ExecuteError\} arcpy.Sort\_management(<featureclass>,<featureclass>,<sortlist>,
   <spatialsort>); report()

\{ExecuteError\} arcpy.Sort\_management(<featureclass>,<featureclass>,
   [["Shape",<sorttype>]],<spatialsort>); report()

\{ExecuteError\} arcpy.CheckGeometry\_management(<classorlayer>,~<errtable>); report()

\{ExecuteError\} arcpy.CheckGeometry\_management(<classorlayerlist>,~<errtable>); report()

\{ExecuteError\} arcpy.Delete\_management(<featureclass>); report()
\end{code}
}
\caption{ArcPy TSTL test harness definition actions, part 2.}
\label{actions2}
\end{figure}



Figures \ref{preamble}-\ref{actions2} show a
version of the actual ArcPy test harness.  This version can reproduce the faults
described in this paper, though in practice some faults are easier to
detect than others, and for practical testing it is best to disable,
e.g., the various {\tt Delete} calls and to use guards to prevent all modifications of
layers or classes on which cursors are active.  Compiled to Python,
this harness defines nearly 2,000 actions, and the standalone
interface is nearly 60KLOC.  Using the harness involves first
compiling it to a Python class, then loading a test case generator
such as the random tester provided with TSTL into a Python environment
that has access to ArcPy.  For our experiments, we compiled the TSTL
using the Cygwin Python installation (for easy command-line access) but ran
tests in the IDLE environment installed with ArcGIS.
Running tests involves no complications beyond modifying parameters of
the random tester, if desired (setting a time limit for testing, the
length of test cases \cite{ASE08}, and whether to search for failures or produce
coverage regression tests, for example).  In most cases, the random tester
eventually terminates abnormally, and the test case causing the crash
is stored in a file in the ArcPy harness directory.  For regression
generation, the tool produces files of the same format to obtain
coverage of ArcPy code.

Figure \ref{preamble} contains the small amount of code needed to
prepare a temporary workspace for each test sequence.  This code
handles deleting any live cursors (the pool variables are guaranteed
to be deleted before each test, and garbage collection ensures the
cursors are deactivated), removing any layers created on feature
classes, and then setting the environment.  The system will scan for a
``free'' environment location, to handle cases where locks prevent
re-using an old directory\footnote{The problem of locked directories
  seldom surfaces, now that layers are deleted before each test, but
  may be needed when saving states.}.  The other utility functions
allow discovering the feature classes in a geodatabase and produce a
report on screen when a complex ArcGIS operation completes
successfully.

Figure \ref{actions1} shows the actions that create
input parameters for ArcGIS engine calls, primarily.  Many of these
simply pick some numeric or string constant (and the sets of constants
could be expanded, at the cost of more expensive normalization and
generalization).  Others, such as SQL query generation, are more
complex, with recursive expansion to theoretically unlimited query
length.  

Finally, Figure \ref{actions2} shows the TSTL for the actual
ArcGIS toolbox calls.  So far, this includes only a small subset of
the functions defined in the Management and Analysis toolboxes.  Note
that after each call there is a call to {\tt report}.  In pure random
testing, successful calls are infrequent enough that it is useful
for the user to see the ArcGIS messages produced by successful calls.
By turning on logging, the user can also see unsuccessful call
messages, but this tends to produce an overwhelming amount of output.

One critical design decision taken early in the development of this
harness was that there is no explicit definition of the starting data
used for testing.  Any data stored in the {\tt Arcbase} directory can
be used, and the harness supports both shapefiles and file geodatabases.
This has two purposes:  first, testing can be customized to use any
starting data, including a user's own shapefiles.  Second, this makes
it easy to implement deep state testing, since no assumptions are made
about the structure or number of data files used.  The lack of
dependence on specific files is implemented by having the test harness
use Python's {\tt glob} function to collect all files of a given
extension in a directory in a list, then chose an item from the list.
In order to make sure that behavior is deterministic (up to the choice
of actual data files), the glob results are sorted.  This approach
does bias random testing to using files earlier in alphabetical order,
but we do not expect base testing data to include a large number of
files.

The key driving requirements for this harness design are given in the
title of this paper:  the test harness must be \emph{extensible} and
it must be \emph{usable}.

Because ArcPy is large and complex, the effort to produce a
complete test harness and more effective specification of correctness
is a long-term effort, and may be carried out by other users:  the
harness therefore must be \emph{extensible}.  The
harness should also be suitable for use by ArcPy developers testing
how their own extensions to ArcPy interact with the base ArcPy API.
Adding new API calls to the harness should be easy:  we have defined
the pools for basic ArcPy object types and provided tools that should
handle much more complex test cases.  The decision discussed above to
make the test harness independent of specific data files is a good
example of our emphasis on extensibility.

The idea that other ArcPy developers, testing researchers, and perhaps
(end-user) software developers in other fields looking for a large
scale example of how to test systems with TSTL, will need to read and
modify the code also drives our emphasis on a truly usable system.
Usability has a more direct impact on the modifications to the TSTL
language and tools than on the test harness itself, but the test
harness is developed in the context of the improvements to the TSTL
ecosystem, such as standalone tests and normalization and
generalization.  However, some usability choices are decisions about
how to write the harness.  Consider the calls to {\tt
  SelectLayerByLocation\_management} shown in Figure \ref{actions2}.
The three lines of code could be more concisely expressed in one line using the {\tt <,[} construct:

\begin{code}
\{ExecuteError\} arcpy.SelectLayerByLocation\_management(
   <newlayer>,select\_features=<newlayer>,overlap\_type=
   <overlaptype><,[,search\_distance=<dist>,,],>
   <,[,selection\_type=<selectiontype>,,],>)
\end{code}

However, this is difficult to read, even for the TSTL developer, and
seems likely to discourage GIS developers trying to understand and
extend the harness.  There is a tension between, on the one hand, concise, abstracted
expression of all possible test actions and, on the other hand,
concrete readable connection between test actions and the lines of code that
appear in real ArcPy scripts.  We aim to stay closer to concrete
forms, even at some cost in increased length for the harness.  An
interesting observation is that this preference on the one hand makes
changing the harness more difficult --- to change the name of an API
call requires multiple edits if multiple parameter arities are used;
on the other hand, changing the code using {\tt <,[} requires
understanding the meaning of the code, each time, which resembles the
difficulty of altering a complex regular expression \cite{RegExp}.  In
this case, the best solution might be to add a specialized construct
to TSTL to handle optional elements of an action --- despite the fact
that the {\tt <,[} construct can express optional elements easily.
Even with such an option, it may be easier for developers to read and
understand the implications of individual calls, however, if they are
written out in the harness, rather than combinatorally generated by
the TSTL compiler.

In the long run, these issues are as complex as the questions of
abstraction vs. ease-of-understanding that have concerned designers
and users of programming languages since early in the history of
computer science.  As test definition comes closer to a specialized
kind of declarative programming, our understanding of the tradeoffs
will likely improve.