%%%%%%%%%%%%%%%%%%%%%%% file template.tex %%%%%%%%%%%%%%%%%%%%%%%%%
%
% This is a general template file for the LaTeX package SVJour3
% for Springer journals.          Springer Heidelberg 2010/09/16
%
% Copy it to a new file with a new name and use it as the basis
% for your article. Delete % signs as needed.
%
% This template includes a few options for different layouts and
% content for various journals. Please consult a previous issue of
% your journal as needed.
%
%%%%%%%%%%%%%%%%%%%%%%%%%%%%%%%%%%%%%%%%%%%%%%%%%%%%%%%%%%%%%%%%%%%
%
% First comes an example EPS file -- just ignore it and
% proceed on the \documentclass line
% your LaTeX will extract the file if required
\begin{filecontents*}{example.eps}
%!PS-Adobe-3.0 EPSF-3.0
%%BoundingBox: 19 19 221 221
%%CreationDate: Mon Sep 29 1997
%%Creator: programmed by hand (JK)
%%EndComments
gsave
newpath
  20 20 moveto
  20 220 lineto
  220 220 lineto
  220 20 lineto
closepath
2 setlinewidth
gsave
  .4 setgray fill
grestore
stroke
grestore
\end{filecontents*}
%
\RequirePackage{fix-cm}
%
%\documentclass{svjour3}                     % onecolumn (standard format)
%\documentclass[smallcondensed]{svjour3}     % onecolumn (ditto)
\documentclass[smallextended]{svjour3}       % onecolumn (second format)
%\documentclass[twocolumn]{svjour3}          % twocolumn
%
\smartqed  % flush right qed marks, e.g. at end of proof
%
\usepackage{graphicx}
\usepackage{code}
\usepackage{xcolor}
%
% \usepackage{mathptmx}      % use Times fonts if available on your TeX system
%
% insert here the call for the packages your document requires
%\usepackage{latexsym}
% etc.
%
% please place your own definitions here and don't use \def but
% \newcommand{}{}
%
% Insert the name of "your journal" with
% \journalname{myjournal}
%
\begin{document}

\title{Extensible and Usable Testing for Geographic Information System Automation}
%\subtitle{Do you have a subtitle?\\ If so, write it here}

%\titlerunning{Short form of title}        % if too long for running head

\author{Josie Holmes         \and
  Alex Groce           \and
  James O'Brien
}

%\authorrunning{Short form of author list} % if too long for running head

\institute{Alex Groce \at
              School of Electrical Engineering and Computer Science\\
              Oregon State University\\
              \email{agroce@gmail.com}           %  \\
%             \emph{Present address:} of F. Author  %  if needed
           \and
           Josie Holmes \at
             Department of Geography\\
             Pennsylvania State University\\
           \email{jdh396@psu.edu}
           \and
           James O'Brien \at
          Risk Frontiers\\
           Macquarie University\\
          \email{James.OBrien@mq.edu.au}
}

\date{Received: date / Accepted: date}
% The correct dates will be entered by the editor


\maketitle

\begin{abstract}
Test case reduction has long been seen as essential to effective automated testing.  However, test case reduction simply reduces the \emph{length} of a test case.  It does not attempt to reduce \emph{semantic} complexity.  This paper presents algorithms for normalizing and generalizing test cases.  Rewriting test cases into a \emph{normal form} can reduce semantic complexity and, often, remove steps from an already delta-debugged test case.  Normalization, more importantly, reduces the \emph{number} of test cases that a reader must examine, partially addressing the ``fuzzer taming'' problem of discovering all faults in a large set of failing test cases.  Generalization, in contrast, takes a test case and reports what aspects of the test could have been changed while preserving the property that the test fails.  These algorithms rely on the features of a recently introduced domain-specific language, TSTL.  Normalization plus generalization aids understanding of test cases, including tests for TSTL itself and for complex and widely used Python APIs such as the NumPy numeric computation library and the ArcPy GIS scripting package.  Normalization frequently reduces the number of test cases to be examined by \emph{over an order of magnitude}, and often to just one test case per fault.  Together, ideally, normalization and generalization allow a user to replace reading a large set of test cases that vary in unimportant ways with reading \emph{one annotated test case, summarizing an entire family of similar failures}.
\keywords{Automated testing
  \and End-user developers \and Geographic Information System \and
  Testing languages \and Debugging}
% \PACS{PACS code1 \and PACS code2 \and more}
% \subclass{MSC code1 \and MSC code2 \and more}
\end{abstract}

\section{Introduction}

It has long been understood that effective automated testing requires
test case reduction \cite{DD,MinUnit,ICSEDiff} to produce test cases
that remove irrelevant operations.  In fact, test case reduction is
now standard practice in industral testing tools such as Mozilla's
{\tt jsfunfuzz}.  However, simply reducing the length of a test case
does not produce any kind of semantic simplicity.  There may be many
1-minimal test cases that present different variations of a single
fault.  In many cases, reading more than one of these test cases
provides no useful additional information on the cause of failure.

Consider the three test cases shown in Figure \ref{threetests}.  These
test cases are obviously very similar, and in fact all lead to a
violation of the property that an AVL tree must always be nearly
balanced, due to a missing call to {\tt rebalance} in {\tt delete} in
a Python implementation of AVL trees.  However, the test cases are
syntactically very different, and a testing system that collects
failing test cases will produce three tests for a user to examine.
While there are methods for attempting to determine which test cases
represent distinct faults \cite{PLDI13}, the ideal solution is
arguably to rewrite all three of these test cases into a single,
normal form that preserves the structure of the failure while removing
such accidental aspects of each test case as the particular integer
values and variables used, and the ordering of assignments and
insertions.

Figure \ref{normalgen} shows the result of applying our \emph{test case
  normalization} method to these three tests, and then applying our
\emph{test case generalization} approach to the normalized test case.
\emph{All three test cases normalize to the same test
case}.  This single test case, in addition to the 10 steps required to produce
the failure, includes comments indicating what about the test case can
be changed while still failing in the same way.  For instance, the
value 1 assigned to {\tt int0} in step 0 is not essential.  It could
be changed to any value in the range 5-20 (the total set of values
allowed by the test generator) without changing the final result.  The
same is true of the assignment of 3 to {\tt int1}.  Similarly, the
exact ordering of many steps in the test case is not important.  Finally,
step 9 is annotated to show that instead of using the existing value
of {\tt int1} (4), a fresh assignment could be inserted before the
{\tt delete} call, setting {\tt
  int1} to 3 instead.  These possible
changes are not meant to be combined --- the annotation claims only
that changing these aspects of the test case one at a time will
preserve failure.

Combining normalization and generalization avoids some common problems
with understanding automatically generated test cases.  For instance,
when a large integer appears in such a test case, the question always
arises --- is this unusual value important, or just a randomly chosen
number of no significance \cite{MakeMost}?  After normalization, any
large values in a test case are sure to be essential, rather than
accidental, because normalization includes value minimization.
Without the additional step of generalization, however, it would be
easy to conclude that \emph{all} small numeric values in normalized tests are
accidental.  Generalization informs a user when a small value is
required to reproduce a failure, and when it is simply an artifact of normalization.

Normalization is not a complete solution to the problem of
identifying distinct faults (one key limitation is that our algorithms do not apply to
complex custom test generators such as CSmith \cite{csmith} or {\tt
  jsfunfuzz} \cite{jsfunfuzz}), but it is highly effective when it applies,
in our experience.
Running 100,000 tests (of length 100) on the faulty AVL tree produces
860 failing test cases with no duplicates.  Normalizing these reduces
the number of distinct failing test cases to only 22.  Of course,
ideally \emph{all} failures due to the same fault would normalize to a
single, representative test case, but we can only aim to approximate
such a canonical form for faults.  Figure \ref{diffnorm} shows a test
case that normalizes differently, and its normalized form (we omit the
generalization, which is not interestingly different than that for the
first normalized test case).

The contributions of this paper are 1) the idea of normalizing and
generalizing test cases, 2) algorithms for normalizing and
generalizing test cases that make use of the abstract graph interface
for testing provided by the TSTL \cite{NFM15,ISSTA15} testing
language, and 3) some initial experimental results showing the value
of test case normalization and generalization.  We show that
normalization and generalization has been key to efforts to understand
complex failing test cases for a widely-used, highly complex library
for GIS automation.

\begin{figure}
{\scriptsize
{\bf Test case \#1}
\begin{code}
avl0 = avl.AVLTree() 
int0 = 4 
int2 = 13 
int3 = 7 
avl0.insert(int2) 
avl0.insert(int3) 
int1 = 15 
avl0.insert(int1) 
avl0.insert(int0) 
avl0.delete(int2)
\end{code}
{\bf Test case \#2}
\begin{code}
int0 = 14 
avl0 = avl.AVLTree() 
int2 = 13 
int1 = 15 
avl0.insert(int1) 
int1 = 11 
avl0.insert(int2) 
avl0.insert(int0) 
avl0.insert(int1) 
avl0.delete(int0) 
\end{code}
{\bf Test case \#3}
\begin{code}
avl1 = avl.AVLTree() 
int3 = 18 
avl1.insert(int3) 
int0 = 5 
int3 = 12 
avl1.insert(int0) 
int0 = 15 
avl1.insert(int0) 
avl1.insert(int3) 
int1 = 15 
avl1.delete(int1) 
\end{code}
}
\caption {Three randomly generated test cases for the same fault.}
\label{threetests}
\end{figure}

\begin{figure}
{\scriptsize
\begin{code}
\textcolor{black!35}{\#[}
int0 = 1                              \textcolor{black!35}{\# STEP 0}
\textcolor{black!35}{\#  or int0 = 5 }
\textcolor{black!35}{\#   - int0 = 20} 
\textcolor{black!35}{\#  swaps with step 4}
int1 = 3                              \textcolor{black!35}{\# STEP 1}
\textcolor{black!35}{\#  or int1 = 5 }
\textcolor{black!35}{\#   - int1 = 20} 
\textcolor{black!35}{\#  swaps with step 6}
avl0 = avl.AVLTree()                  \textcolor{black!35}{\# STEP 2}
\textcolor{black!35}{\#] (steps in [] can be in any order)}
avl0.insert(int0)                     \textcolor{black!35}{\# STEP 3}
\textcolor{black!35}{\#[}
int0 = 2                              \textcolor{black!35}{\# STEP 4}
\textcolor{black!35}{\#  swaps with step 0}
avl0.insert(int1)                     \textcolor{black!35}{\# STEP 5}
\textcolor{black!35}{\#] (steps in [] can be in any order)}
int1 = 4                              \textcolor{black!35}{\# STEP 6}
\textcolor{black!35}{\#  or int1 = 5 }
\textcolor{black!35}{\#   - int1 = 20} 
\textcolor{black!35}{\#  swaps with step 1}
avl0.insert(int1)                     \textcolor{black!35}{\# STEP 7}
avl0.insert(int0)                     \textcolor{black!35}{\# STEP 8}
avl0.delete(int1)                     \textcolor{black!35}{\# STEP 9}
\textcolor{black!35}{\#  or (}
\textcolor{black!35}{\#      int1 = 3  ;}
\textcolor{black!35}{\#      avl0.delete(int1) }
\textcolor{black!35}{\#     )}
\end{code}
}
\caption{Normalization and generalization for all three test cases.
  Lines beginning with \# are comments in Python, used for annotations.}
\label{normalgen}
\end{figure}

\begin{figure}
{\scriptsize
{\bf Test case \#4:}
\begin{code}
int0 = 10 
int2 = 7 
avl1 = avl.AVLTree() 
avl1.insert(int2) 
avl1.insert(int0) 
int1 = 1 
int3 = 1 
avl1.insert(int3) 
int3 = 15 
avl1.insert(int3) 
avl1.delete(int1) 
\end{code}
{\bf Normalized:}
\begin{code}
int0 = 1
int1 = 2
avl0 = avl.AVLTree()
avl0.insert(int0) 
avl0.insert(int1) 
int1 = 3  
avl0.insert(int1) 
int1 = 4  
avl0.insert(int1)  
avl0.delete(int0) 
\end{code}
}
\caption{A differently normalized test case for the same fault}
\label{diffnorm}
\end{figure}
\section{Related Work}

This work builds on the idea behind delta-debugging \cite{DD}: tests should not contain extraneous information that is not needed to
reproduce failure (or some other behavior \cite{icst2014,stvrcausereduce}).  Delta-debugging and slicing
\cite{TCminim} are limited, generally, to producing subsets of the
original test, not modifying parts of the test to obtain further
simplicity.  We extend this concept by also allowing modification or
re-ordering, which also allows further length reduction.

%Some earlier work in bounded model checking modified counterexamples to use
%numerically smaller values \cite{MakeMost} but otherwise did not aim
%to simplify or normalize failures.

Normalization is in part motivated by the fuzzer taming \cite{PLDI13}
problem: determining how many distinct faults are present in a large
set of failing tests.  This is a key problem in practical
application of automated testing.  Previous work on fuzzer taming
\cite{PLDI13} used delta-debugging to reduce some tests to
syntactic duplicates.

Zhang \cite{SaiSimple} proposed an alternative approach to semantic
test simplification that, like our approach, is able to modify, rather
than simply remove, portions of a test.  However, because Zhang
operates directly over a fragment of the Java language, rather than
using an abstraction of test actions allowed, the set of rewrite
operations performed is highly restricted: no new methods can be
invoked, statements cannot be re-ordered, and no new values are used.
These restrictions limit the approach's ability to simplify tests and
make it inappropriate for  normalization, as opposed to simplification.  The approach also performs little
syntactic normalization: e.g., it does not even force a test to use
fixed variable names when variable name is irrelevant.  CReduce
\cite{CReduce} performs some simple normalization as part of a complex
test reduction scheme for C code, and the peephole-rewrite scheme
used in CReduce is also an inspiration for the approach taken by our
normalizer.

Work on automatically producing readable tests \cite{Guava,Readable} is also
related, in that it aims to ``simplify'' tests.  Readable tests are
intended to assist debugging by humans, while our
normalization and generalization aims to increase the information
density of a test, further reduce length, and address the fuzzer taming problem.  The approaches are
orthogonal and could likely be profitably combined: users might be
best served by normalized, generalized tests modified to improve readability.

The most closely related work to our generalization efforts is Pike's
SmartCheck \cite{SmartCheck}.  SmartCheck targets algebraic data in
Haskell, and offers an interesting alternative approach to reduction
and generalization.  Test generalization is also akin to dynamic invariant generation,
in that it informs the user of invariants over a series of test
executions \cite{Daikon}.  The only other work we are aware of that is
similar to generalization concerns essential and accidental aspects of
model checking counterexamples \cite{FreeWill,MakeMost,SPIN03}.  
\section{A Brief Primer on TSTL}

\begin{figure}
{\scriptsize
\begin{code}
@import avl
\vspace{0.05in}
pool: <int> 4 CONST
pool: <avl> 3
\vspace{0.05in}
property: <avl>.check\_balanced()
\vspace{0.05in}
<int> := <[1..20]>
<avl> := avl.AVLTree()
\vspace{0.05in}
<avl>.insert(<int>)
<avl>.delete(<int>)
<avl>.find(<int>)
<avl>.inorder()
\end{code}
}
\caption{Part of a TSTL definition of AVL tree tests.}
\label{fig:example}
\end{figure}


\begin{figure}
{\scriptsize 
\begin{code}
avl1 = avl.AVLTree()  
int3 = 10  
int1 = 11  
avl1.insert(int1) 
int1 = 1  
avl1.insert(int3) 
avl1.insert(int1) 
int3 = 9  
avl1.insert(int3) 
int2 = 11  
avl1.delete(int2) 
\end{code}
}
\caption{An example TSTL-produced test}
\label{fig:avlrun}
\end{figure}


TSTL \cite{NFM15,ISSTA15,tstl} is a language for defining the structure of
test cases (usually API-call sequences, but also grammar-based tests using
string construction), and a set of tools for use in generating,
manipulating, and understanding those test cases.  Figure
\ref{fig:example} shows a simplified portion of a TSTL definition of
tests of an AVL tree class, in the latest syntax for TSTL (which
differs slightly from that in the cited papers introducing TSTL).
TSTL provides numerous features not shown in this small example,
including automatic differential testing, complex logging, support for
complex guards, and use of pre- and post- values.  Given a harness
like the one in Figure \ref{fig:example}, TSTL compiles it into a
class file defining an interface for testing that provides features
such as querying the set of available testing actions, restarting a
test, replaying a test, collecting code coverage data, and so forth.
The TSTL release \cite{tstl} provides testing tools that use the
interface for test generation and debugging.

The key point for our purposes is merely that a TSTL test harness
defines a set of \emph{pools} that hold values produced and used
during testing \cite{AndrewsTR} (a common approach to defining
API-testing sequences) and a set of actions that are possible during
testing, typically API calls and assignments to pool values.  In this
example, there are two pools, one named {\tt int} and one named {\tt
  avl}.  There are four instances of the {\tt int} pool, which means
that a test in progress can store up to 4 {\tt int}s at one time (in
variables named {\tt int0}, {\tt int1}, {\tt int2}, and {\tt int3}), and three
instances of the {\tt avl} pool.  The actions defined here are setting
the value of an {\tt int} pool to any integer in the range 1-20
inclusive, setting the value of an {\tt avl} pool to a newly
constructed AVL tree, and calling an AVL tree's {\tt insert}, {\tt
  delete}, {\tt find} and {\tt inorder} methods.  Figure
\ref{threetests} in the introduction shows three
valid test cases produced by running a random test generator on
the TSTL-compiled interface produced by this definition.  TSTL handles
ensuring that tests are well-formed: for example, no pool instance
(such as {\tt avl1} can appear in an action until it has been assigned
a value), and no pool instance that has been assigned a value can be
assigned a different value until it has been used in an action, to
avoid degenerate sequences such as {\tt int3 = 10} followed by {\tt
  int3 = 4}.  Each action in a test case is called a ``step'' --- the
first step of the first test case in Figure \ref{threetests} is storing a new AVL tree in {\tt
  avl0}, for example.


\section{Language Changes}

The original syntax for TSTL \cite{NFM15} used the {\tt \%} sign to
indicate TSTL constructs and pool variables.  Unfortunately, this
produced code that was difficult to read.  The first author suggested
using a notation more similar to that used to describe the grammar of
TSTL itself, enclosing pools in angle brackets.  Because C++ and a few
other languages use angle brackets for other purposes, and in order to
avoid breaking old harnesses, TSTL continues to allow the {\tt \%}
notation, but future TSTL harnessess for Python will use the more
readable syntax.

We are also considering, as a result of the experience of developing
the ArcPy harness, moving to a more structured form for TSTL files.
The current language allows pool definitions, actions, logging code,
raw Python code, and all other TSTL elements to be freely mixed,
without any requirements as to order.  Each line must indicate if it
is not an action definition, with some prefix such as {\tt pool:},
{\tt logging:}, {\tt reference:}, etc.; in practice, however, TSTL
harnessess are always written in an ordered style, with raw code
first, then pool definitions, properties, and logging information,
followed by a long section of action definitions.  Enforcing this
would allow all pool declarations to be prefaced by a single {\tt
  pool:} line at the beginning of the pool definitions, raw Python
code to be contained in a section marked{\tt raw:}, and all other
non-action declarations to be handled in the same way. 

There is also be a need for richer structure to avoid repeated
elements in action definitions.  For example, in the TSTL harness, 36
actions allow the {\tt arcpy.ExecuteError} exception to be raised,
which has to be stated for every individual action, and to avoid some
faults a large number of actions may eventually be disallowed for
feature classes or layers with active cursors.  Introducing
nested action groups, which can share guards, allowed exceptions, and post-conditions
could make reading complex TSTL code easier.  We are currently working to
define these language changes, without breaking existing TSTL code.
Discovering the need for this kind of feature without testing a system
as complex as ArcPy would be difficult.

\section{Tool Changes}

Rather than major algorithmic innovations, the major changes to the
TSTL system required to test ArcPy were fundamentally engineering
challenges.  Only the last two changes described in this section are
concerned with novel testing algorithms and support a research
agenda.  One element that carried across all of these concerns was
fixing bugs in TSTL.  TSTL has been used in several university classes
at the graduate and undergraduate level, and used fairly extensively
in testing research, but a large number of
significant faults went undetected until we attempted to use TSTL to
test ArcPy.  One change made as a result of these problems is that we
now use TSTL to test TSTL's own API.

\subsection{Sandboxed Test Case Execution}

Previous use of TSTL had been limited to testing systems where failure
resulted in an uncaught exception or a bad return value from a call,
at worst.  With ArcPy, however, it is very common for a failure to
cause a crash, killing not only ArcPy but the Python environment
running the test case.

We added two features to TSTL's test generators to handle this
problem.  First, we modified the random tester to record each action
to a test case log \emph{before the action is performed}, in order to recover a crashing
test after the test generator terminates abnormally.  After further experimentation,
we discovered that recording just the current test case was not
sufficient; some ArcPy failures required maintaining a history of all
executed actions, since the corruption carried across reimports of the
Python module.  Second, we produced a function
to enable running a TSTL case in its own Python subprocess, to allow
test reduction, normalization, and generalization even of crashing
test cases.  It was not neccessary to modify the TSTL interface's
reduction and other functions, as they already took an arbitrary
function as reduction predicate, allowing us to simply produce a
``sandboxed'' version of replay and pass that to TSTL's {\tt reduce}, {\tt
  normalize} etc. calls.  The sandboxing in this case is minimal:  the
only resource limitation is that the sandboxed execution has its own
Python process and writes to a temporary copy of the ArcGIS workspace,
but in principle the approach can be extended to allow more
restrictive test case jails in TSTL, including execution in a virtual machine.

\subsection{Standalone Test Case Generation}

TSTL test cases are saved as text files, where each line of the text
file contains a string representation of an action (with a unique
such string ``name'' for each action).  This format is somewhat human
readable, but is not machine-executable without the assistance of the
TSTL interface.  These test cases are also somewhat misleading for
readers, since guards, post-conditions, reference pool actions, and
allowed exceptions do not appear in this format.

Publishing test cases (or submitting them to Esri) is impractical,
however, if it requires use of the entire TSTL toolchain.  Moreover,
making tests only work within TSTL means it is not possible to
experiment with changing test cases in ways that are not in the TSTL
action set.  It is not  even possible to add print statements to help understand
behavior, without using TSTL's complex logging mechanisms.  

In order to address these problems, we implemented a new TSTL utility
that takes a test case stored in TSTL's internal format and produces a
standalone Python file that does not require any TSTL support.  The
standalone test case generator has options to control whether the
generated test case includes actions on reference pools, property
checks, and handling for allowed exceptions (omitted only if the exception does not
actually take place when the test is replayed).  The reason for disabling
the last functionality is interesting:  most test cases that are
stored are minimized \cite{DD} test cases, from which all actions not
neccessary to produce a failure or obtain desired code coverage have
been removed.  In most cases, this means that most ArcPy calls are
successful.  Adding code to handle potential exceptions makes
standalone test cases much longer and harder to read.

In the process of producing the standalone test case utility, we
introduced a more readable format for TSTL action names that is closer
to the code in standalone test cases.  This has, to our surprise, made
reading TSTL test cases in tool output, not just in standalone test
cases, much easier.  The new format has been integrated into new TSTL features.

\subsection{Regression Generation}

One difficulty for ArcPy users is ensuring that their existing scripts
and tools work on new versions of ArcGIS.  Each recent major release
(10.2 and 10.3) after ArcPy's introduction has potentially included some changes
in the behavior of API calls.  Detecting when such changes cause a
script to break is difficult.  A first step would be an automatic way
to find when the return values for calls differ between ArcPy
versions.
Because installing multiple versions of ArcGIS on the same system is
difficult or impossible, our method for finding differences relies on
choosing a reference version (10.3 in our current efforts), and
generating a set of standalone tests that 1) cover a large amount of
ArcPy functionality, including invalid inputs to functions and 2)
record the return values and exceptions raised by calls.  These tests
can be run on any ArcPy version, and will report differences between
the tests and version 10.3.  Performing this kind of differential
testing \cite{Differential} on old or new major releases, or across 64
bit and 32 bit versions, is easy.  In the long run, we also want to
enabled TSTL to produce Python 3.0+ code, for use with ArcGIS Pro,
which uses Python 3.4 instead of 2.7.  This has motivated a branch to
TSTL to support Python 2.7 (unfortunately, Python 3.0 is not fully
backwards compatible with earlier versions, and Python 2.7 is still
the most widely used Python).

We generate regression tests using an approach called \emph{quick
  testing} \cite{icst2014,stvrcausereduce}, which takes a set of tests
produced by random testing, and applies a test case reduction
algorithm \cite{DD} to produce smaller tests that have the same code
coverage as the very large, highly redundant, original set of test
cases.  Automatic quick-testing was added to TSTL's random test
generator to support ArcPy testing.  Combined with standalone test
generation, this allows us to produce test cases that can be run on
any version of ArcPy, and explore a large variety of behavior of the
code.  With ArcPy, coverage alone, unlike previous quick testing
efforts, is insufficient to ensure a useful regression test.  Because
coverage only considers the Python behavior of ArcPy (since we do not
have access to the source for the ArcGIS engine), it may group
behaviors that are not similar together.  We added the ability to
combine coverage preservation with preservation of all ArcPy messages
indicating a successful GIS engine operation, after abstracting away
such details as the runtime of the operation, and so forth.

However, just producing these coverage-and-engine-behavior preserving
standalone tests is not sufficient for good version comparison, since
standalone test cases as produced only check for properties defined in TSTL.  An
additional option was added to the standalone test generator, allowing
it to record the actual return values of all calls, the set of
exceptions thrown, the success/failure messages from the ArcPy engine, and so forth to more precisely record a test's
behavior on an ArcPy version.


\subsection{Test Case Normalization and Generalization}

Understanding ArcPy failures without delta-debugging \cite{DD} to reduce the
test cases to a readable size is, essentially, impossible \cite{MinUnit}.  The
typical test case, before reduction, is 600-2,000 steps long.  Even
after delta-debugging, however, once test cases are more
comprehensible in size, and contain no purely extraneous steps,
understanding ArcPy failures is difficult.

To address this problem, as well as other issues (some of which, such
as triaging large numbers of failing tests, are not at present
problems for ArcPy testing), we developed an algorithm to
\emph{normalize} test cases \cite{ICSTnorm}.  This algorithm applies a series of term
rewriting rules to reduce the number of variables in a test case,
reduce the complexity of API calls made, and other modifications.  In
the case of ArcPy, this often also further reduces test case length
beyond what standard delta-debugging can achieve.  For example, of the
first five crashes detected (some of which turned out to be variations
of one underlying problem), normalization reduced the length of the
delta-debugged test case from 19 to 11 steps, from 18 to 14 steps,
from 27 to 20 steps, from 20 to 16 steps, and from 10 to 9 steps.  In
the last case, the one step removed gave important information about
the problem.

In addition to normalization, we found it essential to apply
generalization \cite{SmartCheck,ICSTnorm} to test cases.  This
algorithm, also produced to aid ArcPy testing, is in a sense the
opposite of normalization.  Normalization takes many tests that differ
in unimportant ways and converts them to one, simple, sometimes
canonical (one test per fault) form.  Generalization takes a single
test, and produces annotations that describe how the test could be
modified while retaining the property of interest --- e.g.,
generalization answers such questions as:

\begin{itemize}
\item Could this
constant value be different, and the test still fail?
\item Could these two API calls be swapped in their position in the
  test case, and the test still fail?
\item Could this freshly created object replace this complex,
  much-modified object in this API call, and the test still fail?
\end{itemize}

Together, normalization and generalization have greatly aided our
understanding of complex ArcPy test cases:  normalization provides a
standard structure for failures, and makes constant values as small as
possible. Generalization tells us when these values can be changed,
without altering the disposition of the test.  The faults described in
this paper are all presented as normalized and generalized test cases.

\subsection{Deep State Startup Testing}

One of the most challenging problems in testing ArcPy is that
individual test operations may take much more time than in traditional
API-based testing.  In traditional API testing, it is assumed that
each step of execution takes at most a few seconds.  Performing a
complex GIS analysis such as a Buffer or Intersect, however, can require
many seconds, or even many minutes (the largest time for a single
operation during a test we have seen is almost 30 minutes).  If some faults depend
on the interactions of multiple complex analyses, detecting them will
be extremely expensive.  Simply reaching system states that are due to
many successful operations is a rare achievement in pure random
testing.  For software where the state of the system is mostly a
matter of the contents of volatile memory, saving state is difficult
and many methods require access to source code \cite{ModelDriven}.  For ArcPy,
however, the most important system state is the changes made to
feature classes stored in non-volatile memory.

The ArcPy test harness already makes use of a set of ``seed'' files as
an initial state for testing.  There is no reason that all test
sequences must start from the same set of data, however.  The
flexibility that allows users to test ArcPy over their own data also
makes it possible to start testing from files resulting from previous
testing, without repeating the operations involved, trading storage
space (to hold the modified files) for testing runtime.  Simply
copying the workspace after each test completes and randomly selecting a
previously produced workspace to begin testing with is trivial to
implement.  However, this naive approach has two problems:

\begin{enumerate}
\item First, if a fault is detected using a complex starting state,
  the fault may be due to the actions that produced that state, not
  test actions that follow in the ``new'' test.  Providing only the
  seed files and the actions may make it very hard to understand and
  debug the fault.

\item Second, how do we determine which workspaces to store?  In many
  cases, due to the high probability of operation failure, the files
  involved will be unchanged, or only changed in small, uninteresting
  ways, after hundreds of actions.  Storing so many copies of the
  feature classes used in testing is highly inefficient.
\end{enumerate}

We propose combining our approach to producing regression tests with an
analysis of file contents in order to 1) filter out which workspace states to
store and 2) capture the action sequence (a test prefix) that produces a given workspace.
This not only enables debugging (including test case reduction,
normalization, and generalization) of test cases that start from a
saved \emph{deep state}, but allows a workspace to be ``zipped'' for download by only
providing the action sequence to produce it.

Our efforts along these lines are preliminary.  However, we believe
that a thorough investigation of the possibilities of saving system
state after long test sequences and starting testing from complex
states is a promising direction for future research.  Recent work has
shown that test startup costs are very high, relative to the actual
tested behavior \cite{Bell}.

\section{Faults Discovered}

Thus far, our fault-finding process has focused on crashes.  Because
test cases that cause crashes stop the testing process, it is critical
to identify and avoid behavior that causes crashes before proceeding
to add further correctness properties.  Additionally, other than
hard-to-detect data corruption, crashes may be the most frustrating
faults for a developer to fix.  When ArcGIS or a standalone ArcPy
script causes a system crash, there is no readable error message, or
symptom in incorrect data to use in debugging the program.
Understanding system core dumps or analyzing the XML logs produced by
the ArcGIS engine is difficult for end users.  It would be ideal to
fix all ArcPy crashes by (at least) changing the library behavior to
issue an error message on invalid calls, but in the absence of bug
fixes, it is helpful to identify the root causes of various crashes
for users.  These test cases all involve calls that, as far as we
can tell, is not explicitly 

Figures \ref{fault1}-\ref{fault4} show four test cases that result in
an ArcPy crash (the Python interpreter stops functioning, terminating
the test prematurely).  If executed in ArcGIS, these fragments will
crash ArcGIS.  Because these test cases are 1-minimal \cite{DD},
normalized, and generalized, we are able to describe in some detail
the general sequence of actions resulting each crash.

\subsection{First Crash Fault}

ArcPy crashes when the feature class from which a layer is produced is
deleted, and the layer is used in a {\tt SelectLayer} call (this
version shows an attribute-based selection, but location selection
will cause the same problem): (Figure \ref{fault1}).  The underlying issue seems to be that
while operations on a deleted feature class properly notify a user the
feature class does not exist, ArcPy or ArcGIS does not track that
layers depending on that feature should also be deleted/invalidated
when the feature class is deleted.  Layers are not copies
of a feature class, but essentially new views of a feature class.
This means that when the underlying feature class is modified or
deleted, the view needs to be updated to reflect that change, and this
is not always correctly implemented.

\begin{figure}
{\scriptsize 
\begin{code}
shapefilelist0 = glob.glob("C:\\Arctmp\\*.shp")                                           \textcolor{black!60}{\# STEP 0}
\textcolor{black!60}{\#[}
shapefile0 = shapefilelist0 [0]                                           \textcolor{black!60}{\# STEP 1}
newlayer0 = "l1"                                                          \textcolor{black!60}{\# STEP 2}
\textcolor{black!60}{\#  or newlayer0 = "l2" }
\textcolor{black!60}{\#  or newlayer0 = "l3" }
\textcolor{black!60}{\#  swaps with steps 3 4 5 6 7}
\textcolor{black!60}{\#] (steps in [] can be in any order)}
\textcolor{black!60}{\#[}
featureclass0 = shapefile0                                                \textcolor{black!60}{\# STEP 3}
\textcolor{black!60}{\#  swaps with step 2}
fieldname0 = "newf1"                                                      \textcolor{black!60}{\# STEP 4}
\textcolor{black!60}{\#  or fieldname0 = "newf2" }
\textcolor{black!60}{\#  or fieldname0 = "newf3" }
\textcolor{black!60}{\#  swaps with steps 2 8}
selectiontype0 = "SWITCH\_SELECTION"                                       \textcolor{black!60}{\# STEP 5}
\textcolor{black!60}{\#  or selectiontype0 = "NEW\_SELECTION" }
\textcolor{black!60}{\#  or selectiontype0 = "ADD\_TO\_SELECTION" }
\textcolor{black!60}{\#  or selectiontype0 = "REMOVE\_FROM\_SELECTION"}
\textcolor{black!60}{\#  or selectiontype0 = "SUBSET\_SELECTION"}
\textcolor{black!60}{\#  or selectiontype0 = "CLEAR\_SELECTION"   }
\textcolor{black!60}{\#  swaps with steps 2 8}
op0 = ">"                                                                 \textcolor{black!60}{\# STEP 6}
\textcolor{black!60}{\#  or op0 = "<" }
\textcolor{black!60}{\#  swaps with steps 2 8}
val0 = "100"                                                              \textcolor{black!60}{\# STEP 7}
\textcolor{black!60}{\#  or val0 = "1000" }
\textcolor{black!60}{\#  swaps with steps 2 8}
\textcolor{black!60}{\#] (steps in [] can be in any order)}
arcpy.MakeFeatureLayer\_management(featureclass0, newlayer0)                                             \textcolor{black!60}{\# STEP 8}
\textcolor{black!60}{\#  swaps with steps 4 5 6 7}
arcpy.SelectLayerByAttribute\_management(newlayer0,selectiontype0,
   ' "'+fieldname0+'" '+op0+val0)                                         \textcolor{black!60}{\# STEP 9}
arcpy.Delete\_management(featureclass0)                                    \textcolor{black!60}{\# STEP 10}
arcpy.SelectLayerByAttribute\_management(newlayer0,selectiontype0,
   ' "'+ fieldname0+'" '+op0+val0)                                        \textcolor{black!60}{\# STEP 11}
\end{code}
}
\caption{Deleting a feature class does not invalidate or delete layers that depend on it.}
\label{fault1}
\end{figure}

\subsection{Second Crash Fault}

ArcPy crashes when asked to compute statistics (wth the {\tt FIRST} or
{\tt LAST} statistics types) over a field of a layer, when that field
has been deleted from the underlying feature class:  Figure
\ref{fault2}.  This is possibly related to the first crash fault:
ArcGIS again does not seem to properly propagate changes to an underlying
feature class to layers (which seem to be views) created on that feature class using a {\tt
  MakeFeatureLayer} call.

\begin{figure}
{\scriptsize 
\begin{code}
shapefilelist0 = sorted(glob.glob(arcpy.env.workspace + "\\*.shp"))                   \# STEP 0
\#[
shapefile0 = shapefilelist0 [0]                                                      \# STEP 1
newlayer0 = "l1"                                                                     \# STEP 2
\#  or newlayer0 = "l2" 
\#  or newlayer0 = "l3" 
\#  swaps with step 3
\#] (steps in [] can be in any order)
\#[
featureclass0 = shapefile0                                                           \# STEP 3
\#  swaps with step 2
classorlayer0 = newlayer0                                                            \# STEP 4
\#  swaps with steps 10 11 12
fieldtype0 = "DATE"                                                                  \# STEP 5
\#  or fieldtype0 = "TEXT" 
\#  or fieldtype0 = "FLOAT" 
\#  or fieldtype0 = "DOUBLE" 
\#  or fieldtype0 = "SHORT" 
\#  or fieldtype0 = "LONG" 
\#  swaps with steps 10 11 12
fieldname0 = "newf1"                                                                 \# STEP 6
\#  swaps with steps 10 11 12
op0 = ">"                                                                            \# STEP 7
\#  or op0 = "<" 
\#  or op0 = "<=" 
\#  or op0 = ">=" 
\#  or op0 = "=" 
\#  swaps with steps 10 11 12 14
val0 = "100"                                                                         \# STEP 8
\#  or val0 = "1000" 
\#  swaps with steps 10 11 12 14
stattable0 = arcpy.env.workspace + "\\stats.dbf"                                      \# STEP 9
\#  swaps with steps 10 11 12 14
\#] (steps in [] can be in any order)
\#[
fieldlist0 = arcpy.ListFields(featureclass0)                                         \# STEP 10
\#  swaps with steps 4 5 6 7 8 9 14
stattype0 = "FIRST"                                                                  \# STEP 11
\#  or stattype0 = "LAST" 
\#  swaps with steps 4 5 6 7 8 9
statfields0 = []                                                                     \# STEP 12
\#  swaps with steps 4 5 6 7 8 9
\#] (steps in [] can be in any order)
\#[
statfields0.append([fieldname0,stattype0])                                           \# STEP 13
arcpy.AddField\_management(featureclass0,fieldname0,fieldtype0); report()             \# STEP 14
\#  swaps with steps 7 8 9 10
\#] (steps in [] can be in any order)
fieldname0 = fieldlist0 [0].name \# STEP 15
arcpy.MakeFeatureLayer\_management(featureclass0,newlayer0,
   where\_clause=' "' + fieldname0 + '" ' + op0 + val0); report()                     \# STEP 16
fieldname0 = "newf1"                                                                 \# STEP 17
arcpy.DeleteField\_management(featureclass0,fieldname0); report()                     \# STEP 18
arcpy.Statistics\_analysis(classorlayer0,stattable0,statfields0); report()            \# STEP 19
\end{code}
}
\caption{Deleting a field then computing statistics on it causes a crash.}
\label{fault2}
\end{figure}

\begin{figure}
{\scriptsize 
\begin{code}
shapefilelist0 = sorted(glob.glob(arcpy.env.workspace + "\\*.shp"))                   \# STEP 0
shapefile0 = shapefilelist0 [0]                                                      \# STEP 1
featureclass0 = shapefile0                                                           \# STEP 2
\#[
classorlayer0 = featureclass0                                                        \# STEP 3
fieldtype0 = "DOUBLE"                                                                \# STEP 4
\#  or fieldtype0 = "TEXT"
\#  or fieldtype0 = "FLOAT"
\#  or fieldtype0 = "SHORT"
\#  or fieldtype0 = "LONG"
\#  or fieldtype0 = "DATE"
\#  swaps with step 6
fieldname0 = "newf1"                                                                 \# STEP 5
\#  or fieldname0 = "newf3"
\#  swaps with steps 6 8
\#] (steps in [] can be in any order)
\#[
insertcursor0 = arcpy.InsertCursor(classorlayer0)                                    \# STEP 6
\#  swaps with steps 4 5
arcpy.AddField\_management(featureclass0,fieldname0,fieldtype0); report()             \# STEP 7
\#] (steps in [] can be in any order)
fieldname0 = "newf2"                                                                 \# STEP 8
\#  or fieldname0 = "newf3"
\#  swaps with step 5
arcpy.AddField\_management(featureclass0,fieldname0,fieldtype0); report()             \# STEP 9
insertcursor0 = arcpy.InsertCursor(classorlayer0)                                    \# STEP 10
\end{code}
}
\caption{Creating two insert cursors on a layer, after adding two fields to the feature class underlying it, causes a crash.}
\label{fault3}
\end{figure}

\begin{figure}
{\scriptsize
\begin{code}
shapefilelist0 = sorted(glob.glob(arcpy.env.workspace + "\\*.shp"))                   \# STEP 0
\#[
shapefile0 = shapefilelist0 [0]                                                      \# STEP 1
newlayer0 = "l1"                                                                     \# STEP 2
\#  or newlayer0 = "l2" 
\#  swaps with step 3
\#] (steps in [] can be in any order)
\#[
featureclass0 = shapefile0                                                           \# STEP 3
\#  swaps with step 2
classorlayer0 = newlayer0                                                            \# STEP 4
\#  or classorlayer0 = featureclass0 
\#  or (
\#      newlayer0 = "l3"  ;
\#      classorlayer0 = newlayer0 
\#     )
fieldtype0 = "FLOAT"                                                                 \# STEP 5
\#  or fieldtype0 = "DOUBLE" 
\#  or fieldtype0 = "SHORT" 
\#  or fieldtype0 = "LONG" 
fieldname0 = "newf1"                                                                 \# STEP 6
\#  or fieldname0 = "newf3" 
\#  swaps with step 11
op0 = ">"                                                                            \# STEP 7
\#  or op0 = "<" 
\#  or op0 = "<=" 
\#  or op0 = ">=" 
\#  or op0 = "=" 
val0 = "10"                                                                          \# STEP 8
\#  or val0 = "20" 
\#  or val0 = "30" 
\#  or val0 = "100" 
\#  or val0 = "1000" 
\#] (steps in [] can be in any order)
\#[
whereclause0 = '"' + fieldname0 + '" ' + op0 + str(val0)                             \# STEP 9
arcpy.AddField\_management(featureclass0,fieldname0,fieldtype0); report()             \# STEP 10
\#] (steps in [] can be in any order)
\#[
fieldname0 = "newf2"                                                                 \# STEP 11
\#  or fieldname0 = "newf3" 
\#  swaps with step 6
arcpy.MakeFeatureLayer\_management(featureclass0,newlayer0,where\_clause=whereclause0); report()   \# STEP 12
\#] (steps in [] can be in any order)
searchcursor0 = arcpy.SearchCursor(classorlayer0,whereclause0)                       \# STEP 13
\#  or searchcursor0 = arcpy.SearchCursor(classorlayer0) 
arcpy.AddField\_management(featureclass0,fieldname0,fieldtype0); report()             \# STEP 14
\#  or (
\#      fieldtype0 = "TEXT"  ;
\#      arcpy.AddField\_management(featureclass0,fieldname0,fieldtype0); report() 
\#     )
\#  or (
\#      fieldtype0 = "DATE"  ;
\#      arcpy.AddField\_management(featureclass0,fieldname0,fieldtype0); report() 
\#     )
srow0 = searchcursor0.next()                                                         \# STEP 15
\#  or srow1 = searchcursor0.next() 
\#  or srow2 = searchcursor0.next()
\end{code}
}
\caption{Advancing a search cursor on a layer, after adding a field to
  the underlying feature class twice, causes a crash.}
\label{fault4}
\end{figure}
%\subsection{Deep State Startup Testing}

One of the most challenging problems in testing ArcPy is that
individual test operations may take much more time than in traditional
API-based testing.  In traditional API testing, it is assumed that
each step of execution takes at most a few seconds.  Performing a
complex GIS analysis such as a Buffer or Intersect, however, can require
many seconds, or even many minutes (the largest time for a single
operation during a test we have seen is almost 30 minutes).  If some faults depend
on the interactions of multiple complex analyses, detecting them will
be extremely expensive.  Simply reaching system states that are due to
many successful operations is a rare achievement in pure random
testing.  For software where the state of the system is mostly a
matter of the contents of volatile memory, saving state is difficult
and many methods require access to source code \cite{ModelDriven}.  For ArcPy,
however, the most important system state is the changes made to
feature classes stored in non-volatile memory.

The ArcPy test harness already makes use of a set of ``seed'' files as
an initial state for testing.  There is no reason that all test
sequences must start from the same set of data, however.  The
flexibility that allows users to test ArcPy over their own data also
makes it possible to start testing from files resulting from previous
testing, without repeating the operations involved, trading storage
space (to hold the modified files) for testing runtime.  Simply
copying the workspace after each test completes and randomly selecting a
previously produced workspace to begin testing with is trivial to
implement.  However, this naive approach has two problems:

\begin{enumerate}
\item First, if a fault is detected using a complex starting state,
  the fault may be due to the actions that produced that state, not
  test actions that follow in the ``new'' test.  Providing only the
  seed files and the actions may make it very hard to understand and
  debug the fault.

\item Second, how do we determine which workspaces to store?  In many
  cases, due to the high probability of operation failure, the files
  involved will be unchanged, or only changed in small, uninteresting
  ways, after hundreds of actions.  Storing so many copies of the
  feature classes used in testing is highly inefficient.
\end{enumerate}

We propose combining our approach to producing regression tests with an
analysis of file contents in order to 1) filter out which workspace states to
store and 2) capture the action sequence (a test prefix) that produces a given workspace.
This not only enables debugging (including test case reduction,
normalization, and generalization) of test cases that start from a
saved \emph{deep state}, but allows a workspace to be ``zipped'' for download by only
providing the action sequence to produce it.

Our efforts along these lines are preliminary.  However, we believe
that a thorough investigation of the possibilities of saving system
state after long test sequences and starting testing from complex
states is a promising direction for future research.  Recent work has
shown that test startup costs are very high, relative to the actual
tested behavior \cite{Bell}.
\section{Work In-Progress}
\label{future}

While not all of the language and tool changes discussed above have
been introduced in the release version of TSTL, all of these features are at
minimum undergoing testing prior to release.  The features discussed
in this section are in the early, exploratory, design phase.

\subsection{More Complete Correctness Checks}
\label{futureoracle}

One key omission in the ArcPy test harness is the lack of correctness
properties.  This version only fails tests when 1) the system crashes
or 2) an unexpected exception is raised.  Determining the proper
results for complex operation sequences on arbitrary data is
difficult.  To support conditional metamorphic properties \cite{MetaTest}, we plan to
add a TSTL feature to check a property conditional on the set of
actions called.  For example, if fields are never added or deleted, the number of
fields for each feature class should remain constant.  Unfortunately,
the set of properties that are independent of the underlying data used
may be relatively small, or the conditions under which a property
should hold so limited it may be hard to produce highly effective
testing based just on conventional correctness properties or limited
metamorphic testing.

\subsection{Differential Testing within One Version}
\label{sec:reftest}

One way to work around the difficulties of specifying properties of
complex systems is to use differential testing \cite{Differential,ICSEDiff}:  if two systems
implement the same functionality, and are given the same inputs, they
should produce the same output.  This approach can also be applied
within a single version of ArcPy:  because ArcGIS supports more than
one storage format (shapefiles, file geodatabases, and personal
geodatabases), the same data can be stored in multiple formats.  Using
TSTL's support for reference pools, the same operations can be
performed on the same data stored in different formats, and the
results can be checked, including success or failure of ArcPy calls
and spatial properties and field values of the manipulated data.

\subsection{Testing Multiprocessing}

ArcPy scripts can use Python's multiprocessing features to exploit
multicore machines and gain efficiency.  However, this can expose
ArcGIS or ArcPy faults, especially problems with the (not well
documented) locking system.  TSTL can use multiprocessing in actions
(since actions can call arbitrary Python code), but does not include
support for running multiple threads of execution at the TSTL level.
One problem with testing multiprocessing use of ArcPy is that
concurrency is much more likely to produce nondeterministic test
behavior than our existing testing.

%\subsection{Optimizations for Testing}

\subsection{Parallel Testing}

While multiprocessing on a single machine can introduce additional
problems with test determinism and locking interactions, running tests
in parallel using either cloud computing or virtualization on a
single multicore machine could improve test throughput.   This is
possibly more important for ArcPy than in many other testing
scenarios, given the unusually high cost of even single API calls.
\section{The ArcPy TSTL Test Harness}
\label{harness}

\begin{figure}
{\scriptsize
\begin{code}
@import shutil, os, glob, arcpy, exceptions, gc
@from arcpy import ExecuteError
<@
def cleanupFiles():
    gc.collect() \# Get rid of cursors
    for l in ["l1","l2","l3"]:
    	arcpy.Delete\_management(l)
    
    for f in glob.glob("C:\\Arctmp\\*"):
        try:
            shutil.rmtree(f)
        except:
            print "UNABLE TO REMOVE:",f
    for i in xrange(0,1000000): \# Find a workspace without a lock
        new\_workspace = "C:\\Arctmp\\workspace." + str(i)
        if not os.path.exists(new\_workspace):
            break             
    shutil.copytree("C:\\Arcbase",new\_workspace)
    arcpy.env.workspace = new\_workspace
    print sorted(glob.glob(arcpy.env.workspace + "\\*.shp")),
    print sorted(glob.glob(arcpy.env.workspace + "\\*.lyr")),
    print sorted(glob.glob(arcpy.env.workspace + "\\*.gdb"))


def fcsInGdb(gdb):
    old\_workspace = arcpy.env.workspace
    arcpy.env.workspace = gdb
    fcs = []
    for fds in arcpy.ListDatasets('','feature') + ['']:
        for fc in arcpy.ListFeatureClasses('','',fds):
            fcs.append(os.path.join(gdb,fds,fc))
    arcpy.env.workspace = old\_workspace
    return fcs

def report():
    print arcpy.GetMessages()
@>

pool: <shapefilelist> 2
pool: <shapefile> 2 CONST

pool: <gdbfilelist> 2
pool: <gdbfile> 2 CONST

pool: <gdbfeatureclasslist> 2

pool: <featureclass> 4 CONST
pool: <classorlayer> 4 CONST

pool: <classorlayerlist> 2

pool: <spatialref> 2

pool: <prjfilelist> 2
pool: <prjfile> 2 CONST

pool: <transformlist> 2
pool: <transform> 2 CONST

pool: <newlayer> 2 CONST

pool: <layerlist> 2
pool: <layer> 2

pool: <fieldname> 2 CONST
pool: <fieldtype> 2 CONST
pool: <fieldlist> 2
pool: <fieldnamelist> 2

pool: <stattype> 2 CONST
pool: <statfields> 2

pool: <dist> 2 CONST

pool: <sorttype> 2 CONST
pool: <spatialsort> 2 CONST

pool: <sort> 1
pool: <sortlist> 2

pool: <joinattributes> 2

pool: <overlaptype> 2 CONST
pool: <selectiontype> 2 CONST
pool: <op> 2 CONST
pool: <val> 2 CONST
pool: <whereclause> 2 CONST

pool: <errtable> 1 CONST
pool: <polytable> 1 CONST
pool: <stattable> 1 CONST

pool: <insertcursor> 3
pool: <searchcursor> 3
pool: <updatecursor> 3

pool: <irow> 3
pool: <srow> 3
pool: <urow> 3

init: cleanupFiles()

log: 1 arcpy.GetMessages()
\end{code}
}
\caption{ArcPy TSTL test harness definition preamble (pools, functions, logging).}
\label{preamble}
\end{figure}

\begin{figure}
{\scriptsize
\begin{code}
<gdbfilelist> := sorted(glob.glob(arcpy.env.workspace + "\\*.gdb"))
len(<gdbfilelist,1>) >= 1 -> <gdbfile> := <gdbfilelist> [0]
<gdbfilelist> = <gdbfilelist> [1:]

<shapefilelist> := sorted(glob.glob(arcpy.env.workspace + "\\*.shp"))
len(<shapefilelist,1>) >= 1 -> <shapefile> := <shapefilelist> [0]
<shapefilelist> = <shapefilelist> [1:]

<layerfile> := arcpy.env.workspace + <["\\new1.lyr", "\\new2.lyr", "\\new3.lyr"]>

<shapefile> := arcpy.env.workspace + <["\\new1.shp", "\\new2.shp", "\\new3.shp"]>

<prjfilelist> := sorted(glob.glob
   ("C:\\Program Files (x86)\\ArcGIS\\Desktop10.3\\Reference Systems\\*.prj"))
len(<prjfilelist,1>) >= 1 -> <prjfile> := <prjfilelist> [0]
<prjfilelist> = <prjfilelist> [1:]

<transformlist> := arcpy.ListTransformations(<spatialref>,<spatialref>)
<transformlist> = <transformlist> [1:]
len(<transformlist,1>) >= 1 -> <transform> := <transformlist> [0]

<newlayer> := <["l1", "l2", "l3"]>

<spatialref> := arcpy.SpatialReference(<prjfile>)

<gdbfeatureclasslist> := fcsInGdb(<gdbfile>)
<gdbfeatureclasslist> = <gdbfeatureclasslist>[1:]

<featureclass> := <shapefile>
len(<gdbfeatureclasslist,1>) >= 1 -> <featureclass> := <gdbfeatureclasslist>[0]

<classorlayer> := <featureclass>
<classorlayer> := <newlayer>

<classorlayerlist> := []
<classorlayerlist>.append(<classorlayer>)

<fieldtype> := <["TEXT", "FLOAT", "DOUBLE", "SHORT", "LONG", "DATE"]>

<fieldname> := <["newf1", "newf2", "newf3"]>

<dist> := <["100 Feet", "500 Feet", "1000 Feet", "1 Mile", "2 Miles">]

<joinattributes> := <["ALL", "NO\_FID", "ONLY\_FID"]>

<overlaptype> := <["INTERSECT", "CONTAINS", "COMPLETELY\_CONTAINS", "WITHIN",
   "SHARE\_A\_LINE\_SEGMENT\_WITH", "CROSSED\_BY\_THE\_OUTLINE\_OF"]>

<selectiontype> := <["NEW\_SELECTION", "ADD\_TO\_SELECTION", "REMOVE\_FROM\_SELECTION",
   "SUBSET\_SELECTION", "SWITCH\_SELECTION", "CLEAR\_SELECTION"]>

<sorttype> := <["ASCENDING", "DESCENDING">]

<spatialsort> := <["UR", "UL", "LR", "LL", "PEANO"]>

<sort> := [<fieldname>,<sorttype>]
<sortlist> := []
<sortlist>.append(<sort>)

\{IOError\} <insertcursor> := arcpy.InsertCursor(<classorlayer>)
\{IOError\} <insertcursor> := arcpy.InsertCursor(<classorlayer>,<spatialref>)
\{IOError\} <searchcursor> := arcpy.SearchCursor(<classorlayer>)
\{IOError,exceptions.RuntimeError\} <searchcursor> := arcpy.SearchCursor
   (<classorlayer>,<whereclause>)
\{IOError,exceptions.RuntimeError\} <searchcursor> := arcpy.SearchCursor
   (<classorlayer>,<whereclause>,<spatialref>)
\{IOError\} <updatecursor> := arcpy.UpdateCursor(<classorlayer>)
\{IOError\} <updatecursor> := arcpy.UpdateCursor(<classorlayer>,<spatialref>)

<irow> := <insertcursor>.newRow()
\{exceptions.RuntimeError\} <insertcursor>.insertRow(<irow>)

<irow> := <insertcursor>.next()
<urow> := <updatecursor>.next()
<srow> := <searchcursor>.next()

\{exceptions.RuntimeError\} <val> := <irow>.getValue(<fieldname>)
\{exceptions.RuntimeError\} <val> := <srow>.getValue(<fieldname>)
\{exceptions.RuntimeError\} <val> := <urow>.getValue(<fieldname>)

\{exceptions.RuntimeError\} <irow>.setValue(<fieldname>,<val>)

\{exceptions.RuntimeError\} <urow>.setNull(<fieldname>)
\{exceptions.RuntimeError\} <urow>.setValue(<fieldname>,<val>)
\{exceptions.RuntimeError\} <updatecursor>.deleteRow(<urow>)
\{exceptions.RuntimeError\} <updatecursor>.updateRow(<urow>)

<op> := <[">", "<", "<=", ">=", "=", "!=">]  

<val> := <["10", "20", "30", "100", "1000">]

<whereclause> := '"' + <fieldname> + '" ' + <op> + str(<val>)

<whereclause> := <whereclause> + ' AND ' + <whereclause>

<whereclause> := <whereclause> + ' OR ' +  <whereclause>

<whereclause> := 'NOT' + <whereclause>

<errtable> := arcpy.env.workspace + "\\geomerr.dbf"

<polytable> := arcpy.env.workspace + "\\polyneig.dbf"

<stattable> := arcpy.env.workspace + "\\stats.dbf"

\{IOError\} <fieldlist> := arcpy.ListFields(<classorlayer>)
len(<fieldlist,1>) >= 1 -> <fieldname> := <fieldlist> [0].name
<fieldlist> = <fieldlist> [1:]

<fieldnamelist> := []
<fieldnamelist>.append(<fieldname>)

<stattype> := <["SUM", "MEAN", "MIN", "MAX", "RANGE", "STD", "COUNT", "FIRST", "LAST"]>

<statfields> := []
<statfields>.append([<fieldname>,<stattype>])
\end{code}
}
\caption{ArcPy TSTL test harness actions, part 1.}
\label{actions1}
\end{figure}

\begin{figure}
{\scriptsize
\begin{code}
\{ExecuteError\} arcpy.MakeFeatureLayer\_management(<featureclass>,<newlayer>); report()

\{ExecuteError\} arcpy.MakeFeatureLayer\_management(<featureclass>,<newlayer>,
   where\_clause=<whereclause>); report()

\{ExecuteError\} arcpy.Project\_management(<featureclass>,<featureclass>,<spatialref>,
   <transform>); report()

\{ExecuteError\} arcpy.AddField\_management(<featureclass>,<fieldname>,<fieldtype>);
   report()

\{ExecuteError\} arcpy.DeleteField\_management(<featureclass>,<fieldname>); report()

\{ExecuteError\} arcpy.Buffer\_analysis(<classorlayer>,<featureclass>,<dist>); report()

\{ExecuteError\} arcpy.Buffer\_analysis(<classorlayer>,<featureclass>,<dist>,
   dissolve\_option="ALL"); report()

\{ExecuteError\} arcpy.Buffer\_analysis(<classorlayer>,<featureclass>,<dist>,
   dissolve\_option="LIST",dissolve\_field=<fieldnamelist>); report()

\{ExecuteError\} arcpy.Erase\_analysis(<classorlayer>,<classorlayer>,<featureclass>);
   report()

\{ExecuteError\} arcpy.Erase\_analysis(<classorlayer>,<classorlayer>,<featureclass>,
   cluster\_tolerance=<dist>); report()

\{ExecuteError\} arcpy.Intersect\_analysis(<classorlayerlist>,<featureclass>); report()

\{ExecuteError\} arcpy.Intersect\_analysis(<classorlayerlist>,<featureclass>,
   join\_attributes=<joinattributes>); report()

\{ExecuteError\} arcpy.Intersect\_analysis(<classorlayerlist>,<featureclass>,
   cluster\_tolerance=<dist>); report()

\{ExecuteError\} arcpy.Intersect\_analysis(<classorlayerlist>,<featureclass>,
   join\_attributes=<joinattributes>,cluster\_tolerance=<dist>); report()

\{ExecuteError\} arcpy.Union\_analysis(<classorlayerlist>,<featureclass>); report()

\{ExecuteError\} arcpy.Union\_analysis(<classorlayerlist>,<featureclass>,
   join\_attributes=<joinattributes>); report()

\{ExecuteError\} arcpy.Union\_analysis(<classorlayerlist>,<featureclass>,
   cluster\_tolerance=<dist>); report()

\{ExecuteError\} arcpy.Union\_analysis(<classorlayerlist>,<featureclass>,
   join\_attributes=<joinattributes>,cluster\_tolerance=<dist>); report()

\{ExecuteError\} arcpy.SpatialJoin\_analysis(<classorlayer>,<classorlayer>,
   <featureclass>); report()

\{ExecuteError\} arcpy.SymDiff\_analysis(<classorlayer>,<classorlayer>,<featureclass>);
   report()

\{ExecuteError\} arcpy.SymDiff\_analysis(<classorlayer>,<classorlayer>,<featureclass>,
   join\_attributes=<joinattributes>); report()

\{ExecuteError\} arcpy.SymDiff\_analysis(<classorlayer>,<classorlayer>,<featureclass>,
   join\_attributes=<joinattributes>,cluster\_tolerance=<dist>); report()

\{ExecuteError\} arcpy.PolygonNeighbors\_analysis(<classorlayer>,~<polytable>); report()

\{ExecuteError\} arcpy.Statistics\_analysis(<classorlayer>,~<stattable>,<statfields>);
   report()

\{ExecuteError\} arcpy.SelectLayerByLocation\_management(<newlayer>,
   select\_features=<newlayer>,overlap\_type=<overlaptype>)

\{ExecuteError\} arcpy.SelectLayerByLocation\_management(<newlayer>,
   select\_features=<newlayer>,overlap\_type=<overlaptype>,search\_distance=<dist>)

\{ExecuteError\} arcpy.SelectLayerByLocation\_management(<newlayer>,
   select\_features=<newlayer>,overlap\_type=<overlaptype>,search\_distance=<dist>,
   selection\_type=<selectiontype>)

\{ExecuteError\} arcpy.SelectLayerByAttribute\_management(<newlayer>,
   selection\_type=<selectiontype>,where\_clause=<whereclause>)

\{ExecuteError\} arcpy.Select\_analysis(<classorlayer>,<featureclass>,
   where\_clause=<whereclause>)

\{ExecuteError\} arcpy.CopyFeatures\_management(<featureclass>,<featureclass>); report()

\{ExecuteError\} arcpy.Sort\_management(<featureclass>,<featureclass>,<sortlist>);
   report()

\{ExecuteError\} arcpy.Sort\_management(<featureclass>,<featureclass>,<sortlist>,
   <spatialsort>); report()

\{ExecuteError\} arcpy.Sort\_management(<featureclass>,<featureclass>,
   [["Shape",<sorttype>]],<spatialsort>); report()

\{ExecuteError\} arcpy.CheckGeometry\_management(<classorlayer>,~<errtable>); report()

\{ExecuteError\} arcpy.CheckGeometry\_management(<classorlayerlist>,~<errtable>); report()

\{ExecuteError\} arcpy.Delete\_management(<featureclass>); report()
\end{code}
}
\caption{ArcPy TSTL test harness definition actions, part 2.}
\label{actions2}
\end{figure}



Figures \ref{preamble}-\ref{actions2} show a
version of the actual ArcPy test harness.  This version can reproduce the faults
described in this paper, though in practice some faults are easier to
detect than others, and for practical testing it is best to disable,
e.g., the various {\tt Delete} calls and to use guards to prevent all modifications of
layers or classes on which cursors are active.  Compiled to Python,
this harness defines nearly 2,000 actions, and the standalone
interface is nearly 60KLOC.  Using the harness involves first
compiling it to a Python class, then loading a test case generator
such as the random tester provided with TSTL into a Python environment
that has access to ArcPy.  For our experiments, we compiled the TSTL
using the Cygwin Python installation (for easy command-line access) but ran
tests in the IDLE environment installed with ArcGIS.
Running tests involves no complications beyond modifying parameters of
the random tester, if desired (setting a time limit for testing, the
length of test cases \cite{ASE08}, and whether to search for failures or produce
coverage regression tests, for example).  In most cases, the random tester
eventually terminates abnormally, and the test case causing the crash
is stored in a file in the ArcPy harness directory.  For regression
generation, the tool produces files of the same format to obtain
coverage of ArcPy code.

Figure \ref{preamble} contains the small amount of code needed to
prepare a temporary workspace for each test sequence.  This code
handles deleting any live cursors (the pool variables are guaranteed
to be deleted before each test, and garbage collection ensures the
cursors are deactivated), removing any layers created on feature
classes, and then setting the environment.  The system will scan for a
``free'' environment location, to handle cases where locks prevent
re-using an old directory\footnote{The problem of locked directories
  seldom surfaces, now that layers are deleted before each test, but
  may be needed when saving states.}.  The other utility functions
allow discovering the feature classes in a geodatabase and produce a
report on screen when a complex ArcGIS operation completes
successfully.

Figure \ref{actions1} shows the actions that create
input parameters for ArcGIS engine calls, primarily.  Many of these
simply pick some numeric or string constant (and the sets of constants
could be expanded, at the cost of more expensive normalization and
generalization).  Others, such as SQL query generation, are more
complex, with recursive expansion to theoretically unlimited query
length.  

Finally, Figure \ref{actions2} shows the TSTL for the actual
ArcGIS toolbox calls.  So far, this includes only a small subset of
the functions defined in the Management and Analysis toolboxes.  Note
that after each call there is a call to {\tt report}.  In pure random
testing, successful calls are infrequent enough that it is useful
for the user to see the ArcGIS messages produced by successful calls.
By turning on logging, the user can also see unsuccessful call
messages, but this tends to produce an overwhelming amount of output.

One critical design decision taken early in the development of this
harness was that there is no explicit definition of the starting data
used for testing.  Any data stored in the {\tt Arcbase} directory can
be used, and the harness supports both shapefiles and file geodatabases.
This has two purposes:  first, testing can be customized to use any
starting data, including a user's own shapefiles.  Second, this makes
it easy to implement deep state testing, since no assumptions are made
about the structure or number of data files used.  The lack of
dependence on specific files is implemented by having the test harness
use Python's {\tt glob} function to collect all files of a given
extension in a directory in a list, then chose an item from the list.
In order to make sure that behavior is deterministic (up to the choice
of actual data files), the glob results are sorted.  This approach
does bias random testing to using files earlier in alphabetical order,
but we do not expect base testing data to include a large number of
files.

The key driving requirements for this harness design are given in the
title of this paper:  the test harness must be \emph{extensible} and
it must be \emph{usable}.

Because ArcPy is large and complex, the effort to produce a
complete test harness and more effective specification of correctness
is a long-term effort, and may be carried out by other users:  the
harness therefore must be \emph{extensible}.  The
harness should also be suitable for use by ArcPy developers testing
how their own extensions to ArcPy interact with the base ArcPy API.
Adding new API calls to the harness should be easy:  we have defined
the pools for basic ArcPy object types and provided tools that should
handle much more complex test cases.  The decision discussed above to
make the test harness independent of specific data files is a good
example of our emphasis on extensibility.

The idea that other ArcPy developers, testing researchers, and perhaps
(end-user) software developers in other fields looking for a large
scale example of how to test systems with TSTL, will need to read and
modify the code also drives our emphasis on a truly usable system.
Usability has a more direct impact on the modifications to the TSTL
language and tools than on the test harness itself, but the test
harness is developed in the context of the improvements to the TSTL
ecosystem, such as standalone tests and normalization and
generalization.  However, some usability choices are decisions about
how to write the harness.  Consider the calls to {\tt
  SelectLayerByLocation\_management} shown in Figure \ref{actions2}.
The three lines of code could be more concisely expressed in one line using the {\tt <,[} construct:

\begin{code}
\{ExecuteError\} arcpy.SelectLayerByLocation\_management(
   <newlayer>,select\_features=<newlayer>,overlap\_type=
   <overlaptype><,[,search\_distance=<dist>,,],>
   <,[,selection\_type=<selectiontype>,,],>)
\end{code}

However, this is difficult to read, even for the TSTL developer, and
seems likely to discourage GIS developers trying to understand and
extend the harness.  There is a tension between, on the one hand, concise, abstracted
expression of all possible test actions and, on the other hand,
concrete readable connection between test actions and the lines of code that
appear in real ArcPy scripts.  We aim to stay closer to concrete
forms, even at some cost in increased length for the harness.  An
interesting observation is that this preference on the one hand makes
changing the harness more difficult --- to change the name of an API
call requires multiple edits if multiple parameter arities are used;
on the other hand, changing the code using {\tt <,[} requires
understanding the meaning of the code, each time, which resembles the
difficulty of altering a complex regular expression \cite{RegExp}.  In
this case, the best solution might be to add a specialized construct
to TSTL to handle optional elements of an action --- despite the fact
that the {\tt <,[} construct can express optional elements easily.
Even with such an option, it may be easier for developers to read and
understand the implications of individual calls, however, if they are
written out in the harness, rather than combinatorally generated by
the TSTL compiler.

In the long run, these issues are as complex as the questions of
abstraction vs. ease-of-understanding that have concerned designers
and users of programming languages since early in the history of
computer science.  As test definition comes closer to a specialized
kind of declarative programming, our understanding of the tradeoffs
will likely improve.
\section{Conclusions and Future Work}

This paper presents a set of tools, part of the TSTL \cite{tstlsttt}
testing language and tool suite, for letting users make the most of the
tests the tool generates.  In addition to standard
replay, regression, and minimization, TSTL implements some powerful
new techniques from the recent literature for manipulating tests
\cite{OneTest,slippage}.

As future work, we plan to continue to develop TSTL's tools for
working with tests.  Some improvments are simple:  for instance, the TSTL
random tester currently provides simple fault localization over the
tests generated during a run (if there are any failures)
\cite{Tarantula}.  However, the regression tool does not yet provide
the same functionality for a suite of stored tests.  More importantly,
we plan to continue to use TSTL as a platform for experimenting with
and making available novel methods for making use of automatically
generated tests, including methods for composing and de-composing
tests and generating information from tests that can be used to guide
future testing.


% BibTeX users please use one of
%\bibliographystyle{spbasic}      % basic style, author-year citations
\bibliographystyle{spmpsci}      % mathematics and physical sciences
%\bibliographystyle{spphys}       % APS-like style for physics
\bibliography{bibliography}   % name your BibTeX data base

% Non-BibTeX users please use
%\begin{thebibliography}{}
%
% and use \bibitem to create references. Consult the Instructions
% for authors for reference list style.
%
%\bibitem{RefJ}
% Format for Journal Reference
%Author, Article title, Journal, Volume, page numbers (year)
% Format for books
%\bibitem{RefB}
%Author, Book title, page numbers. Publisher, place (year)
% etc
%\end{thebibliography}

\end{document}
% end of file template.tex

