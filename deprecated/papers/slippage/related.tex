\section{Related Work}

Chen et al. introduced the idea of slippage in the course of
describing efforts to automatically detect different faults in a large
set of failing test cases \cite{PLDI13}.  Hughes et
al. \cite{FindMoreBugs} proposed a modification of QuickCheck
to avoid re-producing known bugs that (in theory)
could mitigate the problem of slippage, but is not directly comparable
to our approach.  The approach of Hughes et al. requires
interpretation of test components (e.g. method calls), and analysis of
patterns, while our approaches are purely algorithmic, with no
additional requirements beyond those of delta debugging itself
\cite{DD}.  It is not clear how best to apply such an approach
 to cases such as {\tt jsfunfuzz} where each component is not a
method call but essentially an arbitrary string, without significant
user effort to define abstractions of components.

There are also approaches that sidestep slippage by initially
producing short test sequences (e.g. recent work by Mao et
al. \cite{Mao}).  However, for many generation algorithms
longer sequences are essential for good fault detection \cite{ASE08,LongBetter}.